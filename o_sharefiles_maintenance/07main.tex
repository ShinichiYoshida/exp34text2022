サーバを運用あるいは管理するために必要な要素として,バックアップとレスト
アがある.これらは不測の事態への対処のために必要なものである.ネットワー
クサービスへの依存度の高まりによってサーバ上に保持するデータが大容量化す
ることが予想されるが,そのような場合バックアップのためには大容量のストレ
ージが必要となる.また,サービスを提供するサーバが複数ある場合には,ネッ
トワークを介してのストレージへのアクセスが有効な場合がある.さらに,サー
バなどを長期間運用するようなとき,サーバ機器の交換などによって機器の廃棄
を行うことがあるが,サーバで重要な情報や個人情報などを扱っていた場合は適
切な処置を行う必要がある.

\section{ストレージ}

コンピュータの主要な要素の一つである,外部記憶装置のことを storage (保存
場所) とよぶ.主記憶装置であるメインメモリよりも大きな容量(通常2〜4桁以
上)で,物質の磁化や物理変化を用いて記憶の保持に電源を必要としない.

\subsection{ストレージデバイスの種類}
コンピュータによく用いられるストレージデバイスには,下記のようなものがあ
る.

\begin{itemize}
 \item ハードディスクドライブ (HDD)
 \item ソリッドステートドライブ (SSD)
 \item 光学ディスクドライブ(CD/DVD)
 \item フラッシュメモリ
 \item フロッピーディスクドライブ (FDD)
\end{itemize}


さらに,一般のコンピュータ(デスクトップクライアント,ノート型コンピュー
タ,サーバ)では,接続形態や接続インタフェースによって記憶装置を分類できる.

%\begin{itemize}
%
%\item 接続形態
%\begin{itemize}
% \item 固定ディスク
% \item リムーバブルディスク
%\end{itemize}
%
%\item 接続インターフェース
%\begin{itemize}
% \item 内蔵型インターフェース
% \item 外部接続型インターフェース
%\end{itemize}
%
%\end{itemize}

\subsection*{接続形態による分類}

接続形態には,固定ディスクとリムーバブルディスクがある.

固定ディスクは,OS 起動中は常にデバイスの接続状態,接続場所が固定されて
おり,OS 稼働中に移動することはできないものである.OS やアプリケーション
など,運用に必要なものは一般に固定ディスクへインストールする.固定ディス
クには主に HDD や SSD が用いられる.

リムーバブルディスクは,OS が起動中であっても所定の手順を踏むことで取り外
すことができる.運用に必要でないデータや一時的なデータの保管先として使用
される.リムーバブルディスクには,光学ディスクやフラッシュメモリなどのほ
かに,固定ディスクとして用いられる HDD や SSD に外部接続用の機器が取り付
けられたものも使用されることがある.

\subsection*{接続インターフェースによる分類}

接続のためのインターフェースには,大きく分けて,内蔵型と外部接続型(外付)
に分けられる.

内蔵型には,シリアルATA,SAS があり,前者は一般に広く用い
られているものであり,後者は一部のサーバなどに用いられる規格である.シリ
アルインターフェースであり,現在最も速い規格の SATA 3.0 では 6Gbps 
(600MB/s) の転送速度まで対応できる.かつてはパラレル型が用いられており,
IDE,SCSI はそれぞれ,SATA,SAS の前身であるパラレルインターフェースであっ
たが,現在はほとんど用いられていない.通常のコンピュータは,SATA インター
フェースを4ポート以上備えており,HDD,SSD,光学ドライブなどを接続する.

外部接続ストレージ用のインターフェースとしては,E-SATA,USB,IEEE1394,
SCSI などがある.E-SATA は SATA を外部接続用に変更したものであり,原則固
定ディスク向けであり,規格上の上限は 3Gbps である.USB は Universal とい
う名称にあるように,ストレージ以外にも広く用いられるが,HDD や SSD,フラッ
シュメモリにも用いられる.USB 1.1 が 12Mbps,USB 2.0 が 480Mbps,USB 3.0 
が 5Gbps である.USB 3.0 は通常青色のインターフェースになっている.
IEEE1394 は,メーカによっては,Firewire,i.Link などの名称で呼ばれること
もあるが,同一規格である.400Mbps と 800Mbps の規格がある.E-SATA, USB,
IEEE1394 と異なり SCSI\footnote{スカジーと発音する} はパラレルインター
フェースであり,かつ一般型コンピュータの周辺機器接続規格として古くから存
在 (1986年) している.Macintosh や PC-9801 をはじめ広く用いられた.現在
でも,テープドライブなど一部ストレージで SCSI を使用するものがある.SAS 
は SCSI をシリアル化した規格である.


\subsection*{内蔵ディスクドライブの物理的規格}

ハードディスクや SSD,光学ドライブなどのうち,内蔵用のものについては,大
きさの規格が複数ある.コンピュータ側のドライブベイと同じ大きさのものでな
いと,固定はできない.小さな規格のドライブであれば,マウンタと呼ばれる仲
介器具を介することで,大きな規格のベイに設置できる.

\begin{description}
 \item[3.5インチ] デスクトップ型,サーバ型で広く使われており,最もたくさ
	    んの種類のものが製造されており,記憶容量あたりのコストも他の
	    ものに比べて安価である.
 \item[2.5インチ] ノート型,小型コンピュータ,および SSD 全般に用いられる.
 \item[5.25インチ] 光学ドライブなどで用いられる.
 \item[1.8インチ] 特殊形状の小型コンピュータに一部用いられる.
\end{description}


\subsection*{RAID}

RAID は Redundant Arrays of Inexpensive Disks の略称で,複数のハードディ
スクドライブを組み合わせて,1つのストレージデバイスを構成し,記憶領域の
信頼性の向上,アクセス速度の向上,大容量記憶装置の構成などを行う
\footnote{David A. Patterson, Garth Gibson, and Randy H. Katz: A Case
for Redundant Arrays of Inexpensive Disks (RAID). University of
California Berkley. 1988. ACM(アメリカ情報処理学会)の研究会 SIGMOD でカ
リフォルニア大学バークレイ校のチームにより発表されたものが起源.この論文
は,WWW 上に公開されている.}.
いくつか規格があり,RAID 0〜6 までがあり,その組み合わせの RAID 1+0 (10),
RAID 5+0 (50) などもしばしば用いる.なお,単に複数ディスクを順につなげて
一つにするだけの LVM などもあるが,RAID とは呼ばない\footnote{単なる製品
名や自称の RAID(の亜種),例えば,RAID -1 ,RAID 10,RAID 7,RAID-Z,
RAID-S などは一般には広まっていないので,RAID と呼ばない,または通常の 
RAID レベルの範囲内で呼称する方が無難である}.

NAS や外部接続ストレージでは,2台,4台,8台などの HDD を接続できるように
したシステムにおいて,しばしば RAID が用いられる.このようなシステムで
RAID を行う場合は,接続するコンピュータからは単一のストレージ(ハードディ
スク,あるいはネットワークファイルサーバ)として見える.

ソフトウェア RAID と呼ばれる,OS のソフトウェアによる RAID 機能を用いて
複数のハードディスクあるいはソリッドステードドライブを RAID として運用す
ることが可能である.Windows であればダイナミックボリュームと呼ばれる機能
で設定できる.Linux の場合,mdadm (mdraid コマンド) で RAID 設定が行える.
Linux の場合は,LVM (Logical Volume Manager:論理ボリュームマネージャ) 
と呼ばれる機能でもミラーリングなどを設定できる他,mdadm と LVM を組み合
わせて後から HDD を追加して容量を自在に変更できる(ただし,XFS などの容量
変化に対応したファイルシステムでフォーマットする必要がある)複雑なシステ
ムの構築も可能である\footnote{Drobo社の製品 Drobo が有名であり,
BeyondRAID と呼ばれる.また一般の RAID 製品では,QNAP 社のシリーズが有名
であり,数万円の個人・SOHO向けモデルから,数千万円のスーパーコンピュータ
向け製品まで様々な製品があり,内部は Linux で構築されている.}.

FreeBSD や Solaris では,zfs と呼ばれる高機能ファイルシステムが LVM や
ソフトウェア RAID にあたる機能を備えており,これを用いた RAID システムも
増えている.

\subsection{ストレージのネットワーク上での利用}

\subsection*{ファイル共有プロトコル}

ファイル共有はファイル転送と異なり,オペレーティングシステムのファイルシステムで規定する各種の属性,
ファイルアクセスの方法などに基づいてファイルのやりとりを行うものである.
利点としては,オペレーティングシステムからファイルを読み書きする際,
ローカルのディスクのファイルと,リモートのネットワークの向こうのファイルとで,
区別することなくファイルのアクセスが行えることである.
すなわち,リモートファイルもローカルファイルと同様に区別せずにシームレスにアクセスできることである.
欠点としては,ファイルシステムはOSに依存しているため,
異なるOS間でのファイル共有は特別なソフトウェアを用いる必要があるなど,手順が比較的複雑になることである.

現在,多く用いられるファイル共有の仕組みとしては,Windows などの
Microsoft 系 OS で用いられる SMB(Server Message Block または CIFS:Common Internet File System) と,
UNIX系 OS で標準的な NFS (Network File System)がある.
%ファイル共有プロトコルには,Windows で用いられる SMB (CIFS),UNIX系で一
%般的な NFS がある.
現在のUNIX系,MacOS X は,SMB プロトコルを扱えるため
OS依存が少なく,ファイルのやりとりができれば良い場面では,SMB が用いられ
ることが多くなっている.一方,NFS は Sun が開発し BSD 系でファイルシステ
ムとともに発達したプロトコルで,UNIX ではファイルシステムとの親和性が高
く,UNIX 間でシームレス性の高い共有を行う場合は,現在でもよく用いられる
\footnote{ワークステーションルームやスーパーコンピュータなど,多ノードが
自分のローカルストレージとしてリモートストレージを用いる場面などがあげら
れる}.HTTP 1.1 の拡張規格である,WebDAV による共有も一部存在している.
一般に,HTTP や FTP ではマウントの概念がないため,ファイル共有ではなく,
ファイル転送プロトコルに分類する.


\subsection*{NFS}
%によるUNIX---UNIX 間のファイル共有}
NFS は,TCP/IP 環境で使用されているネットワーク経由のファイルサービスである.
Sun Microsystems 社によって開発され,その仕様が公開されているためほとんどの UNIX 系の OS で広く利用されている.

NFSには三つの重要な特性がある.
\begin{itemize}
\item NFSを用いるとネットワークで接続されたホスト間でファイルを共有できる.
\item UNIX系OS間での相互運用において,ほとんどの場合に問題なく動作する.
\item NFSは,システムにセキュリティ上の問題の原因となる.これらの問題は
      一般に良く知られており,攻撃者にファイルに対するアクセス権が侵される可能性がある.
\end{itemize}
セキュリティに関する部分はIPネットワーキングや関連用語に対するある程度の知識を必要とするので,
ネットワーク管理に関する書籍等\footnote{「TCP/IP ネットワーク管理」オーム社}を読んでTCP/IPの用語に慣れることが必要である.

NFS を利用すれば,リモートマシンのディレクトリ(ディスク領域)をローカルマシンに呼び出して使用することができる.
具体的には,mount コマンドを利用することで,他のマシンのディレクトリを自マシンのディレクトリにマウントすることができる.
NFS マウントされたリモートマシンのディレクトリは,ローカルマシンのディレクトリと全く同じように操作することが可能となる.

\subsection*{Samba}
前節の NFS(Network File System)は,UNIX マシン同士での
ファイル共有を実現するものであり,UNIX におけるファイル共有の
業界標準となっている.
一方,現在市場に流通している PC/AT 互換機のほとんどは,Micorsoft 社の
Windows 系 OS( Windows 2000/XP/Vista/7 など)を搭載しており,
一般ユーザは Windows の上で様々なアプリケーションを動作させている.
企業や学校などの LAN においても,Windows 系 OS を搭載した PC が
クライアントとして多く使われている.

Windows 同士でファイル共有を行うのは非常に簡単で,マウスボタンを何回か
クリックするだけでよく,アクセス権の設定も 容易に行える.しかし,LAN に
接続されている UNIX ,Linuxクライアントや UNIX サーバとの間でファイルを共有する
機能は,標準では提供されていない.

Samba(サンバ)は,
%UNIX-Windows 間でのファイル共有ができるオープンソースソフトである.
%今回の実験では,これを使ってUNIX ,Linuxと Windows との間でのファイル共有ができる環境を構築する.
%\subsubsection{Sambaの機能}
%Samba は,
Australian National University の Andrew Tridgell 氏らによって
開発された UNIX-Windows 間でのファイル共有ができるフリーソフトで\footnote{URL http://www.samba.org/},
そのソースコードも公開されている.
日本では,日本 Samba ユーザ会\footnote{URL http://www.samba.gr.jp/} が,
オリジナルの Samba に対して日本語対応版ソースコードを提供していたが,
Samba 3以降では多言語対応し,ソースコードのローカライズの必要が無くなり,
現在は日本語マニュアルの配布等国内における日本語ドキュメントの整備を行っている.

Samba で提供される主な機能には以下のようなものがある.
%\begin{itemize}
%\item Samba の機能
\begin{description}
\item[ファイルサーバ機能] 共有毎に使用できるユーザを制限可能である.
\item[プリントサーバ機能] 共有プリンタ毎に使用できるユーザを制限可能である.
最近のネットワークプリンタは,Samba や Windows サーバを介さずに Windows クライアント間で共有する事が可能である.
Samba を使ってプリンタを共有するメリットは, Windows 95/98/NT/2000 クライアントに対し,プリンタドライバを自動インストールできる事である.
\item[PDC (プライマリ $\cdot$ ドメイン $\cdot$ コントローラ ) 機能]
Windows クライアントに対し,ドメイン $\cdot$ ログオンを提供し,ログオン時にスクリプトを実行させる事ができる.
サポートするクライアントは, Windows 95/98/Me/NT 4.0/2000/XP/7 である.
%Vistaはどうなんだろう
%ただし,Windows では Professional のみがドメインログオンに対応している.
\item[ドメイン $\cdot$ メンバ機能]
Windows ドメインにSambaサーバをメンバサーバとして追加する事ができる.
ユーザ認証がすべて Windows NT/2000 のドメインコントローラで行なえるので,パスワードの二重管理が必要なくなる.
\item[WINS サーバ機能] WINS (Windows Internet Name Service) サーバになることができる.
\item[マスタ $\cdot$ ブラウザ機能] ブラウズリストを保持し, Windows クライアントにネットワーク $\cdot$ コンピュータ一覧を提供する. 
LMB (ローカル $\cdot$ マスタ $\cdot$ ブラウザ) は Windows マシンも Samba もなることが可能であるが,
DMB (ドメイン $\cdot$ マスタ $\cdot$ ブラウザ) は Windows NT/2000 Server と Samba だけがなることができる.
\item[ユーザ$\cdot$ホーム機能] 共有の表示時にクライアントのユーザ名を共有名としてマッピングすることができる.
\item[ユーザ名マッピング] サーバ側でクライアントのユーザ名をマッピングして変更可能
\item[クライアント毎,ユーザ毎に設定ファイルを用意] クライアント毎,ユーザ毎にアクセスできる共有名やそれぞれの設定を変更可能である.
\item[QUOTA 機能] Windows NT 4.0 では標準サポートされていない QUOTA\footnote{ディレクトリ単位の容量制限} 機能を利用可能
\item[ゴミ箱機能] ユーザーが誤って消してしまったファイルを復活できる機能. Windows NT/2000 では標準で無い機能である.
\end{description}

%\item 
また,Samba には以下に述べるような制限事項と,Windows との相違点がある.
\begin{description}
\item[日本語の問題] 機種依存文字は使用しない方が良い(半角カタカナやカッコ数字・丸数字・ローマ数字,はしご高など).
\item[暗号化パスワードの問題] UNIX では認証に平分パスワードを使用するが, Windows NT 4.0 SP3 以降は暗号化パスワードを使用する.
\item[大文字と小文字の問題] Windows にはファイル名の大文字小文字の区別がない
\end{description}
%\end{itemize}

%図~\ref{fig:13:sambaflow}に Samba 導入までの流れ図を示す.
%最近では,各 OS 毎のパッケージ形式のものや,各ディストリビューション用にコンパイルされたバイナリも
%用意されていることが多いが,今回はソースからコンパイルしてインストールを行う.
%Samba の設定を Web ブラウザで行う SWAT (Samba Web Administration Tool) から行う方法については
%本実験では触れないため各自で調べてみるとよい.
%
%\begin{figure}[hbtp]
%\begin{center}
%\includegraphics[width=10cm,height=10cm,keepaspectratio]{./07/smbflow.eps}
%\caption{Samba導入のフローチャート}
%\label{fig:13:sambaflow}
%\end{center}
%\end{figure}

\subsection*{NAS}

コンピュータに直接接続インターフェースを持たせるのではなく,LAN を通じて
ファイルのやりとりをしながら,ストレージとして用いる形態のものを,Network 
Attached Storage (ネットワーク接続ストレージ,NAS) と呼ぶ.

NAS は,ファイル共有プロトコルを用いてリモートコンピュータのストレージを
マウントし,ローカルのファイルシステムの一部として用いる.


\subsection*{SAN}

外部接続型の規格を,ストレージ向けに特化させ,さらに速度や規模,複雑な構
成も可能にした規格を,Storage Area Network (SAN) と呼ぶ.ストレージ接続専用ネットワークであり,
LAN と同様,スイッチやハブで多くの装置を接続することができる.大規模サー
バなどで用いられる.よく用いられるものに,光ファイバを用いた Fibre
Channel,TCP/IP 上に構築する SAN である iSCSI などがある.両者とも,プロ
トコルにはSCSI を用いており,前者は高速かつ大規模なサーバ群,後者は低コ
ストで中規模サーバ向けに用いられる.


\section{サーバ運用・管理}

\subsection{バックアップ}

いかなるコンピュータシステムにおいても予期せぬ事態は生じるものである.
その例としては以下のようなものが挙げられる.
\begin{itemize}
\item ユーザが誤ってファイルを削除した
\item プログラムのバグによりデータファイルが障害を受けた
\item ハードウェアの故障によりディスク全体が破損した
\item ネットワークを経由してウィルスが侵入した
\end{itemize}

以上のような状況に遭遇した場合,もしバックアップを取っていなければ,
データの復旧には多大な労力と時間を費すことになる.
また,中には復旧不可能なデータが含まれている可能性もある.

このような事態を回避するための有効な方法としてバックアップがある.
バックアップを取ることによって,上記のような事態が生じたとしても
データを適切に復旧することが可能になる.したがって,システムの管理者は
定期的にデータのバックアップを行う必要がある.


\subsection{レストア}

レストア(Restore:リストアとも)とは,再び書き込む(ストア),すなわち
バックアップしたものを,再びハードディスクなどの補助記憶装置に書き込むこ
とを意味する.ハードディスク交換などや,データ消去事故の後の復旧作業で用
いる.


\subsection{機器の廃棄}

情報機器は,自動車などの機器と同様,ある程度の年月が経つと動作や運用に問
題が出始めるため,そのような機器を廃棄し,リプレースなどによる買い替えを
行うなどする.

廃棄を行う際は,基本的には物品そのものは,産業廃棄物(個人のものは,物品・
地域により,リサイクル対象回収品,通常のゴミ収集,あるいは粗大ゴミなど
\footnote{UPSなどのバッテリーは有害化学物質があることから,特定の廃棄方
法をとることが必要とされる場合もある})
として廃棄する.また,環境や経済の面からは,中古業者などに引き取りを依頼
したり,必要とする施設・団体への寄付などもしばしば行われる.

このように廃棄した物品が他者の手に渡ることはよくあることであり,物品とと
もに重要な情報資産までもが意図せず他者に渡ることを抑制することは,情報管
理上重要である.具体的には,ハードディスクなどに書き込まれている情報には,
重要な機密情報,プライバシー,その他知的財産などが含まれる.このため,こ
うした情報を完全に消去しておくことが安全上求められる.こうしたことを行わ
ないと,最悪の場合,悪意のあるものによる犯罪の誘引にもつながる危険がある.

そこで,HDD,SSD,フラッシュメモリなどのリムーバブルデバイスの消去ツール
を用いて,完全な消去を行ってから廃棄を行う.

他者への譲渡を行う際は,さらに初期化しておくことが望ましいため,ルータ・
スイッチ・無線LAN AP・RAID\footnote{さらにはプリンタやスキャナ etc.}など
のOSプレインストールのコンピュータものなどは,それらの初期化を行う.


\subsection*{ディスク消去ツール}

HDD などの補助記憶装置は,通常の OS からのファイルの消去,あるいはフォー
マットでは,ディスク上の情報は完全には削除されない.これは,補助記憶装置
の速度が遅いために完全な消去には時間がかかること,また,消去した後,別の
データを書き込む際は,どのようなデータが書いてあっても単に上書きすれば良
いだけなので,消去の際は「空き領域」であることを示すだけで十分であるため
である.

このようなことから,消去をした後も,データそのものは「空き領域」とされる
部分に残っており,データ復旧ツールなどのソフトウェアも存在する.

すなわちデータを消去するためには,意味のないデータを上書きしておく必要が
あり,HDD 廃棄の際は,HDD 全てのセクタに対してこのような意味のないデータ
を書き込むことが安全である.

このような目的に用いられるのが HDD 消去ツールである.OS から起動し,特定
のファイルを消去するだけのもの,OS とは別にブータブル DVD などから起動し,
HDD 全体(またはパーティション単位)で消去するものなど,複数のものがある
が,廃棄の際は,HDD 全体の消去が望ましい.

\subsection*{廃棄・片付け・原状回復}

情報システムの設置・運用を請け負っていた場合に,契約終了に伴うシステム廃
棄の際は,室内の清掃,ゴミなどの廃棄,相手側のものではない機器や物品・資
料の引きあげ等を確実に行い,設置場所等の状態を,元の状態に戻すことが必要
である.


\section{実験内容(1)}

\subsection*{ファイル共有サーバの設定(SMB(CIFS))}

サーバ上でソフトウェア Samba により SMB (CIFS) サーバを構築し,UNIX
(Linux), Windows, Mac クライアントとファイル共有を行えるようにする.

\subsection*{バックアップ}

\begin{itemize}
 \item ユーザのデータ,ソースコードなどのインストール後にダウンロードした
       データのバックアップを行う.cp, rsync, tar コマンドを用いる.ただ
       し,cp コマンドを用いる場合は,パーミッションやオーナー・グループ
       などが変更されないようにすること.
 \item システム運用で用いる重要なファイル類や設定情報をバックアップしてお
       く.
 \item サーバのファイルシステム全体を tar コマンドを
       用いてバックアップする.
 \item これらのバックアップは,SCP,SMB等を用いてネットワーク上のストレー
       ジに保存するが,ここでは scp コマンドを使って Cent OS クライアン
       トに保存することとする.
\end{itemize}

動作しているファイルシステムのバックアップは困難であるため,本来はシング
ルユーザモードに移行し,すべてのユーザをログアウトさせプロセスも停止させ
る.

現在は,スナップショット機能があるためこれを用いることも多いが,ここでは,
動作しているファイルシステムに対し,サービスはなるべく停止させた状態で,
tar にてバックアップを行う.

なお,tar でシステム全体( / ディレクトリ)のバックアップを行う場合,通
常のファイルでないものも含まれることがあり,これらは除外する必要がある.
Ubuntu サーバの場合,/ 全体を tar のバックアップ対象として指定せず,
/bin, /etc, /home, /lib, /lib64, /root, /sbin, /usr, /var を必要なディレ
クトリとして羅列しておくのが良く,また,/etc/fstab はバックアップ対象と
して指定しないよう exclude (除外)指定する.バックアップファイルは,/ 
に置く(個別にこれら以外にディレクトリやファイルで大切なものがある場合は
個々で注意してバックアップする)\footnote{逆に /dev や /proc などシステ
ムが起動時にメモリ上に自動的に構築ものや,/boot などの起動に関するもの,
外部マウントしているもの,バックアップファイル自身などは含まないように注
意する.起動に関するものでバックアップが必要が場合は手順が複雑になる.}.

\subsection*{ストレージの交換とレストア(実験4Cのみ)}

システムに内蔵されている HDD を SSD に(あるいはその逆)に交換し,OS を
再インストールし,tar でバックアップしたファイルを SCP コマンドで戻し,
tar で展開し再起動して動作確認を行う.

\begin{itemize}
 \item サーバを停止して,ハードディスクを SSD に交換する.
 \item 交換には,スクリュードライバーでねじを回す必要があるが,\textbf
       {ドライバーの力加減に注意し,ねじの山をなめないようにする(なめる
       とPCが使えなくなる).力加減は,7割の力で押し,3割の力で右ネジの
       法則の方向(緩めるときは左回り,締めるときは右回り)に回す.}
 \item SSD に OS を再インストールする.
 \item OSインストール時の設定は最初のインストール時と同じパラメータを用
       いる(特にディスクのパーティション設定(LVMの場合はホスト名も)).
 \item ファイルシステムにバックアップからファイルのレストアを行う.
 \item 設定情報で反映されていないものなどがあれば,バックアップした設定
       を参考に,元に戻す.
 \item 新しい SSD ドライブで運用を継続する.
\end{itemize}
本実験では,3.5 インチの HDD に変えて,2.5インチ SSD を設置するため,マ
ウンタを用いて設置する.

\section{実験内容(2)}

システムの廃棄を行う.コンピュータについては,ストレージを完全に消去し,
それ以外のアプライアンス機器(ルータ・スイッチ等)は,工場出荷時の状態に
初期化し,ケーブル等を元の状態に原状復帰する.

\subsection*{廃棄}
 \begin{itemize}
  \item 各機器・設備ごとの手順に従って廃棄を行う.
 \end{itemize}

\section{必要となる知識}

\subsection{SambaによるUNIX--Windows間のファイル共有}
\setcounter{subsubsection}{0}

まず,パッケージマネージャを用いてサーバに SMB(CIFS) ファイルシステムの
サーバである Samba をインストールする.

\begin{cli}
# apt install samba
\end{cli}

\subsection*{Sambaの設定}
Samba の設定は,smb.conf (Ubuntu では,/etc/samba にある) にて
決められた規則で設定を記述する.

\begin{cli}
[セクション名]
パラメータ値 = 値
パラメータ値 = 値
.
.
[セクション名]  
パラメータ値 = 値
.
.
\end{cli}

となっている.セクションの内容は,[ ]で囲まれたセクション行と
次のセクション行の間に含まれている部分となる.
行頭に ``\#'' もしくは ``;'' がある行はコメント行と見なされ,
行末に ``\verb+\+'' がある場合,次の行は前の行の続きであると見なされる.
%行末に ``\verb+\+'' がある場合,次の行は前の行の続きであると見なされる.

ここでは,共有フォルダとして /home/share を他のコンピュータから共有名
「share」として共有する例を示す.

\begin{cli}
# mkdir /home/share   ←共有フォルダを作成
  (上記ディレクトリの書き込み権限を検討する)

# vi /etc/samba/smb.conf
ファイルの末尾に,[share] セクションを追加
(セクション名が共有名)

[share]
   comment = share  ← この共有の説明文
   browseable = yes  ← 他のコンピュータに共有一覧として表示させるか否か
   path = /home/share ← サーバ上での共有ディレクトリ
   guest ok = no  ← 認証無しで誰でも接続させるか否か
   read only = no ← 書き込み権限を与えるか否か
   invalid users = root ← 接続させないユーザ一覧

書き込み後,Samba の SMB/CIFS デーモン smbd の再起動
# systemctl restart smbd
\end{cli}

\subsection*{Sambaユーザの追加とパスワードの設定}
Samba は UNIX システムアカウントとは別に接続ユーザ・パスワード情報を持つ
ので,ステムの exp ユーザが Samba でアクセスできるよう設定する.

\begin{cli}
# smbpasswd -a exp
(パスワードの設定)
-a は初期追加時のみで,2回目以降のパスワード変更時は使わない
\end{cli}

\subsection*{クライアントからの接続}

\subsubsection*{Linuxシステム (コマンド)}

mount.cifs コマンドを用いる(FreeBSD 等の他のUNIXでも多少違いがあるが,
mount.smbfs などほぼ同様.詳細は マニュアルなどを参照).

\begin{cli}
Cent OSの例
$ su
# mkdir /share  ← 共有をマウントするディレクトリを作成
# mount.cifs -o user=exp //サーバのIP/share /share
\end{cli}%$

使用後は,アンマウントを忘れない.

\begin{cli}
# umount /share
\end{cli}

\subsubsection*{Linuxシステム (GUI)}

GUIでも用いることができるが,用いるデスクトップシステムにより手順は異な
る(GNOME, KDEなど).

\subsubsection*{Windows}

Windows の場合は,マウントポイントとしてディレクトリを用いず\footnote{用
いることもできる},ドライブレターとして,A: ドライブから Z: ドライブまで
のどれか未使用のものをマウント位置として用いる.

コマンドプロンプトから,T: ドライブにマウントする例である.

\begin{cli}
> net use T: \\サーバのIP\share /u:exp *
\end{cli}

エクスプローラから GUI でマウントすることもできる.

\subsubsection*{Mac}

Finder の「サーバへ接続」から,
\begin{center}
 smb://exp@サーバのIP/share
\end{center}
を接続先として選択する.


%%GEEEEEEEEEEEEEEEEEEEEE

\subsection{コマンドによるバックアップ}

\begin{itemize}
% \item バックアップ先は NAS へ SMB マウントしたリモートドライブとする.
 \item cp コマンド,rsync コマンドを用いて,owner や permission 
       を変更しない形で行う場合は,UNIX ファイルシステムに互換のファイル
       システムの場合のみ,完全な形でバックアップが行えるので,サーバ内
       の別の場所にバックアップ用ディレクトリを作成し,
       \texttt{/usr/home} をバックアップする.
\end{itemize}

Tar での復元は,なるべくサービスを停止させた状態で行う.


\subsection{各設備ごとの初期化手順}

各設備について初期化の手順を示す.

\subsection*{ルータ}

設定情報は,startup-config という名前のファイルで nvram: と呼ばれる記憶
領域に保存されている.erase コマンドでこれを消去する.

\begin{cli}
router# erase startup-config
router#dir nvram:  (確認)
Directory of nvram:/

  190  -rw-           0                    <no date>  startup-config
  191  ----          24                    <no date>  private-config
  192  -rw-        1567                    <no date>  underlying-config
    1  ----          16                    <no date>  persistent-data
    2  -rw-           0                    <no date>  ifIndex-table
    3  -rw-         595                    <no date>  IOS-Self-Sig#1.cer
    4  -rw-         595                    <no date>  IOS-Self-Sig#2.cer

↑ startup-config がないか,0バイトになっていれば良い.
\end{cli}

また,VLAN 情報が設定されている場合は,delete vlan.dat コマンドで消去す
る (dir flash: コマンドで確認できる).

\paragraph{注意!} 
delete コマンドは,flash: 領域のファイルを消去するが,
flash: 領域には,OS 本体も保存されている(bin で終るファイル名).vlan.dat 
以外のファイルを決して消去しないこと.

\subsection*{スイッチ}

設定情報は,ルータと同様,nvram: の startup-config に保存されているので,
erase コマンドで消去する.

\begin{cli}
Switch#erase startup-config
Erasing the nvram filesystem will remove all configuration files!
 Continue? [confirm]
[OK]
Erase of nvram: complete

Switch#dir nvram:
Directory of nvram:/

   62  -rw-           0                    <no date>  startup-config
   63  ----           0                    <no date>  private-config
\end{cli}

また,VLAN 情報は,flash: と呼ばれる別の記憶領域に,vlan.dat というファ
イルで保存されている.これを delete コマンドで消去する.

\begin{cli}
Switch#dir flash:
Directory of flash:/

    2  -rwx         796  May 10 1993 01:42:15 +00:00  vlan.dat
    3  -rwx        1914  May 21 1993 01:44:48 +00:00  n
    4  -rwx        2072  Oct 30 1993 21:57:37 +00:00  multiple-fs
    5  drwx         512   Mar 1 1993 00:07:53 +00:00  c2960-lanlitek9-mz.122-50.SE5

27998208 bytes total (16788992 bytes free)

Switch#delete vlan.dat
Delete filename [vlan.dat]?
Delete flash:vlan.dat? [confirm]
Switch#dir flash:
Directory of flash:/

    3  -rwx        1914  May 21 1993 01:44:48 +00:00  n
    5  drwx         512   Mar 1 1993 00:07:53 +00:00  c2960-lanlitek9-mz.122-50.SE5

27998208 bytes total (16793600 bytes free)
\end{cli}

\paragraph{注意!} 
delete コマンドは,flash: 領域のファイルを消去するが,
flash: 領域には,OS 本体も保存されている(c2950 で始まるファイル名).vlan.dat 以外のファイルを決し
て消去しないこと.

\subsection*{PC}

サーバ,Linux, Windows, Mac Mini,ノートパソコン等は,GNU
Coreutils\footnote{GNU が配布する,mv, cp, ls コマンド等ファイル操作コマ
ンド.Linux 等の多くの OS のファイル操作コマンドとして用いられている.
FreeBSD では BSD のファイル操作コマンドが OS 標準であるが,coreutils パッ
ケージを導入することで,gcp, gmv など g を冠した形でコマンドが用いられる
ようになる.カラー表示など本来の UNIX 系OSのファイル操作コマンドに比して
高機能であり,かつ GPL であrることが特徴である.} に付属する shred コマ
ンドを用いる.

shred コマンドは,ストレージ上のファイルあるいはストレージのブロックデバ
イス(セクタ単位のアクセスを行うデバイス)に対して,ランダム値や DoD,
NSA などで規定される消去方式に対応した書き込み動作を行うコマンドである.

\begin{cli}
# shred ファイル名
\end{cli}

OS も含めた HDD 全体の初期化を行う場合は,OS 全体を表すブロックデバイス
を指定する.ブロックデバイスにアクセスする際は,root 権限でしか行えない
ことに注意する.

デフォルトではランダムな値を3回書き込むが,-z オプションではゼロを書き込
むことも可能である\footnote{ゼロを書き込むだけで良ければ,ゼロを出力する
/dev/zero ファイルを in に指定し,OS 標準の dd (disk dump)コマンドを dd
if=/dev/zero of=出力ファイル bs=65536 などと実行しても良い.}.

OS 全体を初期化する場合は,消去するOS とは別の OS から実行する必要がある.
このような用途に,DVD \footnote{USB や CD-ROM から起動するものもある} か
ら起動できる OS がある.Knoppix は,よく用いられる DVD Live Linux で,
Debian をベースに産業総合技術研究所で日本語化が行われている.

Knoppix から,OS を起動し,shred コマンドを用いる.Linux では,ハードディ
スクなどのストレージのブロックファイルは,/dev/sda (一台目の HDD の 全体),
/dev/sdb (2台目以降,sdb, sdc 等と続く)であり,パーティションがある場
合は,さらに,/dev/sda1, /dev/sda2 などパーティション番号が付いたファイ
ルが各パーティション全体を表している.パーティション番号は,1 から 4まで
が基本パーティション,論理(拡張)パーティションは,/dev/sda5 など 5以上
の番号が用いられる.ここでは,ハードディスク全体を消去するよう指定する.

DVD からの起動は,DVD を挿入し起動するか,挿入しても起動しない場合は,
BIOS POST 中に特定のファンクションキーを押すなどして DVD から起動するよ
う指定する.Mac Miniの場合は,「C」キーを押しながら起動し,OS なしにDVD 
を排出したい場合は,マウスのボタンを押しながら起動する.

\subsection*{ケーブル類}

ケーブル類はすべて抜き,数量を確認し,まとめておく.

本来は,ケーブルタイなどで結んだり,長いネットワークケーブルは8の字巻き
と呼ばれる方法\footnote{ケーブルのクセがつきにくく,次回使用する際に便利
と言われる.}で丸めて結んでおくと良いが,本実験ではまとめておくだけで良
い.

\subsection*{マニュアル・書類・DVDメディア}

マニュアル・書類は元あった場所に戻す.

パスワードや顧客情報,ネットワーク情報等が書かれた文書やメモ類はシュレッ
ダーにかけて廃棄する.


\section{動作確認}

\subsection*{ファイル共有}
マウントを行った状態で,ファイルの読みだし,書き込みができることを確認す
る.



\subsection*{バックアップ}

バックアップファイルを正しく読み込めることを確認する(障害が発生するまで,
正しくバックアップが取れていなかったことが分からず,復旧できない場合も起
こるので,バックアップが正しく取られていることを確認することが重要であ
る).


\subsection*{レストア}

新しいドライブあるいは入れ替えたサーバプログラムで,正しくシステムが動作していること.


\section{考慮すべき点}

\subsection*{ストレージ}

磁気ディスク(HDD等),ソリッドステートドライブ(SSD, フラッシュメモリ),
光学ディスク(DVD, CD)などのそれぞれの技術的特徴を考察する.

どのような場面ではどのような仕組みのストレージが有効か考察する.

RAID レベルのそれぞれの違い,特徴と,なぜその特徴が実現されるかの技術的
な方法,原理について考察する.


\subsection*{ファイル共有}

ファイル共有のメリットはどのようなものがあるか.
また,デメリットにはどのようなものがあるか.
UNIX / LinuxからWindowsのファイル共有を利用するにはどのような方法があるか.

外部ネットワークとファイル共有するのはどのような場合が考えられるか.それによりどのようなメリット・デメリットが生じるか.
外部ネットワークと共有する場合のサーバの管理方法\footnote{使用するディスク容量, ファイルに対する読み書きの権限の設定...etc}はどのように行うか.

クライアントからNFSサーバが共有しているディレクトリ名を知るにはどうすればよいか.

Sambaのセキュリティモードにはいくつか種類がある.それぞれどのような特徴があるか.


\subsection*{バックアップ}

クライアントシステム,サーバシステムの障害対策のためには,どのようなバッ
クアップを行うことが良いかを考察する.具体的にどのようなファイルをどのよ
うな場所にバックアップし,どのようにリストアすれば良いのかを考える.

RAID,バックアップなどがカバーできる障害・事故の種類,範囲について明確に
する.特にどのような手法が,どのような障害対策になるかを考察する.

現在のストレージ・障害対策技術でカバーできないものにはどのようなものがあ
るか,それに対策を行うためには,どのような技術や原理が必要とされるかを考
察する.

