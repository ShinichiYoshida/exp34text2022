\textbf{■ \texttt{/etc/named/localhost} ファイル(localhost データベース)}
\begin{center}
\begin{breakbox}
\begin{alltt}
; file name: localhost
; for gX.info.kochi-tech.ac.jp
$TTL       1D
@          IN      SOA     serverX.gX.info.kochi-tech.ac.jp. postmaster.gX.
info.kochi-tech.ac.jp. (  (←前の行の続きで{\rm1}行で記述する)
            2002081601  ;serial
            3h          ;refresh
            1h          ;retry
            1w          ;expire
            1h )        ;negative TTL
           IN      NS      serverX.gX.info.kochi-tech.ac.jp.
0          IN      PTR     loopback.gX.info.kochi-tech.ac.jp.
1          IN      PTR     localhost.gX.info.kochi-tech.ac.jp.
\end{alltt}%$
\end{breakbox}
\end{center}

\noindent
\textbf{◆ \texttt{/etc/namedb/localhost} ファイルの説明}

\begin{itemize}
\item ローカルホストに関するデータベース.
\item ローカルホストが,自分自身に対して責任をもつために必要.
\end{itemize}