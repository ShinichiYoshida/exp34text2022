\noindent
\textbf{■ \texttt{/etc/namedb/gX.rev} ファイル(逆引きデータベース)}
\begin{center}
\begin{breakbox}
\begin{alltt}
; file name: gX.rev
; for 1X.21.172.in-addr.arpa.
$TTL    1H
@       IN      SOA    serverX.gX.info.kochi-tech.ac.jp. postmaster.gX.info.
kochi-tech.ac.jp. (   (←前の行の続きで{\rm1}行で記述する)
                  2002081601	      ; serial number
                  3h			; refresh after an hour
                  1h			; retry after 30 minutes
                  1w			; expire after a week
                  1w )			; minimum TTL
     IN      NS     serverX.gX.info.kochi-tech.ac.jp.
1     IN      PTR    ciscoX.gX.info.kochi-tech.ac.jp.
2     IN      PTR    serverX.gX.info.kochi-tech.ac.jp.
3     IN      PTR    xpX.gX.info.kochi-tech.ac.jp.
4     IN      PTR    linuxX.gX.info.kochi-tech.ac.jp.
5     IN      PTR    printerX.gX.info.kochi-tech.ac.jp.
10    IN      PTR    hubX.gX.info.kochi-tech.ac.jp.
\end{alltt}%$
\end{breakbox}
\end{center}

\noindent
\textbf{◆ \texttt{/etc/namedb/gX.rev} ファイルの説明}

\begin{itemize}
\item IP アドレスからホスト名を検索するためのデータベース.
\item ``\verb|@|''は 1X.21.172.in-addr.arpa の省略形.また,省略形のア
ドレスには 1X.21.172.in-addr-arpa ドメインが付加される.
\item 3$\sim$10 行目は zone ファイルと同じ意味.
\item PTR レコードで IP アドレスとホスト名の対応づけを行う.zone ファイ
ルと矛盾のないようにする.
\end{itemize}