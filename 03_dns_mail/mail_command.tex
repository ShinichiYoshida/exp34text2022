\subsection*{mailコマンドによるメール送信}
UNIX系OSにはメール送信を行うmailコマンドが標準で存在する.
user1 と user2 という名前のアカウントを作ったとして,
ユーザ user1 から user2 にメールを送る場合は

\begin{center}
\begin{breakbox}
\begin{alltt}
# \underline{su - user1}  (←ユーザを変更.パスワードの入力を求められるときがある)
$ \underline{mail user2@gX.info.kochi-tech.ac.jp}
                           (↑{\rm mailコマンドでuser2宛てにメールを出す})
Subject: タイトルを入力
本文
…
(メールを送信するには\keybox{Ctrl}を押しながら\keybox{d}を入力するか,.を入力して\keybox{Enter})
$
\end{alltt}
\end{breakbox}
\end{center}

となる.送られたメールは user2 のメールスプールに蓄えられる.
user2はpopなどを用い,このメールを読むことができる
\footnote{サーバ上で読む場合は,
su コマンドで user2 になった後に,mail コマンドをそのまま立ち上げると,
メールのリストが表示され,メール番号を入力すると,本文が読める.}.

UNIX 系 OS では,MUA として mail コマンドがある.これは,SMTP, POP など
のネットワークプロトコルを用いず,スプールのファイルから直接メッセージを
読み出し,ユーザディレクトリの mbox ファイルにメールを保存する.送信時は,
sendmail コマンド(本サーバの環境では,実体は postfix)を直接呼び出しメッ
セージを送信する.これも操作してみると良い.具体的には,下記の例のように
メッセージの送受信を行う.

\begin{cli}
%mail             ← mail コマンド

Mail version 8.1 6/6/93.  Type ? for help.
"/var/mail/shin": 1 message 1 new
>N  1 shin@phenom           Wed May 30 08:03  23/792   "Cron
>N  2 shin@phenom           Wed May 30 08:03  23/792   "Cron
>N  3 shin@phenom           Wed May 30 08:03  23/792   "Cron

   ↑メッセージの一覧が表示される

& 1    ← プロンプト:「1」を入力すると1つ目のメッセージを閲覧する

Message 1:
From shin@phenom Wed May 30 08:03:00 2012
Date: Wed, 30 May 2012 08:03:00 +0900 (JST)
From: shin@phenom (Cron Daemon)
To: shin@phenom
Subject: Cron <shin@phenom> /usr/home/shin/bin/check-ssh.sh

Hello! My name is who!

    ↑以上がメッセージ 1 の内容
      最初 From の行から,最初の空行までがヘッダ
      それ以降が本文である.

& x   ← 「x」で終了

送信は,以下の通りである.

%mail dareka@sdsadas.dsadsa.jp  ← 送信先アドレスを引数で
                                指定して mail コマンド実行
Subject: test  ← タイトル
sss           ←本文1行目
sss
ss
.             ←本文が終了したら「.」を入力

以上で送信される.
\end{cli}
