\section{ソースコードからの一般的なソフトウェアインストール手順}

ここでは,BIND をはじめ,UNIX 系 OS で用いられる多くのオープンソースのフ
リーソフトウェアのソースコードからのインストール手順を示す.

まずは,ソースコードをダウンロードする.この実験では,ftp サーバ 192.168.0.1 (メインサーバ)に必要なものはおい
てあるので,ここからダウンロードする.

ソースコード類は,/pub/sources 以下にある.

\begin{cli}
% cd ダウンロードするディレクトリ
% ftp 192.168.0.1
 ログイン名:anonymous
 パスワード:なんでも良い
      (以前はメールアドレスを入れるのがネチケットだったが
        昨今はむしろプライバシーの問題
        などから入れない方が安全)
% cd /pub
% ls
% cd sources
% ls
% get 必要なファイル (例えば,bind-9.9.1.tar.gz など)
% quit
\end{cli}

以上でダウンロードは終了である.

ソースコードは,C/C++ 言語のソースコード *.c ファイルや,ヘッダ *.h ファ
イル,それらをコンパイルする手順が書かれている Makefile,その他のスクリ
プトなどからなり,数十〜数百のファイルが TAR 形式\footnote{Tape Archive 
と呼ばれる UNIX 系で一般的なアーカイブ(複数のファイルを一つのファイルに
まとめる)形式}でアーカイブされており,さらに GNU zip (gzip) または,圧
縮率が高い bzip2形式,xz 形式で圧縮されている\footnote{gzip が UNIX 系で
は最も一般的で,bzip2 はその改良型,xz は比較的新しい圧縮率が高い(CPUや
メモリをより消費する)方式で,Windows などでは ZIP(.zip),LHA (.lzh) も
よく用いられる.gzip, bzip2 はほとんどの Linux, FreeBSD などで標準で同梱
されているが,伝統的な商用 UNIX では compress 形式 (.Z) しか用いることが
できないものも多かった.}.

ファイル形式は,\\
**.tar.gz\\
または,\\
**.tgz\\
などとなっているので,

\begin{cli}
% tar tvzf *.tar.gz
\end{cli}
などとして内容を確認し,問題がなければ適当なディレクトリで,
\begin{cli}
% tar xvzf *.tar.gz
\end{cli}
として展開する.

詳しくは,付録B章の B.21 tar を参照すること.

tar で展開し作成されたフォルダに入り,
\begin{cli}
% cd folder
\end{cli}
README.txt あるいは INSTALL.txt などという名前のファイルがあるので,
このファイルを見てその手順に従う,
\begin{cli}
% less README.txt
% less INSTALL.txt
 (q で less を終了できる)
\end{cli}
よくある一般的な手順では,configure というシェルスクリプト(実行ファイル)
があり,これを用いて Makefile を作成し,そして make を実行することで,
コンパイルされ,make install することでファイルが配置され,ソフトウェア
のインストールが行える.

つまり,まとめると,
\begin{cli}
% cd ソースをコンパイルする場所
% tar cvzf ダウンロードしたファイル名
% cd 作成されたフォルダ
% ./configure  (オプションが必要であれば付ける)

(オプション一覧は,./configure --help で
  見ることができる)

% make

コンパイルされて問題がなかったら,インストール
インストールはスーパーユーザ権限が必要なので,

% su
# make install

ただし,make install する前にそれまであった同名の
ファイルがあれば,上書きされてしまうので,問題がないか確認すること.
\end{cli}

\section{パッケージを用いた必要なソフトウェアのインストール手順}

サーバ用ソフトウェアは,基本的には次節で説明するソースコードからのインス
トールにてインストール作業を行うが,いくつか前もって必要なものは,パッケー
ジと呼ばれる,そのOS・バージョンのために既にコンパイル済みのソフトウェア
をインストールする.

ここでは,perl というスクリプト言語を導入する.

まず,環境変数の設定をして,メインサーバをパッケージサーバとして指定し,
pkg\_add コマンドでパッケージ名を指定する.これは,スーパーユーザ権限で行
う.

\begin{cli}
% su
# setenv PACKAGEROOT ftp://192.168.0.1
   (環境変数でパッケージサーバの指定)
# pkg_add -r perl
\end{cli}

以上でパッケージによる perl のインストールは終了である.

その他にも数万のソフトウェア・コマンド類がこの手順でインストールできる.
