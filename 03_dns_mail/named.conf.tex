\noindent
\textbf{■ \texttt{/etc/named.conf} ファイル}
\begin{center}
\begin{breakbox}
\begin{alltt}
options \{
        directory "/etc/namedb";
        pid-file "/etc/namedb/named.pid"; 
        allow-query \{ "any"; \};
        max-ncache-ttl 3600;
        notify no;
\};
zone "." \{
        type hint;
        file "named.root";
\};

zone "0.0.127.in-addr.arpa" \{
        type master;
        file "localhost";
\};

zone "gX.info.kochi-tech.ac.jp" \{
        type master;
        file "gX.zone";
\};

\end{alltt}
\end{breakbox}
\end{center}
%zone "1X.21.172.in-addr.arpa" \{
%        type master;
%        file "gX.rev";
%\};
\noindent
\textbf{◆ \texttt{/etc/named.conf}ファイルの説明}
\begin{itemize}
\item BIND では起動ファイル \texttt{named.conf} ファイルで,ゾーンに対するデー
タが記述されたファイル名の指定,など様々な設定を行う.
\item options は,データベースファイルやキャッシュファイルのあるディ
レクトリなどのグローバル値を指定する.
%\item forwarders はグループサーバで解決できない名前をメインサーバから
%取得するための設定.これにより DNS は,ルートネームサーバに名前
%を問い合わせする前にまずメインサーバに問い合わせる,という挙動
%をする.
\item zone の後はドメイン名(``\verb|.|''はルートドメインを示す)を記
述し,このゾーンに対するデータベースファイル名を file で指定する.
さらに各ゾーンに対する設定を記述することも可能であり,設定され
ていない項目は options で指定したものになる.
\item type は,primary のサーバの場合 master とする.
\item allow-query は DNS に問い合わせ可能なホストを指定する.
ここでは any を指定しているので,どのホストも DNS に問い合わせ可能である.
\end{itemize}
