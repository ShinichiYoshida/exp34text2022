\subsection*{qpopperのコンパイルオプション}
FreeBSDではPAM を経由した認証を行っている\footnote{PAMについては,
ここでは触れないが,Linux など多くの UNIX系OS で標準採用されている認証モ
ジュールであり,多様な認証手段を提供できる仕組みである.興味があれば調べ
てみよ.}ので,configure のオプションとして \texttt{--with-pam=pop3} を
用いる.



Server では OS に標準で付属する sendmail が自動的に動作しているので,
これを停止し,またブート時に自動的に動作を開始しないようにする.

\begin{center}
\begin{breakbox}
\begin{alltt}
# ps aux | grep sendmail
root    785  0.0  0.2  6072  3444  ??  Ss   23Mar10   0:09.62 sendmail: accepti
smmsp   789  0.0  0.2  6072  3388  ??  Is   23Mar10   0:00.19 sendmail: Queue r
# kill -9 785 789 (← 上の ps コマンドで確認したプロセス番号(左
                               から2カラム目の番号))

# \underline{vi /etc/rc.conf}
sendmail_enable="NONE"     (←この行を追加)

\end{alltt}
\end{breakbox}
\end{center}


\paragraph{注意} Postfix は FreeBSD 9.0-RELEASE への対応が
遅れているので,最新版 postfix-2.9.3.tar.gz の他に,
パッチ postfix-2.9.3.1.patch も同じ場所から取得して,postfix-2.9.3 を展
開した後,patch コマンドで patch をあててからコンパイルする.

まず,postfix のプロセスのための専用アカウントを作成する.

\begin{center}
\begin{breakbox}
\begin{alltt}
#pw useradd -n postfix
\end{alltt}
\end{breakbox}
\end{center}

ここで,INSTALL ファイルの 6.2 にあるように,このアカウントは通常ユーザ
のようにログインができるとまずいので,passwd ファイルを vipw コマンドで
編集し,パスワードの欄は '*', ホームディレクトリは '/no/where',ログイン
シェルは '/no/shell' として存在しない情報を書き込んでおく.

\begin{center}
\begin{breakbox}
\begin{alltt}
#vipw
    …
postfix:*:1003:1003::0:0:User \&:/no/where:/no/shell
    …
\end{alltt}
\end{breakbox}
\end{center}

さらに,postdrop グループを作成する.
\begin{center}
\begin{breakbox}
\begin{alltt}
#pw groupadd -n postdrop
\end{alltt}
\end{breakbox}
\end{center}

\texttt{/etc/group} に postfix, postdrop の2つのグループができていること
を確認する.

次にビルドを行う.

\begin{center}
\begin{breakbox}
\begin{alltt}
#make
…
\end{alltt}
\end{breakbox}
\end{center}

コンパイルが終わった後,インストールの前に現在の OS にデフォルトでイン
ストールされていた sendmail 関連のプログラムを退避す
る.postfix は,sendmail,newaliases,mailq の名前で互換コマンドを上書
きインストールするため,これらのファイル名を変更しておく.

\begin{center}
\begin{breakbox}
\begin{alltt}
# \underline{mv /usr/sbin/sendmail /usr/sbin/sendmail.OFF}
# \underline{mv /usr/bin/newaliases /usr/bin/newaliases.OFF}
# \underline{mv /usr/bin/mailq /usr/bin/mailq.OFF}
…
#
\end{alltt}
\end{breakbox}
\end{center}

インストールを行う.

\begin{center}
\begin{breakbox}
\begin{alltt}
# \underline{make install}
…

Please specify the prefix for installed file names. Specify this ONLY
if you are building ready-to-install packages for distribution to other
machines.
install_root: [/]  \underline{[ENTER]}

Please specify a directory for scratch files while installing Postfix.
You must have write permission in this directory.
tempdir: [/展開ディレクトリ]  \underline{[ENTER]}

Please specify the final destination directory for installed Postfix
configuration files.
config_directory: [/etc/postfix]  \underline{[ENTER]}

Please specify the final destination directory for installed Postfix
administrative commands. This directory should be in the command search
path of adminstrative users.
command_directory: [/usr/sbin]  \underline{[ENTER]}

Please specify the final destination directory for installed Postfix
daemon programs. This directory should not be in the command search
path of any users.
daemon_directory: [/usr/libexec/postfix]  \underline{[ENTER]}

Please specify the final destination directory for Postfix-writable
data files such as caches or random numbers. This directory should not
be shared with non-Postfix software.
data_directory: [/var/lib/postfix]  \underline{[ENTER]}

Please specify the destination directory for the Postfix HTML files.
Specify ``no'' if you do not want to install these files.
html_directory: [no]  \underline{[ENTER]}

Please specify the owner of the Postfix queue. Specify an account with
numerical user ID and group ID values that are not used by any other
accounts on the system.
mail_owner: [postfix]  \underline{[ENTER]}

Please specify the final destination pathname for the installed Postfix
mailq command. This is the Sendmail-compatible mail queue listing
command.
mailq_path: [/usr/bin/mailq]  \underline{[ENTER]}

Please specify the destination directory for the Postfix on-line manual
pages. You can no longer specify ``no'' here.
manpage_directory: [/usr/local/man]  \underline{[ENTER]}

Please specify the final destination pathname for the installed Postfix
newaliases command. This is the Sendmail-compatible command to build
alias databases for the Postfix local delivery agent.
newaliases_path: [/usr/bin/newaliases]  \underline{[ENTER]}

Please specify the final destination directory for Postfix queues.
queue_directory: [/var/spool/postfix]  \underline{[ENTER]}

Please specify the destination directory for the Postfix README files.
Specify ``no'' if you do not want to install these files.
readme_directory: [no]  \underline{[ENTER]}

Please specify the final destination pathname for the installed Postfix
sendmail command. This is the Sendmail-compatible mail posting
interface.
sendmail_path: [/usr/sbin/sendmail]  \underline{[ENTER]}

Please specify the group for mail submission and for queue management
commands. Specify a group name with a numerical group ID that is not
shared with other accounts, not even with the Postfix mail_owner
account. You can no longer specify ``no'' here.
setgid_group: [postdrop]  \underline{[ENTER]}

        …

    Warning: you still need to edit myorigin/mydestination/mynetworks
    parameter settings in /etc/postfix/main.cf.

    See also http://www.postfix.org/STANDARD_CONFIGURATION_README.html
    for information about dialup sites or about sites inside a
    firewalled network.

    BTW: Check your /etc/aliases file and be sure to set up aliases
    that send mail for root and postmaster to a real person, then
    run /usr/bin/newaliases.
\end{alltt}
\end{breakbox}
\end{center}

INSTALL 終了後のメッセージにもある通り,\texttt{/etc/postfix/main.cf} が
設定ファイルであるので,この設定ファイルにある,メールサーバのホスト名と
ドメイン名の設定を正しいものに変更する.

\begin{center}
\begin{breakbox}
\begin{alltt}
#myhostname = host.domain.tld
        ↓
myhostname = server.gX.info.kochi-tech.ac.jp

#myorigin = $mydomain
        ↓
myorigin = $mydomain

#mydestination = $myhostname, localhost.$mydomain, localhost, $mydomain
        ↓
mydestination = $myhostname, localhost.$mydomain, localhost, $mydomain

\end{alltt}
\end{breakbox}
\end{center}

postfix を停止したい場合は postfix stop,設定ファイルを修正
し postfix に反映させたい場合は postfix reload コマンドを用いる.

次に,postmaster 宛のメールを root で受け取れるように設定を行う.
/etc/aliases ファイルにて,下記の行があれば良い.
(デフォルトで既に設定されているので,記述されていることを確認すれば良い)

postmaster は通常メールサーバの管理者が受け取るよう設定されているもので,
複数の管理者がいる場合は,そのアカウントをコンマで区切って全て書いておく
(root ユーザでなくともメールを受け取れるユーザであれば良い).

\begin{center}
\begin{breakbox}
\begin{alltt}
postmaster: root
\end{alltt}
\end{breakbox}
\end{center}

aliases ファイルを書き換えたら,newaliases コマンドを実行しデータベースファ
イルを再構築し,現在動作している postfix にそのデータベースを読み込ませ
る.alias 情報は,実行中の postfix デーモンが検索する際の効率を上げるた
めに,ハッシュ化されているのでnewaliases コマンドでハッシュ情報を更新す
る.

\begin{center}
\begin{breakbox}
\begin{alltt}
#newaliases  (←エイリアス・ファイルを再構築)
\end{alltt}
\end{breakbox}
\end{center}

OS 起動時に自動的に postfix が起動するよう,/etc/rc.local に登録する.

\begin{center}
\begin{breakbox}
\begin{alltt}
#vi /etc/rc.local
/usr/sbin/postfix start
\end{alltt}
\end{breakbox}
\end{center}

%
%   qpopperの設定
%        written by Takuji Nakahira and Hiroyuki Kobayashi
%       modified by Takahiko Mendori
%
\section{qpopperの設定}
qpopper をメールサーバである server にインストールして POP サーバを構築する.

\subsection{qpopperの取得とインストール}
qpopper は,メインサーバの \texttt{/pub/sources} に qpopper のソースコードがある
ので,ftp を用いて取得し,解凍・展開を行ったのちインストールを行う.

POP サーバは,クライアント(MUA)からのメールメッセージ読み込み要求に対し
て,要求を処理する前にユーザ認証を行い,認証されたユーザであるか否かを判
定する.
% 
% \begin{center}
% \begin{breakbox}
% \begin{alltt}
% 解凍・展開
% …
% # \underline{./configure --with-pam=pop3}
% …
% コンパイル・インストール
% \end{alltt}
% \end{breakbox}
% \end{center}

\subsection{qpopperの設定}
qpopper は,自らは通常状態ではデーモンとしては起動せず,別のデーモンが
POP のポート接続を受けた際に,プロセスが起動されて制御が渡される.この別
のデーモンのことをスーパーデーモンと呼ぶ.このサーバではスーパーデーモン
として,インターネットデーモン inetd が動作しているので,inetd の設定ファ
イル \texttt{/etc/inetd.conf} の pop の箇所に,qpopper を登録する.

具体的には,inetd\footnote{インターネットソケットを監視するデーモン
(daemon).ソケット(socket)に対して接続要求が出されると,そのソケット
に対応したサービスを判定し,サービスを提供するプログラムを起動する.} の
設定ファイル \texttt{/etc/inetd.conf} ファイル(サービス名,ソケットタイ
プ,実行するプログラムを記述)に次の一行を追加する.
\begin{center}
\begin{breakbox}
\begin{alltt}
# vi /etc/inetd.conf
   …
pop3    stream  tcp     nowait  root    /usr/local/sbin/popper -s
   …
\end{alltt}
\end{breakbox}
\end{center}

もし,\texttt{/etc/services} に pop3 というプロトコルが書き込まれていな
ければ,下記のように TCP 110 番ポートであることを記述する.

\begin{center}
\begin{breakbox}
\begin{alltt}
pop3    110/tcp
\end{alltt}
\end{breakbox}
\end{center}

ps コマンドで,inetd が存在するか確認し,無ければ起動する.
\begin{center}
\begin{breakbox}
\begin{alltt}
#inetd
\end{alltt}
\end{breakbox}
\end{center}

必要があれば,\texttt{/etc/rc.conf} に \texttt{inetd\_enable= "YES"}
を設定しておき,サーバ起動時に自動起動するよう設定しておく.

もし存在していれば,変更した\texttt{/etc/inetd.conf}ファイルを再読み込み
するために,inetd に HUP シグナルを送る.
\begin{center}
\begin{breakbox}
\begin{alltt}
# kill -HUP X  (←XはプロセスID)
\end{alltt}
\end{breakbox}
\end{center}
