\textbf{■ \texttt{/etc/namedb/gX.zone} ファイル(正引きデータベース)}
\begin{center}
\begin{breakbox}
\begin{alltt}
; file name: gX.zone
; for gX.a360.info.kochi-tech.ac.jp
\$TTL       1H
@          IN      SOA     serverX.gX.info.kochi-tech.ac.jp. postmaster.gX.
info.kochi-tech.ac.jp. (          (←{\rm{}前の行の続きで1行で記述する})
                   2011060101	      ; serial number
                   3h			; refresh after an hour
                   1h			; retry after 30 minutes
                   1w			; expire after a week
                   1w )			; minimum TTL
           IN      NS      serverX.gX.info.kochi-tech.ac.jp.
serverX    IN      A       192.168.0.X1  ({\rm{}サーバのIPアドレス})
linuxX     IN      A       192.168.0.X2  ({\rm{}LinuxのIPアドレス})
windowsX   IN      A       192.168.0.X3  ({\rm{}WindowsのIPアドレス})
macX       IN      A       192.168.0.X4  ({\rm{}MacintoshのIPアドレス})
noteX      IN      A       192.168.0.X5  ({\rm{}ノートPCのIPアドレス})
localhost  IN      A       127.0.0.1
\end{alltt}
\end{breakbox}
\end{center}
\noindent
\textbf{◆ \texttt{/etc/namedb/gX.zone} ファイルの説明}
\begin{itemize}
\item ホスト名から IP アドレスを検索するためのデータベース.
\item セミコロン``\verb|;|''から後の文字列はコメントである.
``\verb|(|''や ``\verb|)|'' の前には空白を 1 文字以上挿入する.
\item 5$\sim$11 行目を除いて,すべて行頭から記述しなければならない.
空白があると動作しない.
\item 3 行目,``\verb|$TTL|''はデフォルトの TTL(Time To Live:キャッ%$
シュの生存時間)を指定する.DNS では各ホスト情報毎にTTLを指定で
きるが,TTL が省略されたレコードはこの値が用いられる.
\item 4 行目,``\verb|@|''はドメイン \texttt{gX.info.kochi-tech.ac.jp} を
表す省略形である.SOA レコードにより,このドメインに権威をも
つ(このドメインについて完全な情報をもつ)DNS サーバと管理者の
メールアドレスを空白で分けて記述する.
\item 5 行目はシリアルナンバーを意味しており,ここではファイルを作成
した日付を \verb|yyyymmddNN| の形式で記述した.このデータベース
を書き換えたら数字を増やす.このようにしないとセカンダリネームサーバ
に変更が反映されない.
\item 6$\sim$8 行目はセカンダリネームサーバに関する設定(セカンダリネー
ムサーバがプライマリのゾーン転送を行なう間隔,プライマリの応答
が無い場合にセカンダリがリトライするまでの時間,セカンダリがデー
タを破棄するまでの時間).
\item 9 行目の TTL は,ネガティブキャッシュ(リクエスト失敗という情報
のキャッシュ)に用いられる値.
\item 10 行目は,NS レコードで nameserver とするマシンの名前を記述する.
ここではホスト名の最後には必ずピリオド``\verb|.|''を付ける.
``\verb|.|''がないと省略形とみなされ,\verb|named.conf|ファイ
ルの \verb|zone| で指定した \texttt{gX.info.kochi-tech.ac.jp} がホスト
名の最後に付加される.つまり
``\texttt{serverX.gX.info.\\kochi-tech.ac.jp}''のように指定すると,
``\texttt{serverX.gX.info.kochi-tech.ac.jp.gX.\\info.kochi-tech.ac.jp}''
のように解釈される.また 10,11 行目は,行頭の %
\texttt{gX.info.kochi-\\tech.ac.jp} が省略されているため,行頭には空
白が必要である.
%\item 11 行目は,MX レコードにより,ドメイン %
%\texttt{gX.info.kochi-tech.ac.jp} 宛てに出されたメールを %
%\texttt{serverX.gX.info.kochi-tech.ac.jp} で受けることを宣言する.
%プリファレンス値 10 はメール配送の優先度を表す.
\item 11$\sim$14 行目では,A レコードを用いてマシンの名前に address を割
り振る.``\verb|.|''で終っていないドメイン(省略形)には %
\texttt{gX.info.kochi-tech.ac.jp} が付加されるので,ここではドメイ
ン名は省略されている.もし,address を割り振るマシンが他にもあ
れば(WWW サーバや POP3 サーバが他のマシンにある場合など),16 行
目以下に追加して記述する.
\item 17 行目は localhost のアドレス指定.
\end{itemize}
