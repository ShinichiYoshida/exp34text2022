\clearpage

\section{BIND を用いた DNS 設定の詳細}

ここでは,これまで説明したことをもう少し詳しく記す.必要のある場合は適宜
参照のこと.

BIND では下記の3つのサービス(第2項目と第3項目は実質同じ)の行う必要があ
る.

\begin{description}
 \item[ネームサービス] 各グループ内のホスト名情報を保持して,グループ内
	    外からの問い合わせに対して返答する
 \item[リゾルバ] OSに対して\footnote{通常 OS 自身は DNS を用いた名前解決
	    機能を持たないため,そのままではルートサーバからの階層的な名
	    前解決を行うことはできず,「DNSサーバ」などの項目で指定され
	    たホストに名前解決を依頼する.},DNSによる名前解決機能を提供
	    するリゾルブ(resolv = 解決)サービス.
 \item[キャッシュサービス] 他(通常その組織内)の,単体では DNS による名
	    前解決を行えない端末からの,名前解決要求を受け付け,代わりに
	    キャッシュサーバで DNS 名前解決を行い,端末に返答する\footnote{これ
	    を再帰検索と呼ぶが,通常は世界中からの再帰検索を受け付けるこ
	    とはトラフィックの問題からも,セキュリティの問題からの好まし
	    くない}.
\end{description}

\subsection{BIND の全体設定}

まずは,ネームサービス,キャッシュサービスなどの個々の機能に先だって,
BIND の全体設定を行う.

用いるファイルは,下記のファイルである.
\begin{itemize}
 \item \ttt{/etc/named.conf}
\end{itemize}

このファイルで指定すべき情報は,下記である.
\begin{itemize}
 \item BIND がサービスに用いる情報を格納するディレクトリ
 \item BIND のプロセス番号 PID を格納するファイル(起動・再起動・停止な
       どのため)
 \item 管理するゾーン(ドメイン)の情報
\end{itemize}

主な書式を以下に示す.

\begin{cli}
options {
        directory       "/path/to/where";
        pid-file        "/path/to/where";
};

zone "." { type hint; file "/path/to/hint-file"; };
  → ここはルートゾーン解決の設定を行う
     hint file には,リゾルバとしてルートサーバへ
     問い合わせを行うためのルートサーバの
     IP アドレスが書かれている.

zone "some.zone.name.ac.jp" {
        type    master;
        file    "/path/to/zone-file";
};
  → このサーバがネームサービスを行うゾーンの設定
     type master とは,このサーバがこのゾーンの
     プライマリサーバであることを示し,file に
     ゾーン情報が書かれる.
\end{cli}

上記で設定する情報は下記のようにする.
\begin{description}
 \item[BINDが用いるファイルのディレクトリ] \ttt{/etc/named}
 \item[ゾーンファイル名] \ttt{/etc/named/gX.zone} : X はグループ番号
 \item[PIDのファイル名] \ttt{/var/run/named.pid}
 \item[ネームサービスのドメイン名(ゾーン)]
	    \ttt{gX.info.kochi-tech.ac.jp} : X はグループ番号
 \item[ルートサーバの IP アドレスが書かれたヒントファイル]
	    \ttt{/etc/named/named.root} : メインサーバの /pub/data フォ
	    ルダからダウンロードしたものを用いる.なお,本来は BIND に同
	    梱されているもの,あるいは ISC からダウンロードしたものを用
	    いるが,本実験では通常とは異なるルートサーバを指定する必要が
	    あるので,これらを用いると DNS 名前解決が行えない.
\end{description}

なお,キャッシュサーバとしての機能は,BIND の場合はデフォルトで有効になっ
ている\footnote{これは,パラメータ recursion を no に設定することで,無効にする
ことができる.}.

\subsection{ネームサーバのゾーン設定}

プライマリサーバの gX.info.kochi-tech.ac.jp ゾーンの設定(いわゆる正引き)
設定を行うには,named.conf で書かれたゾーンファイルに,ゾーンのホスト名
情報,すなわちレコードを書き込む.

ゾーンファイルの書式は下記のようなものである.

\begin{cli}
$TTL    100   ←デフォルトのキャッシュ保持時間 100 秒
 [[以下の2行は1行で続けて書く]]
@       IN      SOA     gX.info.kochi-tech.ac.jp.
postmaster.gX.info.kochi-tech.ac.jp. ( 
        2012060701   ← シリアル値
        100       ← リフレッシュ時間 100秒
        100       ← リトライ時間 100秒
        1w        ← エクスパイア時間 1週間
        100 )     ← ネガティブキャッシュ時間 100秒
          IN      NS      server.g11.info.kochi-tech.ac.jp.
server    IN      A       192.168.0.X2
 [[この後,同様に必要なレコードを続ける]]
\end{cli}
%$

\paragraph{TTL} 各レコードは,クライアントが問い合わせた後,クライアント
 側でキャッシュされ,何度も同じ問い合わせを行うことはない.このキャッシュ
 を保持する時間を表す.通常は,86400 (秒=24時間)がデフォルトであり,
 これをそのまま用いるが,ここでは短くしておく.

\paragraph{SOA} Start Of Authority レコードは,このゾーンに関する管理情
 報を書く.具体的には,このゾーンのドメイン名,管理者メールアドレス(た
 だし,最初のピリオドは,メールアドレスの"@"を表す),その後,セカンダリ
 サーバ用のゾーン情報更新についての情報を記す.

シリアル値は,このゾーンファイルのバージョンであり,このゾーンファイルを
更新した場合は,必ず値を上げる.通常は,2012060701 のような日付と,2桁の
数字を組み合わせて10桁程度にしておくことが多い.内容を更新する際は,値を
増加させることを忘れないようにする.

リフレッシュ時間は,セカンダリサーバがプライマリサーバに更新を問い合わせ
る時間間隔,リトライ時間はプライマリに通信できない場合の再取得開始までの
間隔,エクスパイアはプライマリにアクセスできない場合にセカンダリの情報を
保持し続ける間隔\footnote{この時間を越えてプライマリが停止すると,セカン
ダリも情報を削除するので,ネームサービスは行えなくなる},ネガティブキャッ
シュは,「存在しない」という情報をクライアントがキャッシュする間隔である.
これらの時間は,ここでは100秒と短くしているが,通常は 3600 秒程度にして
おく.

\paragraph{NS} NSレコードはこのゾーンのネームサーバのホスト名を書く.
SOA に続けて書く場合は,ドメイン名の部分はこのゾーンのドメイン名が補完さ
れるので,左の列は空白で良い.その後,IN NS に続いて,ネームサーバのホス
ト名を書く.続けて,次の行で,次に説明する A レコードを用いて,ネームサー
バのホスト名の IP アドレスを指定する.

\paragraph{A} Aレコードはアドレスを記述する.上記の例のように,ホスト名
(. で終らないホスト名は,このゾーンのドメイン名が補完されるので,ホスト
名だけで良い.FQDN を書く場合は名前の最後に "." を付けることを忘れない),
IN A と書いた後に,IPアドレスを書く.

ネームサーバの A レコードは必ず必要なので,NSレコードのすぐ次に書くよう
にする.

2台以上のネームサーバがある場合は,複数の NS レコード(と対応するホスト
名の A レコード)を書く.

\paragraph{CNAME} ホスト名の別名を設定する.別名であるホスト名を左に書い
た後,IN CNAME に引き続いて,Aレコードで指定される実体のホスト名を書く.

ここでは,Web サーバのホスト名として,別名 ``www'' を作成する.

\begin{itemize}
 \item www → server
\end{itemize}

この他,メールサーバのMXレコードについては,後日のメールサーバの構築の際
に追記するが,ここで書き方だけ書いておく.下記のように,IN MX に続いて,
メールサーバの優先順位(小さい値が優先)を書く.この値は,複数のメールサー
バがあるときに用いられる.

\begin{cli}
mail.domain.jp.        IN   MX  10       server.domain.jp.
\end{cli}

\section{BINDの起動}

BIND の起動を行う.

\begin{cli}
# /usr/local/sbin/named  (DNSサーバの起動コマンド)
\end{cli}

設定を変更する場合は,kill -9 を行ってから再度起動し直すのが最も確実であ
る.

\section{DNS クライアントの設定}

前項で設定した BIND をリゾルバとして用いるために,端末の DNS サーバ設定
を行う.

UNIX系 OS では,\ttt{/etc/resolv.conf} に下記の内容を書く.

\begin{cli}
domain          gX.info.kochi-tech.ac.jp
search          gX.info.kochi-tech.ac.jp
nameserver      192.168.0.X2
\end{cli}

server でも忘れずに設定しておく.

それ以外の OS では,IP アドレスの手動設定を行う部分にて,DNS サーバの IP
アドレスを指定する.

\section{動作の確認方法}

まずサーバで \ttt{dig} コマンドを用いて,以下のように様々なホスト名,IP アド
レスを入力し,DNS サーバの動作状態を確認する.
\begin{cli}
# dig server.gX.info.kochi-tech.ac.jp
 (ホスト名 server のIPアドレスを問い合わせ)
 :
\end{cli}
正しい IP アドレスが引ければ,DNS は正常に動作している.答えが返って来な
い,または,間違った答えが返ってくる場合は,各設定ファイルを再確認する.

Windows や Linux,Mac からは \ttt{nslookup} コマンドで調べることができる.
\begin{cli}
# nslookup server.gX.info.kochi-tech.ac.jp
 (ホスト名 server のIPアドレスを問い合わせ)
 :
\end{cli}

最後に,端末の Web ブラウザなどから,www.gX.info.kochi-tech.ac.jp で Web
ページが検索できれば設定は終了である.他のグループ(例えば, g12 など)
の Web ページも見られることを確認する.



\clearpage