\subsection{BINDのインストール}
BINDのソースファイルをメインサーバからftpを用い取得し,解凍・展開・コンパイルを行い
BINDをサーバにインストールする.

ここで,ファイルを展開したディレクトリ内の README ファイルに一度目を通して,作業を行う.
README にはどのような機能があるか,バージョンごとにどのような機能が追加
されて来たか,どのようにビルドを行うかなどの情報が書かれている.

主なインストール手順を下記に示す.README の詳細も合わせて参照すること.
\begin{itemize}
 \item FreeBSD 標準の古いバージョンの BIND について,コマンド名・ファイ
       ル名の名前を変えておく.
\begin{cli}
#cd /usr/sbin
#mv named ORG.named
\end{cli}
       同様の手順で,\ttt{/usr/sbin} 以下の named-checkconf,
       named-checkzone, named-compilezone, named-journalprint,
       \ttt{/etc/namedb}, \ttt{/usr/bin} 以下の dig, nslookup を名前を変
       えておく.
 \item ftp サーバから,BIND のソースコードをダウンロードし,コンパイル,
       インストールを行う.
 \item デフォルト設定でインストールを行うと \ttt{/usr/local} 以下のコマ
       ンド・ファイル群が生成される.
\end{itemize}

\subsection*{BINDコンパイル時のオプション}
BINDにもMakefileの生成を自動的に行うconfigure スクリプトが提供されている.
本実験でインストールを行うBINDには特別なオプションは必要としない.
