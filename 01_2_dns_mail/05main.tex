\section{目的}

インターネットを使った様々なアプリケーションでは,通信の宛先・送信元を識
別するのにホスト名を用いることが多い.ホスト名とは,通信を行う端末に対し
何らかの意味を持たせた文字列を用いて名前を付けたものである.一方,インター
ネットの通信プロトコル IP では,IPアドレスを用いて端末の識別を行う.そこ
で,アプリケーションが通信を行うためにホスト名を指定した後,IPパケットで
データを伝送するために,そのホスト名に対応する IP アドレスを求める必要が
ある.これを名前解決と呼ぶ.

DNS(Domain Name System)は,インターネットの標準的な名前解決システムで
あり,ホスト名を用いた通信を行うために DNS サービスを構築する必要がある.

電子メールは,DNSを用いる代表的なアプリケーションの一つであり,DNS と同
じく1980年代より使われ,現在でもインターネットでの主要なアプリケーション
として広く利用されている.電子メールで用いられる電子メールアドレスでは,
メールドメインと呼ばれる部分にDNS で指定される名前が用いられる.

\section{DNS とは}
ネットワーク層プロトコルである IP では,ホストの識別をするのに IP アドレ
スを用いる.IP アドレスは32ビットで表される整数であり,表記する際は,
\texttt{192.168.0.1} というように,1 オクテット(8ビット=1バイト)ずつ 
10 進法で,ピリオドで区切って表記される\footnote{現在のバージョン 4 は 
32 ビットのアドレス空間を持つが,次のバージョン 6(IPv6)では,128 ビッ
トに拡大される.}.これを,英語では dotted-quad decimal notation (ドット
付きの4つの10進表記)と呼ぶ.

一方で,ホストの管理は IPアドレスとは独立して組織あるいは運用者単位で行っ
ている.このため,ホストの識別は,会社や企業,部署の構成や,サーバで提供
するサービスに対応して行えることが望ましい.このように,サービスの種類や,
組織の構成などの情報をもとに決めるホストの識別情報として,\textbf{ホスト
名} を用いる.

実際の IP 通信は,IPアドレスを識別情報として通信を行うので,ホスト名とIP 
アドレスの対応付けを行い,適宜変換する仕組みが,名前解決 (Hostname
Resolver) である.名前解決で最も単純なものは,\textbf{Hosts ファイル} と
呼ばれるもので,IP アドレスとホスト名の対応を,テキストファイルで格納し
ておき,OS はその情報を参照して,両者の変換を行う.

Hosts では,大規模なネットワークの名前管理は困難であるため,インターネッ
トのような大規模なネットワークにおいても実用的に管理・運用できる名前解決
の仕組みとして,DNS (Domain Name System)が使われている.DNS により,電
子メールや WWW などのサービスを利用する際に,利用者はホスト名の情報のみ
を用いて \verb|anyone@ugs.kochi-tech.ac.jp| や 
\verb|http://www.kochi-tech.ac.jp/| などのアプリケーション層アドレスを用
いることが可能となっている.

\section{実験内容 (1)}

DNSサービスを構築する.ここでは,DNSサービスとは下記に挙げる2つ,他ドメイン向けにDNSで自サーバ名の名前提供を行うサービスと,自ドメインのクライアントPC向けの他ドメインのサーバ名解決を行うサービスを構築する.

\begin{itemize}
  \item サーバコンピュータでの DNS サービスの構築
	\begin{itemize}
	 \item 自ドメインのネームサーバ(権威ネームサーバ,コンテンツサーバ)の構築
	 \item キャッシュサーバ(フルサービスリゾルバ)の構築
	\end{itemize}
  \item クライアントから DNS サーバ(キャッシュサーバ)への問い合わせ設定(クライアントのリゾルバ設定)
\end{itemize}
である.

\section{必要となる情報}
\subsection*{DNS設定の方針}
本実験では次のような設定の方針に基づいて,DNS を設定する.
\begin{enumerate}
 \item Ubuntu server でネームサーバを構築する.ソフトウェアは BIND9 を
       用いる.
       \begin{itemize}
	\item 各グループごとにドメインを構成し,それぞれのグループのドメインの
	      ネームサービスを server にて行う.
	\item ドメイン名は,\texttt{gX.exp.info.kochi-tech.ac.jp} とする.Xはグループ番号であり,グループ13であれば,\texttt{g13.exp.info.kochi-tech.ac.jp}である.
	\item 各グループのホスト名から IP アドレスを引けるようにする(正引き)
	\item 逆引き設定は行わない(逆引き=内部の IP アドレスからホスト名を引けること).
	\item ドメインの Web サーバは,
	      \texttt{www.domain名} でアクセスできるようにし,
	      実際の Web サーバは,\texttt{server} となるように
	      する.
       \end{itemize}
    \item \texttt{server} にDNSキャッシュサーバ\footnote{フルサービスリゾルバともいうが,ISC BINDでの正確な用語を使えば,各クライアントPCからの再帰問い合わせ(recursive query)要求を受け付け,自らは root server をはじめ,各ドメインのサーバと直接の問い合わせ(非再帰的=non recursive query)を繰り返す (iterative) リゾルバである.}の機能を構築する.
		\begin{itemize}
        \item \texttt{server} 上のDNSキャッシュサーバにて,各グループ
	      内のコンピュータは \texttt{server} をDNSサーバ(再帰リゾルバ)として名前解決できるよ
	      うにする.
		 \item 各クライアントのドメイン名,ネームサーバアドレスを,各クライアントが \texttt{server} の DNS サーバを検索す
		       るよう設定する.
		\end{itemize}
\end{enumerate}

また,各グループ内の端末の IP アドレス,およびホスト名は,
表~\ref{tab:05:addrname}~となる.

\begin{table}
\begin{center}
\caption{各グループ内の端末の IP アドレスとホスト名}%% (ただし,Yはグルー
 %% プ番号 X に 10 を加えた数)
\label{tab:05:addrname}
\vspace*{2zh}
\begin{tabular}{lcl}
\Hline
計算機 & ホスト名 & IPアドレス \\ \Hline
サーバ& \texttt{server} & serverのIPアドレス \\\hline
Ubuntu Desktop & \texttt{linux} & Ubuntu Desktop のIPアドレス \\\hline
Windows 10 & \texttt{win} & Windows 10 のIPアドレス \\\hline
Macbook & \texttt{mac} & DHCPのためここでは設定しない \\\hline
ChromeBook & \texttt{note} & DHCPのためここでは設定しない \\\hline
%ルータの内側I/F & router & ルータのFE1のIPアドレス \\\Hline
%スイッチ & sw & スイッチのIPアドレス \\\Hline
\Hline
\end{tabular}
\end{center}
\end{table}

その他に追加するレコードとして,ネームサーバ(=server 自身)の NS レコー
ドも忘れずに設定すること.

また,今後,Web サーバ,メールサーバをこのドメインに追加設置する際は,
CNAME レコード,MXレコードをそれぞれ追加する必要がある.

\subsection*{設定ファイルの場所}

設定ファイルは以下の場所に置くこと.

\begin{description}
 \item[named.conf] \ttt{/etc/bind}
 \item[ゾーンファイル等のデータを置くディレクトリ] \ttt{/etc/bind}
 \item[ゾーンファイル名] \ttt{/etc/bind/gX.zone}(Xはグループ番号)
 \item[ワーキングディレクトリ] \ttt{/var/cache/bind}
\end{description}

\section{必要となる知識}
先に述べたように,普段我々は計算機を識別するためにホスト名を用いている.
しかし TCP/IP では計算機の識別に IP アドレスを指定しなければならなず,実際の
運用にあたって,この問題を解決しなければならない.ホスト名から IP アドレスを
求めることを{\bf 名前解決}(Name Resolution)といい,以下に示す方法がある.

\subsection*{hosts ファイル}
最も簡単な名前解決の方法は,各ホストが,ホスト名と IP アドレスとの
対応表を持っておき,それを参照するというものである.この対応を記した
ファイルは hosts ファイルと呼ばれ,以下に示すような書式
となっている.hosts ファイルは,通常 \texttt{/etc} の下に置かれる.
ただし,Microsoft Windows では,
\verb|C:\Windows\system32\drivers\etc\hosts| である.


\begin{cli}
#■/etc/hosts ファイルの例

127.0.0.1    localhost
192.168.0.1  mainserver.info.kochi-tech.ac.jp  mainserver

### 上記の例では,ホスト名「mainserver.info.kochi-tech.ac.jp」
### または,「mainserver」と指定したら,192.168.0.1
### と解決することを示す.
\end{cli}

この方法は単純であり,数が限られている組織内であれば,DNSサーバを構築することなく,
全てのホスト名とそのIPアドレスを上記のテキストファイルに記述するだけで,
簡単に実現できるホスト名解決方式である.しかし,大規模なネットワークでの名前解決や,インター
ネット規模での名前解決には適さない.また,動的なネットワーク構成の変更に
は対応しづらい,すべてのホストが同じファイルを保持しなければならない,な
どという欠点がある.比較的ホスト数が少なく,内に閉じた LAN であればこれ
でも構わないが,世界中のホストと通信するインターネットでは,ホスト名よる
名前解決は不可能である.

\subsection*{LAN内のディレクトリサービス}

組織やLANの中であれば,ディレクトリサービスを使ってホスト名情報(hostsファイルに相当する情報)を,クライアントに提供する方法もある.LDAP (Lightweight Directory Access Protocol),NIS (Network Information Service, yp:yellow page(電話帳)とも),マイクロソフトの NT ドメイン を利用したホスト名解決は,現在でも使われる場面がある.

\subsection*{複数の名前解決手段の併用}

hostsファイル,LDAPやNIS,次に説明するDNSなど,複数の名前解決サービス(name service) 手段を併用することができるが,この場合は問い合わせに優先順位が付けられ,先に問い合わせたもので答えが見つかったものから適用される.

この優先順位は,ネームサービススイッチファイル \texttt{/etc/nsswitch.conf}  に書いてある.

例えば,Ubuntu 20.04 では,
\begin{cli}
...
hosts:          files mdns4_minimal [NOTFOUND=return] dns
...
\end{cli}
と書いてあり,files (hostsファイル)が最優先,次に mDNS,それで見つからない場合は,DNSを調べるよう設定されている.

もし,DNS の設定が反映されていないようであれば,hosts ファイルなどが優先されている場合があるので,注意する.

\subsection*{DNS}
インターネットにおける名前解決の方法として用いられているのが DNS である.
前説の hosts ファイルでは,インターネット全体の名前解決は不可能であるの
で,分散管理ができる DNS が用いられている.

DNS は,分散型データベースであり,クライアントからの名前解決要求(ポート
番号 53 への接続)に対して回答を行なう機能を有する.そして,同一ドメイン
内の各ホスト(クライアント)にこのようなデータベース索引サービスを提供
するサーバである.DNS では,サーバでホスト名のデータベースが一元管理さ
れるので,hosts ファイルの欠点であった動的なネットワーク構成の変化にも
容易に対応できる.

DNS では,名前を管理する名前空間を図~\ref{fig:05:dns-domain}のような階層
的なツリー構造で構成しており,ホスト名は,図~ \ref{fig:05:dns-domain}~に
示したツリーのラベルを最下層から順に並べ,``\verb|.|''で区切って表記する.
例えば \verb|www.kochi-tech.ac.jp| や \verb|ugs.kochi-tech.ac.jp| などであ
る.そして,これらの場合 \verb|kochi-tech.ac.jp| をドメイン名という.す
なわちドメインとは,名前空間の分割の単位であり,階層的なツリー構造の名前
空間のあるノードから先を構成するサブツリーである.このドメインを単位とし
て,名前空間での名前の命名・変更の権利を,世界中の組織に分散させる.ある
ドメインが自分のドメインの一部のサブツリーを別の組織の管理に任せることを,
ドメインの委譲と呼び,委譲されたドメインをサブドメインと呼ぶ.世界で唯一
の名前であるドメインが,唯一の組織にサブドメインを委譲する形で名前空間を
分割する.このような名前の命名方法を用いれば,世界中で計算機のホスト名を
重複することなく,命名することができる.なお,ルート階層のラベルは空ラベ
ル(`` '')であり,実際のホスト名では見えなくなっている.ルート階層での
命名権は IANA にあり,IANA がルートのサブドメインであるトップレベルドメ
イン(TLD)をサブドメインとして定義し,それぞれのサブドメインを委譲する.
   
図~\ref{fig:05:dns-dns}~ は DNS による名前解決処理の様子を示している.
DNS では階層的な名前空間を用いているため,ホスト名 
\verb|www.kochi-tech.ac.jp| のアドレスを取得する際の処理は次のようになる.

\begin{figure}[ht]
  \begin{center}
   \includegraphics[width=.5\linewidth]{\chapdns dns-domain.eps}
   \vspace*{1zh}
   \caption{階層型ドメイン空間}
   \label{fig:05:dns-domain}
  \end{center}
 \end{figure}

 \begin{figure}[ht]
   \begin{center}
    \includegraphics[width=.8\linewidth,bb=0 0 720 540,clip]{\chapdns dns1.pdf}
    \vspace*{1zh}
    \caption{DNSによる名前解決}
    \label{fig:05:dns-dns}
   \end{center}
  \end{figure}
  
  \begin{enumerate}
   \item クライアントがローカルネームサーバに %
         \verb|www.kochi-tech.ac.jp| のアドレスを問い合わせ.
   \item ルートネームサーバにアドレスの問い合わせ.ルートネームサーバは %
         \verb|jp| ネームサーバのアドレスを教える.
   \item \verb|jp| ネームサーバにアドレスの問い合わせ.\verb|jp| ネームサー
         バは \verb|ac.jp| ネームサーバのアドレスを教える.
   \item \verb|ac.jp| ネームサーバにアドレスの問い合わせ.\verb|ac.jp| ネー
         ムサーバは \verb|kochi-tech.ac.jp| ネームサーバのアドレスを教える.
   \item \verb|kochi-tech.ac.jp| ネームサーバにアドレスの問い合わせ.%
         \verb|kochi-tech.ac.jp| ネームサーバは\\
         \verb|www.kochi-tech.ac.jp| のアドレスを教える.
   \item ローカルネームサーバがクライアントにアドレスを応答する.
  \end{enumerate}
  
DNS サーバは世界中のコンピュータのホスト名と IP アドレスを知っているわけではな
い.名前空間の第一階層にある \verb|jp| ネームサーバは,
第二階層である \verb|ac.jp|,\verb|co.jp|,\verb|ne.jp| などドメインのネー
ムサーバのホスト名と IP アドレスを知っている.
同様に,\verb|kochi-tech.ac.jp| ネームサーバは,
\verb|ugs.kochi-tech.ac.jp|,\verb|www.kochi-tech.ac.jp| のホスト名と IP %
アドレスを知っているのである.このようにおのおののネームサーバ
は,ゾーンとよばれるドメイン空間の一部についての情報(データベース)を
完全に保持している.この場合,「ネームサーバはこのゾーンに対する
権威(authority)をもっている」という.

図~\ref{fig:05:dns-domain}~のドメイン名空間の場合,ホスト名から IP アドレスを検
索することはできるが,IP アドレスからホスト名を検索することができない.
つまり,このドメイン名空間ではドメイン名を索引としてしているため,ドメ
イン名から IP アドレスを検索することは容易であるが,IP アドレスからドメイ
ン名を検索するためにはすべてのデータを調べなければならないからである.
そこでDNSでは,図~\ref{fig:05:dns-ip-domain}~ に示すようなアドレスをラベルとす
る \texttt{in-addr.arpa} というドメインを作りこの問題に対応している.

  \begin{figure}[hbtp]
   \begin{center}
    \includegraphics[width=.5\linewidth]{\chapdns dns-ip-domain.eps}
    \vspace*{1zh}
    \caption{in-addr.arpaドメイン名空間}
   \label{fig:05:dns-ip-domain}
   \end{center}
  \end{figure}

図~\ref{fig:05:dns-ip-domain}~に示すのは,\verb|www.kochi-tech.ac.jp| の IP アド
レス 210.163.144.42 の \texttt{in-addr.arpa} ドメイン名である.ドメイン名の読み方の
規則に従うと,このホストの \texttt{in-addr.arpa} サブ
ドメイン名は \texttt{42.144.163.210.in-addr.arpa} となる.DNS はこ
のような IP アドレスを索引とするドメイン名空間により,ホスト名から IP アド
レスを検索するのと同じ仕組みで IP アドレスからホスト名を検索することがで
きる.

以上から DNS は,ゾーン内のホストの,ホスト名から IP アドレスを検索するデー
タベース,IP アドレスからホスト名を検索するデータベース,ルートネームサー
バの情報を保持する必要がある.また安全性の観点から,DNS は通常,プラ
イマリマスタとスレーブと呼ばれる 2 つのサーバで運用する.スレーブはプラ
イマリマスタのデータベースファイルを定期的にコピー,バックアップし,プ
ライマリマスタが停止している時などは,プライマリの DNS サーバとして機能
する.

\subsection*{BIND}

DNS サービスのための実装プログラムとしては BIND を用いる.BIND は,
California 大学 Berkeley 校(UCB)で開発・実装された DNS サービスのため
のプログラムで,4.2 BSD UNIX 上で最初に実装された.現在はISC(Internet
Systems Consortium)によりメンテナンスが行わており,同サイトで最新版が活
発にリリースされている.

UNIX はもちろん,その他のプラットフォーム上で利用できる DNS サーバは,ほ
とんどがこの BIND のプログラムをもとに開発されている.また,BIND が DNS
プロトコルのリファレンス実装\footnote{同プロトコルのサービスプログラムとして中心
的な役割を担い,DNS プログラムを開発する人は BIND を参考にすべきである,
という位置付け}となっている.

\clearpage

\section{DNSサービスの構築の手順}

\subsection{BINDのインストール}

Server にパッケージシステムを用いて BIND をインストールする手順を説明す
る.

パッケージシステムとは,Linux のディストリビューションや FreeBSD, NetBSD 
など,OSごとに様々なオープンソースなどの追加ソフトウェアについて,ダウン
ロード・コンパイル・インストールなどの処理をパッケージ化して,簡単なコマ
ンドのみでインストールできるように準備されたものである.

Debian 系(Ubuntu)では apt コマンド,Redhat系(CentOS, SuSE等)では yum コ
マンドを用いる.


\begin{cli}
$ sudo su
   (↑管理者権限になる)

# apt update
   (↑パッケージサーバに最新情報を照会)
# apt install bind9
   (↑BIND 9 をインストール)
\end{cli}%$

これで,インストールは終了である.DNSの設定ファイルなどは,Ubuntu のパッ
ケージの場合,\texttt{/etc/bind} に格納される\footnote{BINDをソースコー
ドからインストールした場合は,\texttt{/etc/named}である.}.

\subsection{BINDの全体設定}

まず,BIND の全体設定として,設定ファイル named.conf の設定を行う.

\begin{cli}
# vi /etc/bind/named.conf
\end{cli}

このファイルは,BIND 全体の動作を設定する.
具体的には,どのドメインを管理するのか,そのドメインに関する情報はどのディ
レクトリに格納するか,マスターかスレーブか等である.

Ubuntu のパッケージの場合,\texttt{named.conf} の内容は以下のようなものとなってい
るはずである.
\begin{cli}
// This is the primary configuration file for the BIND DNS server named.
//
// Please read /usr/share/doc/bind9/README.Debian.gz for information on the 
// structure of BIND configuration files in Debian, *BEFORE* you customize 
// this configuration file.
//
// If you are just adding zones, please do that in /etc/bind/named.conf.local

include "/etc/bind/named.conf.options";
include "/etc/bind/named.conf.local";
include "/etc/bind/named.conf.default-zones";
\end{cli}

これは,具体的な設定はここでは行っておらず,include の右側で指定される別ファイ
ルの内容を取り込み,設定に反映されることを示している.すなわち,named.conf で直接設定を行うの
ではなく,指定されている各ファイル内での設定を推奨していることを示している.
以下ではそれぞれのファイルでの設定内容について示す.

各ファイルでの設定を行う前に,以下のようにしてすでに配置されているファイルを
別名で待避させておく.
\begin{cli}
# cd /etc/bind
# cp -p named.conf.options named.conf.options.org
\end{cli}

\texttt{cp} コマンドはコピーを行う(本テキスト末尾の付録参照)が,\texttt{-p} オプションを付けることで,所有者,パーミッション,タイムスタンプ(作成時刻,変更時刻,アクセス時刻)などを保存 (preserve) してコピーする.付けない場合は,所有者は実行者,タイムスタンプは実行時,パーミッションもデフォルトのものに変更されてコピーされるため,ファイルバックアップには適さない.

named.conf.options 内には BIND の動作に関するオプションを記述する.ワーキン
グディレクトリの指定はここで行う.named.conf.options を新規に作成し,次のよ
うに記述する.
\begin{cli}
options {
    directory    "/var/cache/bind";
};
// BIND が動作する際のワーキングディレクトリ
\end{cli}

named.conf.local 内で自分が管理するドメインを zone で指定する.named.conf.local 
を新規に作成し,次を参考に記述する.以下の例は,gX.info.kochi-tech.ac.jp 
ドメインの情報を,ゾーンファイル \texttt{/etc/bind/gX.zone} に記述する場合である.
\begin{cli}
//↓下記記述中の X はグループ番号にすること
zone    "gX.exp.info.kochi-tech.ac.jp" {
    type    master;
    file    "/etc/bind/gX.zone";
};
// gX... ドメインのプライマリサーバであり,その情報はゾーンファイル
// に書かれていて,それを用いて解決する.file にはゾーンファイルの
// パスを書く.自分がセカンダリの場合は,type を slave とし,
// ゾーンファイルは自ら作成せずに,代わりに
// プライマリサーバの IP アドレスを指定し,そこからゾーン転送してくる
\end{cli}

named.conf.default-zones 内で,ルートゾーンなどが既に指定されているので,確認する.
"." はルートゾーンを表しており,
ルートサーバにに問い合わせるよう設定されている(問い合わせタイプ=hint).
ヒントファイル \texttt{/usr/share/dns/root.hints} には,ルートサーバの IP アドレスが記述されている.めったにないが,ルートサーバの IP アドレスが変更され世界中のDNSサーバの設定変更が要請される場合がある.このようなときは,ヒントファイルを最新のものをダウンロードして差し替えるか,新しいルートサーバのIPアドレスに修正する.
\begin{cli}
zone "." {
    type    hint;
    file    "/usr/share/dns/root.hints";
};
// ルートドメインは,ヒントファイル(通常は A から M までの13個のルートサーバのIPアドレスを記載)
// を使って解決.ヒントファイルのパスを file で示す.
\end{cli}

\subsection{ドメイン(ゾーン)の設定}

BIND の全体設定が終わったら,自ドメインのドメイン情報(ゾーン情報)をゾーンファイルに記述する.
ゾーンファイルでは,そのドメインの IP アドレス・ホスト名等の情報を記述する.

\begin{cli}
# vi /etc/bind/gX.zone
ゾーンファイルに記述する内容はこの先のBINDゾーンファイル例を参照.

(以下は任意だが行うのを推奨)
# named-checkconf /etc/bind/named.conf
# named-checkzone (管理ドメイン) /etc/bind/gX.zone

# systemctl start bind9

(各種動作確認)

\end{cli}

動作確認でうまくいっていなければ,設定ファイルを確認する.
ログ等も必要に応じて確認する.

ログは,/var/log/syslogなどを確認する.

\textbf{注意:設定ファイルを書き換えた場合は再度 named を再起動する.そ
うしないと,設定が反映されない.これは,postfix 等,systemctl で動作を制
御する多くのソフトウェアで同様である.}

\begin{cli}
# systemctl restart bind9
\end{cli}

\subsubsection{BINDゾーンファイルの書式}

BINDでのゾーンファイルの書き方は下記の通りである.

ゾーンファイル内では,\texttt{named.conf.local}の zone で指定したゾーン(ドメイン)名が,ホスト名に補完して追記され,完全なドメイン名(FQDN)となる.「.」で終わるホスト名は補完されず,それ自体が完成した FQDN ホスト名である.

\begin{cli}
$TTL       1H
(TTLは,キャッシュの有効期限.DNSを変更しても世界中のDNSサーバのキャッシュに情報が残り,これは最大でTTL時間は古い情報が保持される.)

@          IN      SOA     ドメイン名のFQDN. (空白) メールアドレス.(ただし,@を.に置き換えたもの)
         (
                   2011072504           ; serial number 何でも良いが日付+番号が良く使われるゾーンファイルを修正したら必ずこの値を上げる.上げないとバージョンが上がっていないと認識され,セカンダリサーバ(スレーブサーバ)で内容が更新されない.
                   3h                   ; refresh after an hour セカンダリサーバがマスターに情報更新がないか問い合わせる間隔
                   1h                   ; retry after 1 hour  リフレッシュに失敗した場合に再度リフレッシュするまでの間隔
                   1w                   ; expire after a week マスターにリフレッシュできない場合にスレーブがいつまで情報を保持するかの時間
                   1d )                 ; minimum TTL is one day  ネガティブキャッシュ=存在しないという情報をいつまでキャッシュさせるか
                IN      NS      ネームサーバのFQDN
ホスト名(FQDN)  IN      A       IPアドレス
別名            IN      CNAME   実ホストのFQDN
メールドメイン  IN      MX      メールサーバのホスト名
(メールドメインとは,メールアドレスの@の右側である)

\end{cli}
%$

\subsubsection{DNSの設定項目の詳細}
BIND が DNS サービスを提供するには,以下のファイルを作成する必要がある.
\begin{itemize}
\item データベースファイルが置かれたディレクトリ情報や named の起動オプ
      ションなどを記述したコンフィグファイル (named.conf)
\item 各ドメインごとのホスト名と IP アドレスの対応を記述したデータベース
      情報を格納するゾーンファイル
\item ルートサーバの IP アドレスを記述したヒントファイル
\end{itemize}

設定ファイルの作成には,ファイルを作成する場所(ディレクトリ)や,ピリ
オド``\verb|.|'',セミコロン``\verb|;|''に十分注意をして,vi 等のエディ
タで作成する.ここでの内容を間違って設定すると,DNS は正しい挙動を示さず,
今後作成するサーバは正常に動作しなくなる.
  
ゾーンファイルの各行の意味は下記の通りである.

左端のカラムがホスト名であり,. (ピリオド)で終わるホスト名は FQDN,そ
うでないものはゾーンのドメイン名が補完される.名前が @ となっているもの
は,ドメイン名が補完される.空白のホスト名は,直前の行のホスト名が補完さ
れる.

\begin{description}
 \item[SOA] SOAレコードには,サーバの FQDN,メールアドレスを表す FQDN(メー
	    ルアドレスの @ のかわりに . に置き換える),
	    さらにかっこの中に,シリアル番号(日付と通し番号が良い),リ
	    フレッシュ時間,リトライ時間,expire 時間,TTL などを書く.
	    これらの時間は,実験においては短くしておく.
 \item[NS] ネームサーバの FQDN.自分自身も定義する.セカンダリがある場合はこれも書く.
 \item[MX] メールサーバの FQDN.MX の後に優先順位を決める番号を書く.小さい数ほど優先される.
 \item[A] ホスト名と IP アドレスを書く.
 \item[CNAME] ホスト名の別名を書く.
\end{description}

named.confファイル,ゾーンファイルの作成が終了したら,設定ファイルの文法的なチェックを行う.
コマンド実行の結果,エラーが出力されないことを確認する.

root.hints (システムによっては named.root とも)ファイルは下記のような意味を持っている.
\begin{itemize}
\item ルートサーバの情報を記述するヒントファイル.
\item 世界の 13 のルートネームサーバをこのファイルにすべて記述する.
\item ヒントファイルは最新のルートネームサーバの情報を記述
\end{itemize}

以上の設定を行った上で,サーバのリゾルバ設定を変更する.
新しく立ち上げたDNSサーバアドレス(自分自身のIPアドレス)に変更し,
同時にドメイン名も変更する.

%%%%%%%%%%%%%%%%%%%%%%%%%%%%%%%%
\subsection{本実験でのBINDゾーンファイル例}

本実験でのゾーンファイルの例は下記の通りである(下記はグループXの例であり,グルー
プ番号は自分のものを使う).

\begin{cli}
$TTL    100
@       IN   SOA  server.gX.exp.info.kochi-tech.ac.jp.  postmaster.gX.exp.info.kochi-t
ech.ac.jp.  (
        2020041701
        100
        100
        100
        100 )
        IN      NS      server.gX.info.kochi-tech.ac.jp.
server  IN      A       サーバのIP
www     IN      CNAME   server
linux   IN      A       デスクトップlinuxのIP
win     IN      A       Windows 10 の IP
\end{cli}


\subsection{DNSサーバの起動と動作確認}

bind の起動を行う.

\begin{cli}
# systemctl enable bind9
    ↑再起動時に BIND の自動起動

# systemctl start bind9
    ↑BIND の(手動での)起動
(直接コマンドで起動する場合は /usr/sbin/named だがパッケージでは通常用いない)

# ps auxww | grep named
    ↑named プロセスがあることを確認
\end{cli}

もし起動していなければ,ログなどでエラーを確認し修正する.
\begin{cli}
# cd /var/log
# less syslog (など)
 …
\end{cli}


\subsection{サーバOSでの参照先DNSサーバ(リゾルバ)設定}

構築したDNSキャッシュサーバを参照するよう,DNSクライアントの設定を変更する.

現在は,高知工科大学DNSキャッシュサーバ 172.30.0.2 が設定されているはずなので,これを今回構築したDNSサーバのIPアドレスに変更する.

\begin{itemize}
  \item サーバ\\
	\ttt{/etc/netplan/} 以下の YAML ファイルを修正する.
\begin{cli}
      nameservers:
        addresses:
          - DNSキャッシュサーバのIPアドレス(自分自身なら127.0.0.1)
        search: [自グループのドメイン名]
\end{cli}
\end{itemize}

netplan 設定後,有効化する.

\begin{cli}
# netplan apply
\end{cli}

サーバで \texttt{dig} コマンドを用いて,
以下のように様々なホスト名,IP アドレスを入力し,
DNS サーバの動作状態を確認する.
\begin{cli}
# dig server.gX.exp.info.kochi-tech.ac.jp
 (ホスト名 server のIPアドレスを問い合わせ)
 :

\end{cli}
正しい IP アドレスが引ければ,DNS は正常に動作している.答えが返って来な
い,または,間違った答えが返ってくる場合は,各設定ファイルを再確認する.
また, \texttt{ /var/log} 以下のファイルにエラーログが記録され
るので参考にするとよい.
\begin{cli}
# cd /var/log
# less syslog (など)
 …
\end{cli}

また,nslookup コマンドや,pingコマンドでも確認する.

\begin{cli}
# nslookup server.g12.exp.info.kochi-tech.ac.jp
# ping server.g12.exp.info.kochi-tech.ac.jp

nslookup や ping では,自ドメイン名を省略できる.

# nslookup server
# ping server

\end{cli}

\subsection{クライアントOSのリゾルバ設定}
 サーバでの名前解決が正常に行えるようになったら,各クライアントのリゾルバ設定を
 立ち上げたDNSサーバに変更する.

\begin{itemize}
  \item Windows 10\\
	「コントロールパネル」→「ネットワークとインターネット」→「ネッ
	トワークと共有センター」→「アダプタの設定の変更」から,用いてい
	るネットワークインターフェースのプロパティから,IPv4 の設定の項
	の DNS サーバを変更する.
  \item Ubuntu Desktop\\
    /etc/netplan/ 以下のYAMLファイルをサーバと同様に変更する.ただしDNSキャッシュサーバのIPアドレスに気を付ける.
\end{itemize}

\section{発展:Macbook, ChromeBook の固定IP設定}

現在,Macbook, ChromeBook は DHCP によるIPアドレス動的割り当て(自動割り当て)になっており,DNSキャッシュサーバもDHCPにより大学のアドレスが設定されている.

これを,固定IPアドレスにし,DNSサーバも今回設定したサーバに変更する.

用いるIPアドレスは,DNSの設定方針にある表の通りとする.


\section{動作確認}
 自グループで構築したDNSサーバを利用し,
 \begin{itemize}
     \item 自グループ内のホスト名の正引き
     \item 他グループのホスト名の正引き
     \item 外部ネットワークの名前解決
 \end{itemize}
 が行えるか確認を行う.dig, nslookup で確認する.

また,実際のアプリケーション動作確認として下記も行う.
 \begin{itemize}
     \item 自グループマシンへの ping をホスト名で実施
     \item 自グループWebサーバを \url{http(s)://www.gX.exp.info.kochi-tech.ac.jp/} でアクセス
     \item 外部 Web サービスへアクセス
 \end{itemize}

 
\section{考慮すべき点}
\paragraph{名前の解決方法}
			TCP/IPネットワーク上で名前解決を行うにはどのような技術が存在し,
			それぞれどのようなメリット・デメリットがあるか.
\paragraph{DNSによる名前解決の仕組み}
		DNSとはどのような仕組みで名前の解決を行うのか.
		また,リゾルバ・キャッシュサーバはどのように違うものか.
\paragraph{管理外ホストの名前解決}
		自分の管理外のホストに対しどのように名前の解決を行うのか.
\paragraph{hostsファイルなどとの併用}
	hosts ファイルなど,他の名前解決サービスとDNSサーバを併用するときの注意.この挙動を変更したい場合はどうするか.
 
