

レポートは,下記の体裁にて提出すること.\textbf{全てのレポートについて,
1つでも下記に該当する場合,採点および単位認定されない} ので注意すること.

\begin{itemize}
 \item \textbf{指定された期日・時刻までに提出されない場合}
 \item \textbf{指定された様式でない場合,特に抜けている項目がある場合}
\end{itemize}

加えて,\textbf{下記の場合は,単位認定されない上に,不正行為として処罰さ
れる行為}であるので,厳に慎むこと.

\begin{itemize}
 \item \textbf{他人のレポートからの転載}
 \item \textbf{他の Web ページ,本などからの転載}
\end{itemize}

本科目のレポートは,全て,KUT LMS (Learning Management System,
https://lms.kochi-tech.ac.jp )の本科目の項目から PDF にて提出する.提出
されたレポートは,\textbf{Turnitin システムにより,過去も含む他人のレポー
ト,Webページ,出版された本などと一致する文章があるか照合され,剽窃チェッ
クが行われる.}

LaTeX (platex) にて,実験の冒頭で指定されたテンプレートを用いて作成する
こと.

\textbf{章立ては下記の通りとすること.下記の項目に従っていないもの,一部の項目に
抜けがある不完全なものは,提出されても受理しない.}

\begin{description}
 \item[1. 目的] それぞれの回で,どのようなことができるシステムを構築する
	    のかと,なぜそのシステムが必要であるか,構築することでどのよ
	    うな機能・サービスが実現できるのかを述べる.ただし,自らの学
	    習目的,学ぶ内容を説明しないように注意する(○○を習得する,
	    学ぶ,などと書かない)(2点満点).
 \item[2. 内容] 構築するシステムの技術的な説明(どのような要素技術を用い
	    て,どのような仕組み・原理で実現するのか)を,システムの全体
	    像が分かるような\textbf{説明図を1つ以上含めて説明する}.図を用いて,
	    実際の挙動の例などを説明する(2点満点).
 \item[3. 要素技術] システムで用いられる要素技術について,指定されたもの
	    について,それがどのようなものであるか,技術的背景,仕組み,
	    構造,動作について説明する.ただし,あくまで技術の解説をし,
	    単なる用語説明にならないように注意する.また,適切な参考文献
	    を挙げて根拠を示すこと(2点満点).
 \item[4. 作業記録] システムを構築するために,具体的にどのような作業をし
	    たかを記録する.用いたソフトウェア,行った操作,コマンド,設
	    定した内容,編集したファイルなどをもれなく書く.システムが壊
	    れた場合など,一から構築するときも,記述された作業を行うこと
	    で,全く同じシステムを構築することができるように書く(2 点満
	    点).
 \item[5. 考察] 考察のポイントを踏まえて,取り扱うシステム・技術の利点や
	    欠点,その技術の本質,類似システムとの相違・相似などの比較・
	    考察を行い,自分なりの考え,結論を書く(結論は感想ではない.
	    ○○という技術は▽▽の一種であるから本質的にこのような問題を
	    抱えている.ゆえに,将来□□を改良・解決した◇◇などの研究・
	    開発が重要である,など)(2点満点).
 \item[参考文献] 第1節から第5節までで述べたことのうち,自ら考案したオリ
	    ジナルの意見でなければ(あるいは考案したことであっても,過去
	    同じことが発明・提案されているならば),その内容について述べ
	    ている文献を適切に挙げること.記述の例として,下記のように書
	    くこと.\textbf{最低5つ程度の参考文献は挙げること.また,一
	    般のWebページの引用は原則できない.文中で適切に引用すること.
	    参考文献がほとんどないものは項目抜けとして扱う.}

	    \begin{itemize}
	     \item 本を参考にする例\\

	    (例) 佐伯サチオ,福冨エイジ著,福本昌弘監修,吉田真一訳:
	    「情報実験の方法論」,pp. ○〜○ (どのぺージに書いてあるか),
	    KUT出版,2012(発行年).

	     \item 雑誌を参考にする例\\

	    (例) 吉田真一:「情報実験の方法論(記事名)」,○○雑誌名,
	    ○巻,○号,pp. ○〜○ (ぺージ to ページ),KUT出版,2012 
	    (発行年).

	     \item 英文等の例\\

	    (例) Shinichi Yoshida, Sachio Saiki, and Eiji Fukutomi, ``
		   Computer Network,'' Journal of KUT, Vol. 3, No. 4,
		   pp. 501-511, 2012.\\

		   2種類のダブルクォーテーション,始まり「``」(LaTeXでは,
		   バッククォーテーション \verb|`| を2つタイプ \verb|`| \verb|`|) と,
		   終わり '' (LaTeXでは,シングルクォーテーション
		   \verb|'| を2つタイプ \verb|'| \verb|'|)に注意し,項
		   目の区切りには半角コ
		   ンマ「,」を使い,コンマの後ろには必ず半角空白を1つタ
		   イプする.終わりのダブルクォーテーションの後に項目を
		   続ける場合は,コンマはダブルクォーテーションの前に入
		   れ,最後は半角ピリオド「.」で終わる.


	    \end{itemize}

 \item[(付録や感想)] その他,気がついたことで報告しておきたいことを付
	    録として報告しても良い.また感想を付けても構わない.ただし,
	    節番号を付けないこと.ただし,「付録」「感想」とも直接的には
	    採点の対象とはならない.また,「付録」「感想」に書かれた内容
	    は,レポートの採点には(良い方にも悪い方にも)影響しない.

\end{description}

忘れがちな点を下記に示す.
\begin{itemize}
 \item {\bf 日本語文の場合,コンマ「.」やピリオド「.」は全角を
		   用いる.}
 \item {\bf 英文の場合,コンマ「,」やピリオド「.」は半角を用い,コン
       マの後に半角空白を1つ,ピリオドの後に半角空白を2つタイプする.}
 \item 意味的に1つのトピックごとに段落分け(パラグラフ)し,パラ
       グラフの区切りには,\LaTeX の場合,{\bf 空行を1行}入れると,自動的に次
       段落1行目を1文字インデントするので,これを用いる.{\bf 強制改
       行「\verb|\|\verb|\|」あるいは「\bf \yen\yen」は段落分けには用いず},段落の1行目の最初にも全角空
       白「 」は挿入しない.
\end{itemize}


分量は特に指定しないが,目的,内容,考察の各項目を十分に説明するには,そ
れぞれ,数百字程度は必要であると期待される.また,要素技術について
も,それぞれの技術項目について数百字程度は期待される.

\textbf{著作権 (Copyright)} やその他の知的財産権(Intellectual Property),
の処理については,十分に注意をすること.具体的には,著作者の許可を得ずに
本・教科書・Webページ等の記述・図・表の転載などを行わないようにすること.

また,\textbf{倫理上の問題やコンプライアンス(法令,政令等の法律や国の規定,通達,学則
や「高知工科大学における研究活動の不正行為への対応等に関する規程」等の学
内規則,判例等の司法判断,一般的な常識(社会通念上の規範)の遵守)} につ
いても十分留意すること.具体的には,不正と認定されるような行為(例えば他
人のレポート,文書,本,雑誌,Webからの転載)などを行わないこと.

最初にも述べたが,\textbf{他の文書(インターネット上の情報や他人のレポー
ト等)から文書の転載は,上記に違反するものであり,Q末試験におけるカンニ
ング等の不正行為と同様のものとして扱う.}

