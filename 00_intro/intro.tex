%
%   はじめに
%        written by Shinichi Yoshida
%

昨今,サブプライム問題,そして,リーマンショックを起点とするアメリカ経済
の破綻,そして,日本への波及による日本経済のかつてない冷え込み,就職問題
の表面化,さらには,ギリシャなどヨーロッパ諸国への波及など,世界経済は,
全世界的な恐慌状態となり,1929年の世界大恐慌以来といっても過言ではない不
況にみまわれている.

こうした経済状態のもとで,雇用状態も急速に悪化を一途をたどり,大手,中小
を問わず企業の倒産が相次いでいる.労働者にとっても解雇,採用の減少などの
雇用情勢の悪化で,職につくことのできない人々であふれている.情報産業や通
信業界,電機業界においても厳しい世相であることには変わりない.しかし,イ
ンターネットやコンピュータの個人への普及は止まることがなく,またコストダ
ウンの流れはオンライン上でのビジネスの発展につながっており,インターネッ
トを中心とする情報通信産業の技術者が求められている.

オンラインショッピングやネットオークション,ビデオ会議システム,ビデオ
オンデマンド,IP電話,メッセンジャーやチャット,ブログやSNSの普及な
ど,新たなサービスがオンライン上に出現するごとにインターネット技術は複
雑化してきている.こうした技術を支える技術者は,今後もなくなることはな
い.しかし,複雑化したシステムに対応できる優秀な技術者でないと生き残る
こともできない.

本実験では,こうしたインターネット技術を支える技術について,実際に構築を
しながら,幅広い角度からで学ぶ.本実験で扱う技術は,必ずしも最新のサービ
スではなく,今後も長く使われ続ける基礎技術に重点を置いている.実験3 では,
FreeBSD や Linux,Windows といった広く普及しているオペレーティングシステ
ムを用いて,電子メールやWWWなどの身近に使われているサービスの構築を行う.
実験4では,一転してアクセス回線の構築,ルーティング,ネットワーク設計,
セキュリティといったインターネットを支える基盤技術の構築を学ぶ.

本実験の内容は初学の者にとっては十分に複雑で難しいものであるかも知れな
い.しかし,本実験の内容はまだまだ基礎であり,この先にはさらに複雑で設
計の難易度の高いシステムが数多く存在している.ぜひ,この実験を通して基
礎技術,基盤技術を身に付け,将来,更なる発展技術の習得,および新技術,
新サービスの開発へと進んでいき,インターネットの発展に寄与して欲しいと
考える.
