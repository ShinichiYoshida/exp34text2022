%% %% %% %%
%% %% %%
%% %% %%  traceroute
%% %% %%
%% %% %%

\section{traceroute, tracert}
パケットをネットワーク上での経路を調査する\par
\label{cmd:traceroute}
\noindent
{\bf ◆書式}
\begin{center}
\begin{screen}
\begin{alltt}
% traceroute [宛先ホスト名 or IPアドレス]
\end{alltt}
\end{screen}
\end{center}

\noindent
{\bf ◆機能説明}

ping は特定ホストとの間でパケットの送受信ができるか調べるだけであるが,
traceroute を用いると,さらに,どのような途中経路(ルータ)を通って宛先
までパケットが到達するかを調べることができる.

状況によっては,このコマンドを使うことができない場合,途中一部分が表示さ
れない場合,途中から先が表示されない場合など,完全に動作しない場合もある.
(ファイアウォールやセキュリティ関連の設定で,許可されていない場合が多い)

また,状況によっては,ping -r コマンドで,行き・帰りのルートを調査するこ
とが可能な場合もある.

-n オプションで,IP アドレスのみを表示させる(ホスト名を解決しない),-P 
 オプションで送出するパケットの種類を指定(icmp, udp, tcp:デフォルトは 
 udp)することができる.

この他にも多くのオプションがあるが,うまく動作しない場合,オプションを変
えると調査可能な場合がある.

Windows では古いバージョンで8文字までのコマンド名しか許されなかったため,
tracert となっている.

\noindent
{\bf ◆使用例}
\begin{center}
\begin{breakbox}
\begin{alltt}
% \underline{traceroute -n -P icmp 172.30.0.1}
traceroute to 172.30.0.1 (172.30.0.1), 64 hops max, 60 byte packets
 1  * * *
 2  172.21.30.10  0.228 ms  0.235 ms  0.245 ms
 3  222.229.72.1  3.111 ms  2.490 ms  2.498 ms
 4  172.17.1.1  2.611 ms  2.497 ms  2.368 ms
 5  222.229.65.41  2.366 ms  0.743 ms  0.744 ms
 6  222.229.65.25  1.117 ms  1.118 ms  0.994 ms
 7  172.30.0.1  1.117 ms  0.994 ms  0.994 ms
%
(宛先ホストまで,6つのルータを通過しており,それぞれの IP アドレス
 が表示されている.1つ目のルータは調査できなかったことを示している.)
\end{alltt}
\end{breakbox}
\end{center}
\clearpage
%% %% %%
%% %% %% %%