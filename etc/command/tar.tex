%% %% %% %%
%% %% %%
%% %% %%  ps
%% %% %%
%% %% %%

\section{tar(Tape ARchive)}
複数のファイルを一つのファイルにまとめるアーカイブ化およびアーカイブの展
開を行う.\\
\label{cmd:tar}
\noindent
{\bf ◆書式}
\begin{center}
\begin{screen}
\begin{alltt}
tar [xtcvzf] filename
  x : extract, アーカイブを展開する
  t : list, アーカイブの内容(ファイル)を表示する
  c : create, アーカイブを作成する
  v : verbose, 実行時に詳細の内容を表示する
  z : 圧縮を行う(tar 内部で gzip 等の圧縮ソフトウェアを用いる)
  f : filename, アーカイブファイル名をこの後に続いて書く
\end{alltt}
\end{screen}
\end{center}

\noindent
{\bf ◆機能説明}

tar は,UNIX で標準のアーカイバソフトウェアである.バックアップの際など
に,ディレクトリごとまとめたい場合に用いたり,アーカイブ化されたファイル
を展開する際に用いる.

サーバ等のフリーソフトェアのソースコード配布も,tar (+ gzip ) 形式が多い
ので,tarを用いて展開しコンパイルする.

\noindent
{\bf ◆使用例}
\begin{center}
\begin{breakbox}
\begin{alltt}
freebsdx# \underline{tar cvzf archive.tar.gz folder}
(フォルダ folder 以下の全てのファイルを achive.tar.gz というファイルにアー
 カイブする.)
freebsdx# \underline{tar xvzf software.tar.gz}
(現在のフォルダに software.tar.gz にアーカイブされている全てのファイルを
 展開する.注意点として,software.tar.gz が単一フォルダ以下で構成されてい
 る場合は問題ないが,そうでない場合は,多くのファイルがカレントフォルダ
 に作成されるので,その際は,mkdir で新しいフォルダを前もって作成し,
 その中で展開するなどする.単一フォルダか否かの内容の確認は下の例を参照.)
freebsdx# \underline{tar tvzf software.tar.gz}
  (内容の確認.実際にはファイルは作成されない.)
\end{alltt}
\end{breakbox}
\end{center}
\clearpage
%% %% %%
%% %% %% %%
