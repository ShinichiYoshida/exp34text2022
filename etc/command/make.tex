%% %% %% %%
%% %% %%
%% %% %%  make
%% %% %%
%% %% %%
\label{cmd:make}
\section{make}
プログラムの依存関係をメンテナンスする\par

{\bf ◆書式}
\begin{center}
\begin{screen}
\begin{alltt}
 % make ターゲット
\end{alltt}
\end{screen}
\end{center}

{\bf ◆機能説明}
 make は,複数のファイルからなるプログラムのコンパイルを支援する.実行すると,同一フォルダ内のmakefile か Makefile というファイルの生成とプログラムの依存関係を記したファイル(両方ある場合は最初に見つかったほう)を読み込み,最初にあるターゲットを読み込む.ターゲットを指定した場合,そのターゲットの内容を実行する.Makefile の内容を簡略的に以下に示す.\\

\begin{center}
\begin{breakbox}
\begin{alltt}
 変数の設定
 (まず, Makefile 内で使用する変数を定義する)

 ターゲット1 : ファイル,ファイル...
   コマンド
   コマンド
 (ターゲット1の依存関係を記述)

 ターゲット2 : ファイル,ファイル...
   コマンド
   コマンド
 (ターゲット2の依存関係を記述)
\end{alltt}
\end{breakbox}
\end{center}

Makefile は,このような書式になっている. make がターゲットなしで実行された場合は,最初のターゲットであるターゲット1の内容に従って make が実行される.もしコマンド引数のターゲットにターゲット2を指定すると,ターゲット2の内容に従って make が実行される.\par

{\bf ◆使用例}

\begin{center}
\begin{breakbox}
\begin{alltt}
 % \underline{make}  \keybox{Enter}
\end{alltt}
\end{breakbox}
\end{center}
\clearpage
%% %% %%
%% %% %% %%