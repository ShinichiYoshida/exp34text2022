%% %% %% %%
%% %% %%
%% %% %%  ifconfig
%% %% %%
%% %% %%
\label{cmd:ifconfig}
\section{ip, ifconfig(InterFace Configration), ipconfig}
ネットワークインタフェースのパラメータ設定及び確認を行う\par

{\bf ◆書式}

\begin{center}
\begin{screen}
\begin{alltt}
 インタフェース(IF)にパラメータを振る
 % ifconfig IF名 [inet IPアドレス] [netmask ネットマスク] [broadcast ブロードキャストアドレス]
 インタフェースのパラメータを表示する
 % ifconfig -a
 % ip address (ip a と省略可)
 インタフェース IF の有効化
 % ifconfig IF名 up
 インタフェース IF の無効化
 % ifconfig IF名 down
 インタフェース IF に2つ目の IP アドレスの付与
 % ifconfig IF名 alias IP アドレス netmask ネットマスク
 インタフェース IF の2つ目の IP アドレスの削除
 % ifconfig IF名 -alias IP アドレス netmask ネットマスク
\end{alltt}
\end{screen}
\end{center}

broadcast は省略することができる.省略した場合は,ホスト部のビットが全て
1 のアドレスがブロードキャストアドレスとして使われる(All 1 broadcast).

{\bf ◆機能説明}

 ifconfig はネットワークインタフェースにパラメータ( IP アドレスやネットマスク,ブロードキャストアドレス)を割り振ったり,ネットワークインタフェースに割り振られたパラメータを見たりする場合に利用する.ネットワークインタフェースの設定は,スーパーユーザのみしか行えないので注意が必要である.\par

{\bf ◆使用例( 1 )ネットワークインタフェースにパラメータを割り振る}
\begin{center}
\begin{breakbox}
\begin{alltt}
 % \underline{ifconfig bge0 inet 192.168.0.151 netmask 255.255.255.0 up}  \keybox{Enter}
 % ここでは,IPアドレス割り当てと同時にインターフェースの有効化を行って
 % いる.
\end{alltt}
\end{breakbox}
\end{center}

これにより,ネットワークインタフェース fxp0 にIP アドレス 172.21.20.15 , 24 ビットのネットマスクが割り振られ,172.21.20.191にブロードキャストするよう設定された.\par

{\bf ◆使用例( 2 )ネットワークインタフェースのパラメータを参照する}
\begin{center}
\begin{breakbox}
\begin{alltt}
 % \underline{ifconfig -a}  \keybox{Enter}
 lo0: flags=849<UP,LOOPBACK,RUNNING,MULTICAST> mtu 8232
         inet 127.0.0.1 netmask ff000000 
 hme0: flags=863<UP,BROADCAST,NOTRAILERS,RUNNING,MULTICAST> mtu 1500
         inet 172.21.30.10 netmask ffffff00 broadcast 172.21.30.255

\end{alltt}
\end{breakbox}
\end{center}
ネットワークインタフェースに関する情報が表示された.\par

windows では ipconfig となっている.
\clearpage
%% %% %%
%% %% %% %%