%% %% %% %%
%% %% %%
%% %% %%  chown
%% %% %%
%% %% %%

\section{chown(CHange OWNer)}
指定したファイルの所有者およびグループを変更する\par
\label{cmd:chown}
\noindent
{\bf ◆書式}
\begin{center}
\begin{screen}
\begin{alltt}
# chown [-R] ユーザー名[:グループ名] ファイル名

    R:ディレクトリに対して変更内容を再帰的に処理する
\end{alltt}
\end{screen}
\end{center}

\noindent
{\bf ◆機能説明}

chown は引数にユーザとファイル名を指定してファイルの所有者を
変更する.このコマンドはスーパーユーザしか実行できない.

\noindent
{\bf ◆使用例}
\begin{center}
\begin{breakbox}
\begin{alltt}
% \underline{ls -l file}  (←fileの詳細な情報を表示する)
-rw-r--r--   1 user1   group1  2058 Sep 18 15:33 file

# \underline{chown user2 file}  (←fileの持ち主をuser2に変更する)
% \underline{ls -l file}
-rw-r--r--   1 user2   group1  2058 Sep 18 15:33 file
%
\end{alltt}
\end{breakbox}
\end{center}
\clearpage
%% %% %%
%% %% %% %%