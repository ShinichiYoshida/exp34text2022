%% %% %% %%
%% %% %%
%% %% %%  grep
%% %% %%
%% %% %%

\section{grep(Global Regular Expression Print)}
パターンにマッチする行を表示する\par
\label{cmd:grep}
\noindent
{\bf ◆書式}
\begin{center}
\begin{screen}
\begin{alltt}
% grep 文字列 [ファイル名]
\end{alltt}
\end{screen}
\end{center}

\noindent
{\bf ◆機能説明}

ファイルの中身から引数に指定された文字列を含む行を表示する.文字列の指定には正規表現を用いる事ができる.\par
また,ファイルを指定しない場合は標準入力から読み込むので,下の例のように出力が多い処理の出力を grep に向ける事により,必要な情報のみを抜き出して表示する事ができる.

\noindent
{\bf ◆使用例}
\begin{center}
\begin{breakbox}
\begin{alltt}
% \underline{ls /dev | grep console}   (←/devの中からconsoleという文字列があれば表示する)
console
%
\end{alltt}
\end{breakbox}
\end{center}
\clearpage
%% %% %%
%% %% %% %%
