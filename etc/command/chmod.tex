%% %% %% %%
%% %% %%
%% %% %%  chmod
%% %% %%
%% %% %%

\section{chmod(CHange MODe)}
ファイルのパーミッションを変更する\par
\label{cmd:chmod}
\noindent
{\bf ◆書式}
\begin{center}
\begin{screen}
\begin{alltt}
% chmod [-R] パーミッション ファイル名

    R:ディレクトリに対して変更内容を再帰的に処理する

% \underline{ls -l file}  (←fileの詳細な情報を表示する)
-rwxr-xr-x   1 user1   group2  2058 Sep 18 15:33 file

rwxr-xr-x で,文字があるところを1,ないところを0とすると
111101101 → これを3ビットずつ8進数にすると 755
(詳細は下記)

これを使って chmod 755 ファイル名などとする

\end{alltt}
\end{screen}
\end{center}

\noindent
{\bf ◆機能説明}

ファイルやディレクトリのパーミッションを変更する.
引数として,パーミッションと変更したいファイルを指定する.
パーミッションの指定には,数字で指定する方法と文字列で
指定する方法の 2 通りある.\par
文字表記において,r は読み込み,w は書き込み, x は実行の権利を示す.\par
数字で指定する場合,引数のパーミッション部にパーミッションに対応した数列を指定して変更を行う.\par
パーミッションの指定は,3 桁の数列で指定をする.その3桁の数列は,左から作成ユーザ,グループ,その他のユーザに対してのパーミッションを示しており,読み込みを許可する場合は 4 ,書き込みを許可する場合は 2 ,実行を許可する場合は 1 の値をプラスする.\par
下の例のモード 754 は,
\begin{itemize}
\item 左の桁が 7 なので,作成ユーザに対して読み込み(4)と書き込み(2)と実行(1)を許可する(4+2+1=7).
\item 中の桁が 5 なので,所属グループのメンバに対して読み込み(4)と実行(1)を許可する(4+1=5).
\item 右の桁が 4 なので,その他のユーザに対して読み込み(4)を許可する(4=4).
\end{itemize}

\noindent
{\bf ◆使用例}
\begin{center}
\begin{breakbox}
\begin{alltt}
% \underline{ls -l file}  (←fileの詳細な情報を表示する)
-rwxr-xr-x   1 user1   group2  2058 Sep 18 15:33 file

% \underline{chmod 754 file}  (←fileのモードを754に変更)

% \underline{ls -l file}  (←fileの詳細な情報を表示する)
-rwxr-xr--   1 user1   group2  2058 Sep 18 15:33 file

% \underline{ls -l file}  (←fileの詳細な情報を表示する)
-rwxr-xr-x   1 user1   group2  2058 Sep 18 15:33 file

% \underline{chmod o-r file}  (←その他のユーザの読み込みを不許可にするにはo-rとする)

% \underline{ls -l file}  (←fileの詳細な情報を表示する)
-rwxr-x--x   1 user1   group2  2058 Sep 18 15:33 file
%
\end{alltt}
\end{breakbox}
\end{center}
\clearpage
%% %% %%
%% %% %% %%