%% %% %% %%
%% %% %%
%% %% %%  ping
%% %% %%
%% %% %%

\section{ping}
パケットをネットワーク上のホストへ送る\par
\label{cmd:ping}
\noindent
{\bf ◆書式}
\begin{center}
\begin{screen}
\begin{alltt}
% ping [ホスト名 or IPアドレス]
\end{alltt}
\end{screen}
\end{center}

\noindent
{\bf ◆機能説明}

特別なパケットを指定したホストとやり取りする事によって,そのホストがネッ
トワークにつながっているかどうか調べる.\par pingによってつながっている
事が確認できた事を「 ping が通った」と言う場合が多い.\par ネットワーク
がつながっているかどうか調べる時に ping はとても便利なコマンドであるが,
昨今セキュリティーの問題上 ping を返さないネットワークも出て来ているので 
ping が通らないからと言ってネットワーク上つながっていないと一概に言えな
い場合もある.しかし,それは特殊な例であり,ping が通らない場合はつながっ
ていないと判断して構わない.

ping での導通確認は,ネットワークの OSI レイヤーでの第3層が正しく動作し
ていることを示す.

\noindent
{\bf ◆使用例〜 Linux で実行した場合}
\begin{center}
\begin{breakbox}
\begin{alltt}
% \underline{ping 192.168.1.2}  (←IPアドレスが192.168.1.2のマシンにpingをおくる)
PING 192.168.1.2 (192.168.1.2): 56 data bytes
64 bytes from 192.168.1.2: icmp_seq=0 ttl=254 time=1.047 ms
64 bytes from 192.168.1.2: icmp_seq=1 ttl=254 time=0.978 ms
64 bytes from 192.168.1.2: icmp_seq=2 ttl=254 time=18.238 ms
64 bytes from 192.168.1.2: icmp_seq=3 ttl=254 time=0.973 ms
64 bytes from 192.168.1.2: icmp_seq=4 ttl=254 time=0.983 ms
^C (←\keybox{Ctrl}+\keybox{c}で中断)
--- 192.168.1.2 ping statistics ---
5 packets transmitted, 5 packets received, 0% packet loss
round-trip min/avg/max/stddev = 0.973/4.444/18.238/6.897 ms   (←ping が通った)
% \underline{ping 192.168.1.1}  (←IPアドレスが192.168.1.1のマシンにpingをおくる)
PING 192.168.1.1 (192.168.1.1): 56 data bytes
^C (←\keybox{Ctrl}+\keybox{c}で中断)
--- 192.168.1.1 ping statistics ---
16 packets transmitted, 0 packets received, 100% packet loss   (←ping が通らなかった)
%
\end{alltt}
\end{breakbox}
\end{center}
\clearpage
%% %% %%
%% %% %% %%