%% %% %% %%
%% %% %%
%% %% %%  df / du
%% %% %%
%% %% %%

\section{df(Disk Free)/ du(Disk Usage)}
ディスクの状態を表示する\par
\label{cmd:dfdu}
\noindent
{\bf ◆書式}
\begin{center}
\begin{screen}
\begin{alltt}
% df [-k] ファイルシステム
% du [-k] ディレクトリ名

    k:kbytes単位で表示
\end{alltt}
\end{screen}
\end{center}

\noindent
{\bf ◆機能説明}

df は,指定したファイルシステムの使用可能な容量を表示する.また,全体の容量と使用済み容量も表示してくれる.\par
du は,あるディレクリ中の全ファイルが使用している領域を計算し,ファイルのなかにディレクトリがあれば,再帰的にその中の領域を計算する.\par

\noindent
{\bf ◆使用例}
\begin{center}
\begin{breakbox}
\begin{alltt}
% \underline{df -k}  (←各パーティションの容量の情報を1kバイト単位で表示する)
Filesystem            kbytes    used   avail capacity  Mounted on
/dev/dsk/c0t0d0s0      96975   17785   69493    21%    /
/dev/dsk/c0t0d0s6     770943  619640   97337    87%    /usr
/proc                      0       0       0     0%    /proc
fd                         0       0       0     0%    /dev/fd
/dev/dsk/c0t0d0s5      96975   21743   65535    25%    /var
/dev/dsk/c0t0d0s7  14332795 3420514 10768954    25%    /export/home
/dev/dsk/c0t0d0s4    1018191  143243  813857    15%    /opt
swap                 1091416      64 1091352     1%    /tmp
% \underline{du}  (←カレンントディレクトリにあるファイルの容量をみる)
488     ./.netscape
1       ./nsmail
1       ./Mail/inbox
1       ./Mail/draft
1       ./Mail/trash
4       ./Mail
2       ./.ssh
1493    .
%
\end{alltt}
\end{breakbox}
\end{center}
\clearpage
%% %% %%
%% %% %% %%