%% %% %% %%
%% %% %%
%% %% %%  ps
%% %% %%
%% %% %%

\section{ps(Process Status)}
プロセスの状態の表示\\
\label{cmd:ps}
\noindent
{\bf ◆書式}
\begin{center}
\begin{screen}
\begin{alltt}
(Solaris)/usr/bin/ps [-uAef]
  u : 指定するユーザ ID を持つ全てのプロセスを表示する
  A : すべてのプロセスを表示する
  e : 現在実行中のすべてのプロセスを表示する
  f : 完全形式で表示する
(FreeBSD, Linux)  ps [uaxef]
  u : ユーザ形式で表示する
  a : プロセスグループリーダを除く全プロセスを表示する
  x : 制御端末を持たないプロセスを含めて表示する
  e : 全てのプロセスを表示する
  f : 完全形式で表示する
  w : 1行に入らないものも続けて表示.
  ww: さらに長いものも続けて\textbf{全て}表示.
\end{alltt}
\end{screen}
\end{center}

\noindent
{\bf ◆機能説明}

ps はシステムで動作しているプロセスの情報をプロセス ID 順に表示する.
表示される情報はデフォルト(何もオプションをつけない)で,
プロセス ID,制御端末名,プロセス状態,cpu 時間などを表示する.
ps はシステムより速く実行できず,
他のプロセスと同様にスケジュールされて実行されるので,
表示される情報は正確ではない.オプションをつける事により,
より多くの情報を表示する.

ps コマンドはサーバでサービスがきちんと動いているか確認したり,
プロセスのゾンビが残っていたりしないか確認するときに使い プロセスを指定する kill のようなコマンドと一緒
に使うこともある.

\noindent
{\bf ◆使用例}
%% \begin{cli}
\begin{center}
\begin{breakbox}
\begin{alltt}
freebsdx# \underline{ps auxww} (←全プロセスの表示)
USER    PID %CPU %MEM   VSZ   RSS  TT  STAT STARTED      TIME COMMAND
root     11 100.0  0.0     0     8  ??  RL   12:37PM 1263:15.43 [idle]
root      0  0.0  0.0     0    48  ??  DLs  12:37PM   0:00.08 [kernel]
root    132  0.0  0.0  1536   848  ??  Is   12:37PM   0:00.00 adjkerntz -i
root    449  0.0  0.0  1888   540  ??  Is   12:37PM   0:00.00 /sbin/devd
root    556  0.0  0.1  3344  1308  ??  Ss   12:37PM   0:00.11 /usr/sbin/syslogd -s
root    782  0.0  0.2  6676  3596  ??  Is   12:37PM   0:00.00 /usr/sbin/sshd
root    797  0.0  0.2  6072  3448  ??  Ss   12:37PM   0:00.90 sendmail: accepting conn
smmsp   801  0.0  0.2  6072  3388  ??  Is   12:37PM   0:00.02 sendmail: Queue runner@0
root    807  0.0  0.1  3372  1352  ??  Is   12:37PM   0:00.16 /usr/sbin/cron -s
root    865  0.0  0.2  9400  4416  ??  Is   12:37PM   0:03.36 sshd: root@pts/0 (sshd)
root   3217  0.0  0.2  9400  4428  ??  Ss    7:58AM   0:00.07 sshd: root@pts/1 (sshd)
root    857  0.0  0.1  3344  1160  v0  Is+  12:37PM   0:00.00 /usr/libexec/getty Pc tt
root    858  0.0  0.1  3344  1160  v1  Is+  12:37PM   0:00.00 /usr/libexec/getty Pc tt
root   3220  0.0  0.1  4664  2588   1  Rs    7:58AM   0:00.04 -csh (csh)
root   3431  0.0  0.1  3424  1144   1  R+    9:41AM   0:00.00 ps aux
\end{alltt}
\end{breakbox}
\end{center}
%% \end{cli}
\clearpage
%% %% %%
%% %% %% %%
