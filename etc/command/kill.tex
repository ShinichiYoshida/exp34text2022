%% %% %% %%
%% %% %%
%% %% %%  kill
%% %% %%
%% %% %%

\section{kill}
プロセスを終了させたりシグナルを送ったりする\par
\label{cmd:kill}
\noindent
{\bf ◆書式}
\begin{center}
\begin{screen}
\begin{alltt}
% kill プロセス ID
% kill [signal] プロセス ID
\end{alltt}
\end{screen}
\end{center}

\noindent
{\bf ◆機能説明}

kill コマンドは,指定されたプロセス ID のプロセスに対し
シグナルを送るコマンドである.シグナル名がついていない時は,
指定されたプロセスの実行を停止させる.\par
自分のプロセスにはシグナルを送信する事ができるが,他人のプロセスに対してはシグナルを送信する事ができない.しかし,スーパーユーザだけは,他人のプロセスに対してもシグナルを送る事ができる.\par
使えるシグナル名は -l オプションでわかる.よく使われるシグナル名として HUP(Hang UP)がある.HUP シグナルは,
指定されたプロセスを一度停止させ再開する信号である.例えば,inetd.conf を変更した場合,inetd プロセスに設定を
改めて読み込ませるために以下のように HUP シグナルを送る.

\noindent
{\bf ◆使用例}
\begin{center}
\begin{breakbox}
\begin{alltt}
% \underline{kill -l}  (←使えるシグナル名を探す)
HUP INT QUIT ILL TRAP ABRT EMT FPE KILL BUS SEGV SYS PIPE ALRM TERM USR1 USR2 
CHLD PWR WINCH URG IO STOP TSTP CONT TTIN TTOU VTALRM PROF XCPU XFSZ

# \underline{ps aux}
freebsdx# ps aux
freebsdx# ps aux
USER    PID %CPU %MEM   VSZ   RSS  TT  STAT STARTED      TIME COMMAND
root    132  0.0  0.0  1536   848  ??  Is   12:37PM   0:00.00 adjkerntz -i
root    449  0.0  0.0  1888   540  ??  Is   12:37PM   0:00.00 /sbin/devd
root    556  0.0  0.1  3344  1308  ??  Is   12:37PM   0:00.01 /usr/sbin/syslogd -s
root    782  0.0  0.2  6676  3596  ??  Is   12:37PM   0:00.00 /usr/sbin/sshd
root    797  0.0  0.2  6072  3340  ??  Ss   12:37PM   0:00.02 sendmail: accepting conn
smmsp   801  0.0  0.2  6072  3388  ??  Is   12:37PM   0:00.00 sendmail: Queue runner@0
root    807  0.0  0.1  3372  1352  ??  Is   12:37PM   0:00.01 /usr/sbin/cron -s
root    865  0.0  0.2  9400  4372  ??  Ss   12:37PM   0:00.06 sshd: root@pts/0 (sshd)
root    874  0.0  0.1  3376  1412  ??  Ss   12:38PM   0:00.01 rpcbind
root    877  0.0  0.1  3344  1484  ??  Is   12:38PM   0:00.00 mountd
root    879  0.0  0.1  3284  1380  ??  Ss   12:38PM   0:00.02 nfsd: master (nfsd)
root    857  0.0  0.1  3344  1160  v0  Is+  12:37PM   0:00.00 /usr/libexec/getty Pc tt
root    868  0.0  0.1  4664  2460   0  Rs   12:37PM   0:00.02 -csh (csh)
root    936  0.0  0.1  3424  1144   0  R+    1:07PM   0:00.00 ps aux
# \underline{kill -KILL 807}(←プロセス ID 807番を再起動させた)
(kill -9 807 でも良い.9はKILLシグナルのシグナル番号
KILLは強制終了,TERMは終了(中断),HUPは(再起動)である)
#
\end{alltt}
\end{breakbox}
\end{center}
\clearpage
%% %% %%
%% %% %% %%