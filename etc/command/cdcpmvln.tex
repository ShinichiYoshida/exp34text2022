%% %% %% %%
%% %% %%
%% %% %%  c
%% %% %%
%% %% %%

\section{mv, cp, cd, ln}
mv, cp, cd, ln は,それぞれ,ファイルの移動 (MoVe), ファイルのコピー
(CoPy),カレントディレクトリ(今いるディレクトリ)の移動 (ChangeDirectory),
リンク(あるファイルを別のファイル名で参照する) LiNk に用いる.

\label{cmd:cp}
\noindent
{\bf ◆書式}
\begin{center}
\begin{screen}
\begin{alltt}
% mv 古いファイル名 新しいファイル名
% cp 元ファイル名 コピーのファイル名
% cd 移動先ディレクトリ
% ln -s 元ファイル名 別名
\end{alltt}
\end{screen}
\end{center}

\noindent
{\bf ◆機能説明}

chgrp は,引数にグループ名とファイルを指定してファイルのグループを
変更する.このコマンドを実行するユーザーは,変更後のグループに
属していないといけない.

\noindent
{\bf ◆使用例}
\begin{center}
\begin{breakbox}
\begin{alltt}
% \underline{ls -l file}  (←fileの詳細な情報を表示する)
-rw-r--r--   1 user1   group1  2058 Sep 18 15:33 file

% \underline{chgrp  group2 file}  (←fileの所属するグループをgroup2に変更する)
% \underline{ls -l file}
-rw-r--r--   1 user1   group2  2058 Sep 18 15:33 file
%
\end{alltt}
\end{breakbox}
\end{center}
\clearpage
%% %% %%
%% %% %% %%