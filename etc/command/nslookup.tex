%% %% %% %%
%% %% %%
%% %% %%  nslookup
%% %% %%
%% %% %%
\section{nslookup}
ネームサーバに対話的に問い合わせを行う\\
\label{cmd:nslookup}
\noindent
{\bf ◆書式}
\begin{center}
\begin{screen}
\begin{alltt}
% nslookup
\end{alltt}
\end{screen}
\end{center}

\noindent
{\bf ◆機能説明}

nslookup は IP アドレスとホスト名の対応を調べる.
nslookup を実行して,その後に調べたいホスト名あるいは 
IP アドレスを入力する.終わるときは exit と入力する.
設定した DNS サーバで正しくホスト名と %
IP アドレスの変換が行われているか調べるときに使う.

\noindent
{\bf ◆使用例}
\begin{center}
\begin{breakbox}
\begin{alltt}
% \underline{nslookup}
Default Server:  dns.xxx.yyy.zzz
Address:  192.168.1.3

> \underline{192.168.1.1}  (←IPアドレス192.167.1.1のホスト名を調べる)
Server:  dns.xxx.yyy.zzz
Address:  192.168.1.3

Name:    machine1.xxx.yyy.zzz (←ホスト名の情報)
Address:  192.168.1.1

> \underline{machine1.xxx.yyy.zzz}  (←ホスト名machine1のIPアドレスを調べる)
Server:  dns.xxx.yyy.zzz
Address:  192.168.1.3

Name:    machine1.xxx.yyy.zzz
Address:  192.168.1.1

> \underline{exit}   (← nslookup を終了する)
%
\end{alltt}
\end{breakbox}
\end{center}
\clearpage
%% %% %%
%% %% %% %%