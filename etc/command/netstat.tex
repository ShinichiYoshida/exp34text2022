%% %% %% %%
%% %% %%
%% %% %%  netstat
%% %% %%
%% %% %%
\label{cmd:netstat}
\section{netstat(NETSTATus)}
ネットワークの状態を表示する\par

{\bf ◆書式}
\begin{center}
\begin{screen}
\begin{alltt}
 % netstat [-rn]
   r : ルーティングテーブルを表示する
   n : ネットワークアドレスの表示を数字で行う.
\end{alltt}
\end{screen}
\end{center}

{\bf ◆機能説明}

 netstat は,ネットワークに関連したさまざまな情報を表示するプログラムである.ここで説明するのはネットワークインタフェースのパケットトラフィックに関するルーティングテーブルをひょうじする r オプションに関してのみであるが,他にもアクティブソケットの一覧や他のネットワークの状態等オプションによって用途に適した情報が得られるので,他の機能を利用したい場合はマニュアルを読んで欲しい.\par
 また,通常 netstat はできる限り IP アドレスをホスト名で表示しようとするが, n オプションをつけることによりアドレスを数字で表示する.\par

{\bf ◆使用例}
\begin{center}
\begin{breakbox}
\begin{alltt}
 % \underline{netstat -r}  \keybox{Enter}
 Routing Table:
   Destination           Gateway           Flags  Ref   Use   Interface
 -------------------- -------------------- ----- ----- ------ ---------
 172.21.54.0          test                  U        3  47873  hme0
 BASE-ADDRESS.NET     test                  U        3      0  hme0
 default              dss.kochi-tech.ac.jp  UG       0 208343  
 localhost            localhost             UH       02753237  lo0
 %
\end{alltt}
\end{breakbox}
\end{center}
\clearpage
%% %% %%
%% %% %% %%