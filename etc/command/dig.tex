%% %% %% %%
%% %% %%
%% %% %%  dig
%% %% %%
%% %% %%
\section{dig}
DNSサーバの挙動を表示する\\
\label{cmd:dig}
\noindent
{\bf ◆書式}
\begin{center}
\begin{screen}
\begin{alltt}
\% dig  @[DNSサーバ] [確認したいドメイン名] [レコード]
\end{alltt}
\end{screen}
\end{center}

\noindent
{\bf ◆機能説明}

DNSサーバに対して問い合わせを行い応答結果をセクション毎に表示する.
3番目引数にレコードを追加することでそのタイプの検索結果を表示し,ANYとすると
全てのレコードの検索結果を表示する.レコードを指定しない場合はA(ネットワークアドレス)についての情報となる.
単にdigとするとルートサーバの表示をする.
nslookupと異なりDNSサーバの応答パケットをほぼそのまま表示する.

\noindent
{\bf ◆使用例(1)〜DNSサーバに指定したドメイン名の応答を全て表示する}
\begin{center}
\begin{breakbox}
\begin{alltt}
\% \underline{dig @server.gX.info.kochi-tech.ac.jp gX.info.kochi-tech.ac.jp ANY}  (←DNSサーバserver.gX〜におけるgX.info〜の応答を全て表示する)

; <<>> DiG 9.3.4-P1 <<>> @server.gX.info.kochi-tech.ac.jp gX.info.
kochi-tech.ac.jp ANY 
; (1 server found)
;; global options:  printcmd
;; Got answer:
;; ->>HEADER<<- opcode: QUERY, status: NOERROR, id: 24369
;; flags: qr aa rd ra; QUERY: 1, ANSWER: 3, AUTHORITY: 0, ADDITIONAL: 1

;; QUESTION SECTION:
;gX.info.kochi-tech.ac.jp. IN      ANY

;; ANSWER SECTION:
gX.info.kochi-tech.ac.jp. 3600 IN  SOA  server.gX.info.kochi-tech.ac.jp.
 postmaster.gX.info.kochi-tech.ac.jp. 2002081601 10800 3600 604800 604800
gX.info.kochi-tech.ac.jp. 3600 IN  NS   server.gX.info.kochi-tech.ac.jp.
gX.info.kochi-tech.ac.jp. 3600 IN  MX   10 server.gX.info.kochi-tech.ac.jp.

;; ADDITIONAL SECTION:
server.gX.info.kochi-tech.ac.jp. 3600 IN A   172.21.1X.2

;; Query time: 1 msec
;; SERVER: 172.21.1X.2\#53(172.21.1X.2)
;; WHEN: Mon Mar  2 21:18:02 2009
;; MSG SIZE  rcvd: 145

\end{alltt}
\end{breakbox}
\end{center}
{\bf ◆digコマンド実行時の表示内容}
\begin{itemize}
\item {\bf ヘッダ}\\
 ヘッダ部分は1行目にはdigコマンドのバージョンとコマンドの内容,2行目には該当したサーバ数,3行目にはグローバルオプションの表示,4行目には応答を受けたことを表示する.5行目には応答パケットのヘッダ情報を表示し,正しく情報を得られた場合はNOERROR,ドメイン名が存在しない場合は NXDOMAIN とstatusの後に表示する.6行目 flagsでこの応答がどのようなものか知ることができる.aaは権威のある回答であることを示しており,指定したドメイン名の情報を持ったDNSサーバからの応答だということを示している. QUERYには各セクションにいくつ該当するものがあったか表示する.  

\item {\bf QUESTION SECTION}\\
 表示させる内容.[ドメイン名] [クラス] [レコード] の順に並んでいる.例の場合は指定したDNSサーバの持つgX.a360.info.kochi-tech.ac.jpの全ての情報を表示する.

\item {\bf ANSWER SECTION}\\
 DNSサーバの応答内容. [ドメイン名] [リフレッシュ間隔] [クラス] [レコード] [データ]の順に並んでいる.例の場合は要求をANYで行っているので,SOA,NS,MXの3つが返って来ている.

\item {\bf ADDITIONAL SECTION}\\
 追加としてDNSサーバの情報を表示する.

\item {\bf フッタ}\\
 フッタには,検索にかかった時間,使用したDNSサーバ,検索した日時,受け取ったデータサイズなどを表示する.
\end{itemize}
%% 正引き %%
\noindent
{\bf ◆使用例(2)〜正引き(ドメイン名からIPアドレスを調べる)}
\begin{center}
\begin{breakbox}
\begin{alltt}
\% \underline{dig machine.gX.info.kochi-tech.ac.jp} (←mashine.gX〜のドメイン名のIPアドレスを調べる)

…

;; ANSWER SECTION:
machine.gX.info.kochi-tech.ac.jp. 3600 IN A  172.21.1X.4

…

\end{alltt}
\end{breakbox}
\end{center}

%% 逆引き %%
\noindent
{\bf ◆使用例(3)〜逆引き(IPアドレスからドメイン名を調べる)}
\begin{center}
\begin{breakbox}
\begin{alltt}
\% \underline{dig @server.gX.info.kochi-tech.ac.jp 4.1X.21.172.in-addr.arpa. PTR}  (←IPアドレス172.21.1X.4のドメイン名を調べる)

…

;; ANSWER SECTION:
4.1X.21.172.in-addr.arpa. 3600  IN      PTR     server.gX.info.kochi-tech.ac.jp.

…

\end{alltt}
\end{breakbox}
\end{center}
上記のようにして,逆引きのレコードであるPTRを呼び出すことでIPアドレスからドメイン名を調べることができるが,-xというオプションを用いて
\begin{center}
\begin{breakbox}
\begin{alltt}
\% \underline{dig -x 172.21.1X.4}
\end{alltt}
\end{breakbox}
\end{center}
とすることでも同様の結果を得ることができる.
\clearpage
