%% %% %% %%
%% %% %%
%% %% %%  man
%% %% %%
%% %% %%

\section{man}
オンラインマニュアルのフォーマット、表示を行なう\\
\label{cmd:man}
\noindent
{\bf ◆書式}
\begin{center}
\begin{screen}
\begin{alltt}
% man コマンド名
\end{alltt}
\end{screen}
\end{center}

\noindent
{\bf ◆機能説明}

引数としてコマンド名を指定して,コマンドのオンラインマニュアルを
表示する.\par
manコマンドは,/usr/share/man に格納されているマニュアルを参照する。\par
\noindent
{\bf ◆使用例}
\begin{center}
\begin{breakbox}
\begin{alltt}
% \underline{man man}

Reformatting page.  Wait... done

User Commands                                              man(1)

NAME
     man - find and display reference manual pages

SYNOPSIS
     man [ - ] [ -adFlrt ] [ -M path ] [ -T macro-package ]
          [-s section ] name ...
     man [ -M path ] -k keyword ...
     man [ -M path ] -f file ...

DESCRIPTION
     The man command  displays  information  from  the  reference
     manuals.   It displays complete manual pages that you select
     by name, or one-line summaries selected  either  by  keyword
     (-k),  or  by  the  name  of an associated file (-f).  If no
     manual page is located, man prints an error message.

  Source Format
     Reference Manual pages are marked up with either nroff(1) or
     sgml(5)  (Standard  Generalized  Markup Language) tags.  The
--More--(5%)
%
\end{alltt}
\end{breakbox}
\end{center}
\clearpage
%% %% %%
%% %% %% %%