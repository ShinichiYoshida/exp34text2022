%% %% %% %%
%% %% %%
%% %% %%  tail
%% %% %%
%% %% %%

\section{tail}
ファイルの最後の部分を表示する\par
\label{cmd:tail}
\noindent
{\bf ◆書式}
\begin{center}
\begin{screen}
\begin{alltt}
% tail ファイル名
\end{alltt}
\end{screen}
\end{center}

\noindent
{\bf ◆機能説明}

ファイルの最後の数行を表示する.
ログあるいはファイルなどで最後の方にあるデータだけを
調べたいときに便利.\par

\noindent
{\bf ◆使用例}
\begin{center}
\begin{breakbox}
\begin{alltt}
# \underline{tail setuid.today} (← setsid.today の最後の数行を表示する)
-r-sr-xr--  1 root  network  222240 Sep 17 08:46:54 1999 /usr/sbin/ppp
-r-sr-xr-x  1 root  wheel     85504 Sep 17 08:47:04 1999 /usr/sbin/pppd
-r-xr-sr-x  2 root  kmem      13184 Sep 17 07:48:20 1999 /usr/sbin/pstat
-r-sr-xr-x  5 root  wheel    290288 Sep 17 07:48:37 1999 /usr/sbin/purgestat
-r-sr-xr-x  5 root  wheel    290288 Sep 17 07:48:37 1999 /usr/sbin/sendmail
-r-sr-x---  1 root  network    9768 Sep 17 07:48:24 1999 /usr/sbin/sliplogin
-r-xr-sr-x  2 root  kmem      13184 Sep 17 07:48:20 1999 /usr/sbin/swapinfo
-r-sr-xr-x  1 root  wheel     13440 Sep 17 07:48:28 1999 /usr/sbin/timedc
-r-sr-xr-x  1 root  wheel     11232 Sep 17 07:48:29 1999 /usr/sbin/traceroute
-r-xr-sr-x  1 root  kmem       7036 Sep 17 07:48:29 1999 /usr/sbin/trpt
#
\end{alltt}
\end{breakbox}
\end{center}
\clearpage
%% %% %%
%% %% %% %%