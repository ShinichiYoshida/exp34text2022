%% %% %% %%
%% %% %%
%% %% %%  ip route
%% %% %%
%% %% %%
\section{show mac-address-table}
\label{cmd:ios-mac-address}
Cisco IOS上で,スイッチ(ブリッジ)の MAC アドレステーブルを確認する.

\noindent
{\bf ◆書式}
\begin{center}
\begin{screen}
\begin{alltt}
ルータの場合
Router#show mac-address-table

スイッチの場合
Switch#show mac address-table

\end{alltt}
\end{screen}
\end{center}


{\bf ◆機能説明}

どのポートの先に,どの MAC アドレスの端末がいるかが確認できる.

スイッチやルータで確認できるが,L2スイッチ機能のないルータでは実行できない(MAC アドレステーブルそのものがない).

{\bf ◆使用例}
\begin{center}
\begin{breakbox}
\begin{alltt}
Switch#show mac address-table
          Mac Address Table
-------------------------------------------

Vlan    Mac Address       Type        Ports
----    -----------       --------    -----
   1    109a.dd4f.0df5    DYNAMIC     Fa0/9
   1    e05f.b90d.47b5    DYNAMIC     Fa0/1

スイッチの9番ポートの先に,10:9A:DD:4F:0D:F5
の MAC アドレスを持つ端末が存在する.

スイッチの10番ポートの先に E0:5F:B9:0D:47:B5
の MAC アドレスを持つ端末が存在する.

\end{alltt}
\end{breakbox}
\end{center}
                

