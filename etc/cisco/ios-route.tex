%% %% %% %%
%% %% %%
%% %% %%  ip route
%% %% %%
%% %% %%
\section{ip route}
\label{cmd:ios-ip-routeping}
Cisco IOS上で,静的経路の設定を行う.

\noindent
{\bf ◆書式}
\begin{center}
\begin{screen}
\begin{alltt}
 (config)#ip route 宛先NWアドレス 宛先ネットマスク 次ルータアドレス (メトリック)
\end{alltt}
\end{screen}
\end{center}


{\bf ◆機能説明}

Cisco IOS 上で,ルーティングテーブルに静的に経路を追加する.管理者が手
動で削除するまで,消去されることはない.

{\bf ◆使用例    ルーティングテーブルの追加}
\begin{center}
\begin{breakbox}
\begin{alltt}
cisco(config)#ip route 192.168.0.0 255.255.255.0 172.21.10.101 2
                         ↑             ↑           ↑        ↑
                     NWアドレス      マスク     次ルータ      距離
\end{alltt}
\end{breakbox}
\end{center}
                

{\bf ◆削除}

削除する場合は,行頭に no を付けると削除される.

\begin{center}
\begin{screen}
\begin{alltt}
 (config)#no ip route 宛先NWアドレス 宛先ネットマスク 次ルータアドレス (メトリック)
\end{alltt}
\end{screen}
\end{center}

222.222.22.22 宛ての経路を削除する例
 
\begin{center}
\begin{screen}
\begin{alltt}
cisco#conf t
cisco(config)#no ip route 222.222.22.22 255.255.255.0 222.221.1.1
\end{alltt}
\end{screen}
\end{center}