%% %% %% %%
%% %% %%
%% %% %%  ip nat
%% %% %%
%% %% %%
\section{NATの設定: ip nat}
\label{cmd:ios-ip-nat}
Cisco IOS上で,NAT (NAPT, IP masquerading) の設定,および,NATテーブル確認を行う.

\noindent
{\bf ◆書式}
\begin{center}
\begin{screen}
\begin{alltt}
【静的 NAT の設定】
(ポートフォワーディング)
Router(config)#ip nat inside source static tcp 172.21.22.100 80 222.229.69.3 8080

内側(inside) の(送信元)IPアドレス 172.21.22.100, TCP ポート 80 を,
外側(outside, インターネット側) に対して,222.229.69.3 TCP ポート 8080 で
公開する.

【NATテーブルの確認】
Router#show ip nat translations
Pro Inside global         Inside local          Outside local         Outside global
tcp 222.229.69.3:8080     172.21.22.100:80      ---                   ---

【削除する場合】

NATテーブルから削除
Router#clear ip nat translations *

静的 NAT の設定削除
Router(config)#no ip nat inside source static tcp 172.21.22.100 80 222.229.69.3 8080
\end{alltt}
\end{screen}
\end{center}

\textbf{動的NAT (PAT, NAPT, IP Masquerading) の設定}

\begin{center}
\begin{screen}
\begin{alltt}

【変換元のアドレス】
(変換対象の変換前(内部PC)のアドレスを
 アクセスリストで設定)

Router(config)# access-list 1 permit 192.168.11.0 0.0.0.255
(192.168.11.0~192.168.11.255を変換する例)

【変換後のアドレス】
(変換後のパブリック(外部)IPアドレスを
 IPアドレスプールとして設定)

Router(config)# ip nat pool groupX 222.229.22.2 222.229.22.2 255.255.255.0

(222.229.22.2から222.229.22.2まで(1つのIPアドレスのみ
例)を使う)

【上記リストとプールを使ったNAPTの設定】

Router(config)# ip nat inside source list 1 pool groupX overload
(overloadを付けることで,NAPTを行う)
(1つのIPアドレスで複数PCが外部接続できる)
\end{alltt}
\end{screen}
\end{center}

