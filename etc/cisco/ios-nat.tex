%% %% %% %%
%% %% %%
%% %% %%  ip route
%% %% %%
%% %% %%
\section{show ip nat translations, ip nat}
\label{cmd:ios-show-int-status-nat}
Cisco IOS上で,NAT (NAPT, IP masquerading) の設定,および,NATテーブル確認を行う.

\noindent
{\bf ◆書式}
\begin{center}
\begin{screen}
\begin{alltt}
静的 NAT の設定
Router(config)#ip nat inside source static tcp 172.21.22.100 80 222.229.69.3 8080

内側(inside) の(送信元)IPアドレス 172.21.22.100, TCP ポート 80 を,
外側(outside, インターネット側) に対して,222.229.69.3 TCP ポート 8080 で
公開する.

NATテーブルの確認
Router#show ip nat translations
Pro Inside global         Inside local          Outside local         Outside global
tcp 222.229.69.3:8080     172.21.22.100:80      ---                   ---

静的 NAT の設定削除
Router(config)#no ip nat inside source static tcp 172.21.22.100 80 222.229.69.3 8080

NATテーブルから削除
Router#clear ip nat translations *

\end{alltt}
\end{screen}
\end{center}


{\bf ◆機能説明}

どのポートが,どの VLAN に所属しているかを確認する.

{\bf ◆使用例}
\begin{center}
\begin{breakbox}
\begin{alltt}
静的 NAT の設定
Router(config)#ip nat inside source static tcp 172.21.22.100 80 222.229.69.3 8080

内側(inside) の(送信元)IPアドレス 172.21.22.100, TCP ポート 80 を,
外側(outside, インターネット側) に対して,222.229.69.3 TCP ポート 8080 で
公開する.

すなわち外から,222.229.69.3 TCP ポート 8080 へのアクセスがあれば,
内側のサーバ 172.21.22.100, TCP ポート 80 へ転送
(ポートフォワーディング)する.

Router#show ip nat translations
Pro Inside global         Inside local          Outside local         Outside global
tcp 222.229.69.3:8080     172.21.22.100:80      ---                   ---

NAT 設定の確認
\end{alltt}
\end{breakbox}
\end{center}
                

