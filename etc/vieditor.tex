%
%       「viエディタの使い方」
%               written by Masahiro Fukumoto
%               modified by Takahiko Mendori
%

\newcommand{\ckeywidth}{1.5zw}
\newcommand{\ckey}[1]{
 \framebox[\ckeywidth][c]{\texttt{#1}\phantom{Jj}\hspace*{-.85zw}}
}
\newcommand{\Esckey}{
 \framebox[2zw][c]{\textsf{Esc}\phantom{Jj}\hspace*{-.85zw}}
}
\newcommand{\Enterkey}{
 \framebox[3zw][c]{\textsf{Enter}\phantom{Jj}\hspace*{-.85zw}}
}


UNIXには Emacs や mule などの他に vi と呼ばれるエディタがある.
vi は UNIX で標準提供されているエディタで,
すべての UNIXシステム上で利用することができる
(Emacs は豊富な機能を持ったエディタであるが,
通常は標準で提供されてはいないため,
使用するためにはインストールする必要がある).

\section{vi の起動}

vi はファイル名を引数に与えて起動する
(ファイル名は省略できる).

\texttt{\% vi} [\textit{filename}]

\section{vi のモード}
vi には,コマンドモードと入力モードがある.
vi を起動した直後のモードはコマンドモードであり,文字を入力できる
モードにはなっていない.
コマンドモードでは,カーソルの移動,ファイルの保存,テキストの削除,
複写などの操作を行う.
文字を入力するには入力モードに入る必要がある.
コマンドモードから入力モードへの切替えは,\ckey{i}などのキーで,
また入力モードからコマンドモードへの切替えは\Esckey で行う.

\section{コマンドモード}
コマンドモードでは,コマンドに先行して数引数を与えることができ,
ほとんどのコマンドは数引数をコマンドの繰り返し数と解釈する.
例えば,\ckey{5}\ckey{j}と入力すれば 
\ckey{j}\ckey{j}\ckey{j}\ckey{j}\ckey{j}
と入力することと同じになる.

\subsection{カーソルの移動}
上下左右のカーソルの移動には,コマンドモードでそれぞれ
\ckey{k}\ckey{j}\ckey{h}\ckey{l}のキーを使う.
その他,カーソル移動のコマンド(キー)を以下にまとめる.
\\

\begin{center}
\begin{tabular}{c|l||c|l}
\hline
{\bf コマンド} & {\bf 移動先} & {\bf コマンド} & {\bf 移動先} \\ \hline
\texttt{j} & 1行下 & \texttt{G} & 最終行 \\
\texttt{k} & 1行上 & \textit{n}\texttt{G} & \textit{n}行目 \\
\texttt{h} & 1文字左 & \texttt{H} & 画面の最上行 \\
\texttt{l} & 1文字右 & \texttt{M} & 画面の中央行 \\
\texttt{0} & 現在行の先頭 & \texttt{L} & 画面の最下行 \\
\texttt{\$} & 現在行の行末 & \texttt{b, B} & 前の単語の先頭 \\
$\mathbf{\hat{\phantom{a}}}$ & 現在行の先頭の単語の先頭
 & \texttt{w, W} & 次の単語の先頭 \\
\framebox[3zw][c]{\textsf{Enter}} & 1行下の先頭 & \texttt{e, E}
 & 現在もしくは次の単語の末尾 \\
\hline
\end{tabular}
\end{center}

\subsection{削除,複写,移動}
テキストの削除には次のコマンドを使う.

\begin{center}
\begin{tabular}{c|l}
\hline
{\bf コマンド} & {\bf 意味} \\ \hline
\texttt{x} & カーソル位置の文字を削除 \\
\texttt{X} & カーソル位置の直前の文字を削除 \\
\texttt{dd} & 現在行を削除 \\
\texttt{dw} & カーソル位置からその単語の最後までを削除 \\
\texttt{d\$} & カーソル位置から行の最後までを削除 \\
\texttt{d}$\mathbf{\hat{\phantom{a}}}$ & カーソル位置から行の先頭までを削除 \\
\texttt{J} & 現在行と次行を 1行に連結 \\
\hline
\end{tabular}
\end{center}

vi は直前の削除あるいは置換コマンドによって消されたテキストを
退避バッファに保存している.
また,指定した領域を退避バッファに複写することもできる.
テキストを退避バッファに保存した後でカーソルを移動し,
退避バッファの内容を現在変集中のテキストに挿入することで,
テキストの移動や複写ができる.
以下にコマンドを示す.
\\

\texttt{
\begin{center}
\begin{tabular}{c|l}
\hline
{\bf コマンド} & {\bf 意味} \\ \hline
P & 退避バッファの内容をカーソル位置に挿入 \\
p & 退避バッファの内容をカーソル位置の直後に挿入 \\
yy & 現在行を退避バッファに保存 \\
yw & カーソル位置から単語の終りまでを退避バッファに保存 \\
\hline
\end{tabular} 
\end{center}
}

\subsection{検索,置換}
文字列の検索は,\ckey{/}の後に検索したい文字列を入力して 
\Enterkey キーを押す.
検索を繰り返す場合には,\ckey{n}を入力する.
また,前にさかのぼって検索する場合には\ckey{/}のかわりに 
\ckey{?}を用いる.

文字列を置換する場合は,まず\ckey{/}で文字列を検索し,
ここで\ckey{c}\ckey{w}の後に変更する文字列を入力して
\Esckey キーを押す.

\subsection{コマンドの取り消しと再実行}
vi では,直前の編集コマンドのみ取り消すことができる.
\ckey{u}を入力すると直前に実行されたコマンドが取り消される.
また,\ckey{.}は直前のコマンドを再実行する.

\subsection{ファイルの読み書き}
ファイルの読み書きはコマンドモードで行う.
作成(修正)したテキストをファイルとして書き込むためには 
\ckey{:}の後に\ckey{w}を入力する\footnote{
ただし,UNIXではファイルにオーナー,グループ,
パーミッション(使用許可)という属性があるため,
``\texttt{Read-only file, not written; use ! to override.}''
などのメッセージが表示され,書き込めないことがある
(書き込み不許可のパーミッションのとき).
パーミッションを変更する場合には \texttt{chmod}コマンド
({\bf 付録\ref{etc:unixcmd}} 参照)を用いる.}.

ファイルの読み書きや vi の終了などを行う場合には,通常 
\ckey{:}を入力する.
\ckey{:}を入力することにより最下行に`` : ''が表示されるから,
ここにコマンドを入力する.
ファイル操作のコマンド\footnote{
これらのコマンドは入力後に\Enterkey キーを押す必要がある.}
を以下に示す.
\\

\begin{center}
\begin{tabular}{l|l}
\hline
{\bf コマンド} & {\bf 機能} \\ \hline
\texttt{:w} & 現在のファイル名で保存 \\
\texttt{:w} \textit{filename} & \textit{filename}に保存 \\
\texttt{:w!} & 強制書き込み \\
\texttt{:e} \textit{filename} & \textit{filename}を新たに読み込み \\
\texttt{:e!} & ファイルの再読み込み \\
\texttt{:r} \textit{filename}
 & \textit{filename}の内容をカーソル行の次行に挿入 \\
\hline
\end{tabular}
\end{center}

\subsection{vi の終了}
vi の終了の方法には,ファイルに保存後終了,内容が更新されて
いる場合には警告メッセージを表示する終了,強制終了の 3通りがある.
それぞれの終了コマンドは次の通りである.
\\

\begin{center}
\begin{tabular}{l|l}
\hline
{\bf コマンド} & {\bf 機能} \\ \hline
\texttt{ZZ} & 編集中の内容を保存して vi を終了 \\
\texttt{:wq} [\textit{filename}] \framebox[3zw][c]{\textsf{Enter}}
 & 編集中の内容を \textit{filename} に保存し vi を終了 \\
\texttt{:q} \framebox[3zw][c]{\textsf{Enter}} & vi を終了 \\
\texttt{:q!} \framebox[3zw][c]{\textsf{Enter}} & vi を強制終了 \\
\hline
\end{tabular}
\end{center}

\section{モードの切替え}
コマンドモードから入力モードには\ckey{i}キーを押すと切り替わる.
入力モードに切り替わると自由に文字列を入力することができる.
\ckey{i}キーでは,現在のカーソル位置で入力モードに切り替わる.
入力モードへの切替えコマンドを以下に示す.
\\

\begin{center}
\begin{tabular}{c|l}
\hline
{\bf コマンド} & {\bf 機能} \\ \hline
\texttt{a} & カーソルの直後で入力モード \\
\texttt{A} & 現在行の行末で入力モード \\
\texttt{i} & カーソル位置で入力モード \\
\texttt{I} & 現在行の最初の単語の先頭で入力モード \\
\texttt{o} & カーソル行の 1行下に空行を挿入し入力モード \\
\texttt{O} & カーソル行の 1行上に空行を挿入し入力モード \\
\texttt{R} & カーソル位置で上書き入力モード \\
\hline
\end{tabular}
\end{center}

入力モードからコマンドモードへの切替えは\Esckey キーにより行う.

