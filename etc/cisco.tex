\section{IOS の基本操作}

\subsection*{基本オペレーション}

 

後述するモードからコマンドを実行する.

? ヘルプ,タブ補完,省略形を用いて素早い操作ができる.

\subsubsection*{? ヘルプ}
? キーをタイプすることで,コマンド一覧と簡単な説明が表示される.

\subsubsection*{? ヘルプ(2)}
途中までキータイプしてから ? キーをタイプすると,そこで実行可能なコマン
ド一覧が表示される.

\begin{center}
\begin{screen}
\begin{alltt}
cisco#show ip r?
redirects  rip  route  rpf  rsvp
rtp
\end{alltt}
\end{screen}
\end{center}

\subsubsection*{タブ補完}

コマンドを途中までキータイプし,そのコマンドのみが補完可能なコマンドで
あれば,タブを入力することで補完される.

\begin{center}
\begin{screen}
\begin{alltt}
cisco#conf[TAB] term[TAB]
       ↓
cisco#configure terminal
\end{alltt}
\end{screen}
\end{center}

\subsubsection*{省略形}

コマンドを途中までキータイプし,そのコマンドのみが補完可能なコマンドで
あれば,補完することなく Enter をすれば,そのコマンドが実行できる.

\begin{center}
\begin{screen}
\begin{alltt}
cisco#en
       ↓
    enableが実行される

cisco#conf t
       ↓
   configure terminal が実行される
\end{alltt}
\end{screen}
\end{center}

\subsection*{モード}

Cisco IOS にはいくつかのモードがあり,それぞれのモードで必要な設定を行
うモーダルなシステムである.設定ファイルの概念はなく,設定はコマンドを
実行するごとに内部コンフィグファイル追加されていく.

各モードから,\textbf{前のモードに戻るには,``exit'' コマンドを実行する.}

\subsection*{一般ユーザモード}

ログインしてすぐに入るモード.ごく限られたコマンドしか使えない.

\begin{center}
\begin{screen}
\begin{alltt}
User Access Verification
Password: Kerberos:     No default realm defined for Kerberos!
Password:
cisco>
\end{alltt}
\end{screen}
\end{center}

\subsection*{特権モード}

一般ユーザモードから,``enable'' コマンドで特権モードに遷移する.プロン
プトが \# に変わり,多くのシステム閲覧コマンド ( show xxxx ... )が使えるよ
うになる.\textbf{設定の確認はできるが,設定は行えない}.

\begin{center}
\begin{screen}
\begin{alltt}
cisco>enable
Password:
cisco#
\end{alltt}
\end{screen}
\end{center}

\subsection*{コンフィグモード}

設定変更を行うモード.特権モードから `` configure terminal'' コマンドで
遷移する.ルーティングやアクセスコントロールリスト等,様々なシステム全
体に及ぶ設定変更コマンドが使える.\textbf{設定を行うことができるが,設定の確認は行えない.}exit にて特権モードに戻ってからせて値を確認するか,do コマンドを使うことでコンフィグコードで一時的に設定を確認できる(例:do show int).

\begin{center}
\begin{screen}
\begin{alltt}
cisco#configure terminal
Enter configuration commands, one per line.  End with CNTL/Z.
cisco(config)#
\end{alltt}
\end{screen}
\end{center}

\subsection*{インターフェイスモード}

各インターフェイスの設定変更を行うモード.コンフィグモードから
``interface インターフェイス名'' コマンドで遷移する.

IP アドレスやネットマスクの変更,アクセスコントロールリストのインターフェイスへの適用が行える.

\begin{center}
\begin{screen}
\begin{alltt}
cisco(config)#interface fastEthernet 0/0
cisco(config-if)#
\end{alltt}
\end{screen}
\end{center}

\subsection*{ルータモード}

ルーティングの設定変更を行うモード.コンフィグモードから
``router ルーティングプロトコル'' コマンドで遷移する.

\begin{center}
\begin{screen}
\begin{alltt}
cisco(config)#router rip
cisco(config-router)#
\end{alltt}
\end{screen}
\end{center}

\subsection*{インターフェイスの確認:show interface (sh int)}

\begin{center}
\begin{screen}
\begin{alltt}
cisco#show interfaces
FastEthernet0/0 is up, line protocol is up
  Hardware is AmdFE, address is 000f.2309.a620 (bia 000f.2309.a620)
  Internet address is 172.21.10.11/24
        :
\end{alltt}
\end{screen}
\end{center}

\subsection*{インターフェイスの確認2:show ip interface brief (sh ip int br)}

\textbf{インターフェースに線が接続されているか,有効か無効か,IP アドレスは何か}などの基本的な情報を得ることができる.

\begin{center}
\begin{screen}
\begin{alltt}
cisco#show ip interface brief
Interface                  IP-Address      OK? Method Status                Protocol
FastEthernet0/0            172.21.10.11    YES NVRAM  up                    up
FastEthernet0/1            172.21.11.1     YES NVRAM  up                    up
\end{alltt}
\end{screen}
\end{center}

\subsection*{IPアドレスの設定:ip address コマンド}
インターフェイスモードから行う.

\begin{center}
\begin{screen}
\begin{alltt}
cisco#conf t
cisco(config)#int fa0/0
cisco(config-if)#ip address 192.168.1.2 255.255.255.0
                              (IP)        (netmask)
\end{alltt}
\end{screen}
\end{center}

\subsection*{設定取り消し,IPアドレスの変更など}
IP アドレスの変更は,そのまま別の IP を設定すれば,上書きされる.

IP アドレスの無効化は,取り消したいコマンドに \texttt{no} を付けて実行する.

\begin{center}
\begin{screen}
\begin{alltt}
cisco#conf t
cisco(config)#int fa0/0
cisco(config-if)#no ip address 192.168.1.2 255.255.255.0
                    (先に設定したコマンド)
\end{alltt}
\end{screen}
\end{center}

同様に,アクセスリストなど,既に行った設定を元に戻したい,無効化したい場合は,``no'' を行頭に付けて,消去したいコマンドをコンフィグモードで入力する.

\begin{center}
\begin{screen}
\begin{alltt}
cisco#conf t
cisco(config)#no ip route 222.222.22.22 255.255.255.0 222.221.1.1
 (222.222.22.22 宛ての経路を消したい場合)
\end{alltt}
\end{screen}
\end{center}


\subsection*{インターフェイスの無効化:shutdown コマンド}
インターフェイスを無効化(down)させたい場合に用いる.

Cisco ルータでは,デフォルトでインターフェイスは down になっているの
で,使用時には後述の``no'' を付けた ``no shutdown'' コマンドを実行し,
インターフェイスを有効化する.

\begin{center}
\begin{screen}
\begin{alltt}
cisco#conf t
cisco(config)#int fa0/0
cisco(config-if)#shutdown         ( ← down)
cisco(config-if)#no shutdown      ( ← up)
\end{alltt}
\end{screen}
\end{center}

\subsection*{現在の設定の閲覧}
現在の設定 running-config を表示する.

\begin{center}
\begin{screen}
\begin{alltt}
cisco#show running-config   (show run)
\end{alltt}
\end{screen}
\end{center}

古いOS の場合

\begin{center}
\begin{screen}
\begin{alltt}
cisco#write term
\end{alltt}
\end{screen}
\end{center}


\subsection*{起動設定の閲覧}
再起動時に読み込まれる設定ファイルを表示する.

\begin{center}
\begin{screen}
\begin{alltt}
cisco#show startup-config   (show start = show conf)
\end{alltt}
\end{screen}
\end{center}

\subsection*{現在の設定の保存}
現在の設定を,再起動時に読み込まれる設定ファイルに保存する.

\begin{center}
\begin{screen}
\begin{alltt}
cisco#copy running-config startup-config    (copy run start)
\end{alltt}
\end{screen}
\end{center}

古いOS の場合

\begin{center}
\begin{screen}
\begin{alltt}
cisco#write memory    (wr)
\end{alltt}
\end{screen}
\end{center}

\subsection*{設定の削除 (1行単位)}
設定の削除は,削除したい行のコマンドの行頭に ``no'' を付けたコマンドを実行する.

\begin{center}
\begin{screen}
\begin{alltt}
#cisco(config-if)#no shutdown
   (shutdown されているインターフェイスを up させる)
\end{alltt}
\end{screen}
\end{center}

\subsection*{起動設定の全削除}
設定の全部削除し工場出荷時に戻すには,startup-config を削除する.

\begin{center}
\begin{screen}
\begin{alltt}
#erase startup-config
\end{alltt}
\end{screen}
\end{center}


\clearpage
%% %% %% %%
%% %% %%
%% %% %%  ip route
%% %% %%
%% %% %%
\section{ip route}
\label{cmd:ios-ip-routeping}
Cisco IOS上で,静的経路の設定を行う.

\noindent
{\bf ◆書式}
\begin{center}
\begin{screen}
\begin{alltt}
 (config)#ip route 宛先NWアドレス 宛先ネットマスク 次ルータアドレス (メトリック)
\end{alltt}
\end{screen}
\end{center}


{\bf ◆機能説明}

Cisco IOS 上で,ルーティングテーブルに静的に経路を追加する.管理者が手
動で削除するまで,消去されることはない.

{\bf ◆使用例    ルーティングテーブルの追加}
\begin{center}
\begin{breakbox}
\begin{alltt}
cisco(config)#ip route 192.168.0.0 255.255.255.0 172.21.10.101 2
                         ↑             ↑           ↑        ↑
                     NWアドレス      マスク     次ルータ      距離
\end{alltt}
\end{breakbox}
\end{center}
                

{\bf ◆削除}

削除する場合は,行頭に no を付けると削除される.

\begin{center}
\begin{screen}
\begin{alltt}
 (config)#no ip route 宛先NWアドレス 宛先ネットマスク 次ルータアドレス (メトリック)
\end{alltt}
\end{screen}
\end{center}

\clearpage
%% %% %% %%
%% %% %%
%% %% %%  route
%% %% %%
%% %% %%
\section{show ip route}
\label{cmd:show-ip-route}
Cisco IOS上で,ルーティングテーブルを閲覧する\par

\noindent
{\bf ◆書式}
\begin{center}
\begin{screen}
\begin{alltt}
 #show ip route
\end{alltt}
\end{screen}
\end{center}

{\bf ◆機能説明}

Cisco IOS 上で,ルーティングテーブルの情報を表示する.ルーティングテー
ブルは,宛先ネットワークのネットワークアドレス,次ホップルータのアドレ
ス,接続インターフェースの情報等が表示される.

{\bf ◆使用例    ルーティングテーブルの追加}
\begin{center}
\begin{breakbox}
\begin{alltt}
cisco1>show ip route
Codes: C - connected, S - static, R - RIP, M - mobile, B - BGP
       D - EIGRP, EX - EIGRP external, O - OSPF, IA - OSPF inter area
       N1 - OSPF NSSA external type 1, N2 - OSPF NSSA external type 2
       E1 - OSPF external type 1, E2 - OSPF external type 2
       i - IS-IS, L1 - IS-IS level-1, L2 - IS-IS level-2, ia - IS-IS inter area
       * - candidate default, U - per-user static route, o - ODR
       P - periodic downloaded static route

Gateway of last resort is not set
                (↑該当する経路が設定されていない場合にデフォルトでパケットを
                    転送する先)
     172.21.0.0/24 is subnetted, 10 subnets
                (↑クラスB 172.21.0.0 は /24 でサブネット化されていることを示す
                    VLSM の場合は,Variably subnetted,サブネット化されていない
                    場合は,表示されない)
R       172.21.17.0 [120/1] via 172.21.10.17, 00:00:27, FastEthernet0/0
                (↑172.21.17.0/24 の次ホップは 172.21.10.17 であることを示す.
                    R は RIP で動的に学習した経路であることを示す)
S       172.21.22.0 [1/0] via 172.21.10.19
                (↑172.21.22.0/24 の次ホップは 172.21.10.19 であることを示す.
                    S は ip route コマンドで管理者が手動で設定した経路であることを示す)
C       172.21.11.0 is directly connected, FastEthernet0/1
                (↑172.21.11.0/24 は,このルータの FastEthernet0/1 インターフェースに
                    直接接続されていることを示す.IP アドレスを付与したインターフェースが
                    アップしていれば自動的に追加される)
C       172.21.10.0 is directly connected, FastEthernet0/0
                (↑172.21.10.0/24 は,このルータの FastEthernet0/1 インターフェースに
                    直接接続されていることを示す.IP アドレスを付与したインターフェースが
                    アップしていれば自動的に追加される)
\end{alltt}
\end{breakbox}
\end{center}

{\bf ルーティングテーブルの経路表示情報}
\begin{center}
\begin{breakbox}
\begin{alltt}
R       172.21.17.0 [120/1] via 172.21.10.17, 00:00:27, FastEthernet0/0
↑          ↑       ↑ ↑           ↑          ↑          ↑
(1)         (2)     (3) (4)          (5)         (6)         (7)

(1) : 経路の学習方式の意味
          C: 直接接続された経路   S: 静的経路  R: RIPで学習した経路
          O: OSPFで学習した経路   B: BGPで学習した経路
(2) : 宛先ネットワークのネットワークアドレス
          VLSM などの場合は,172.21.15.0/26 等のようにサブネットマスク情報も
          prefix 表記で表示される.
(3) : Administrative Distance
          経路の管理距離(学習する方式によって決まる)
          同一宛先について複数経路がある場合は値が小さいものが優先される
(4) : メトリック
          経路の距離
          同一宛先について複数経路があり,(3) の値が等しい場合は,
          メトリック値が小さいものが優先される
(5) : 次ホップルータの IP アドレス
          パケットを次に転送するべきルータの IP アドレス
(6) : 動的ルーティングで学習した経路の経過時間
          動的ルーティングで経路を学習してから経過した時間
(7) : 次ホップルータのインターフェース
          次ホップルータが接続されているインターフェース
\end{alltt}
\end{breakbox}
\end{center}

\clearpage
%% %% %%
%% %% %% %%

\clearpage
%% %% %% %%
%% %% %%
%% %% %%  ip route
%% %% %%
%% %% %%
\section{show mac-address-table}
\label{cmd:ios-mac-address}
Cisco IOS上で,スイッチ(ブリッジ)の MAC アドレステーブルを確認する.

\noindent
{\bf ◆書式}
\begin{center}
\begin{screen}
\begin{alltt}
ルータの場合
Router#show mac-address-table

スイッチの場合
Switch#show mac address-table

\end{alltt}
\end{screen}
\end{center}


{\bf ◆機能説明}

どのポートの先に,どの MAC アドレスの端末がいるかが確認できる.

スイッチやルータで確認できるが,L2スイッチ機能のないルータでは実行できない(MAC アドレステーブルそのものがない).

{\bf ◆使用例}
\begin{center}
\begin{breakbox}
\begin{alltt}
Switch#show mac address-table
          Mac Address Table
-------------------------------------------

Vlan    Mac Address       Type        Ports
----    -----------       --------    -----
   1    109a.dd4f.0df5    DYNAMIC     Fa0/9
   1    e05f.b90d.47b5    DYNAMIC     Fa0/1

スイッチの9番ポートの先に,10:9A:DD:4F:0D:F5
の MAC アドレスを持つ端末が存在する.

スイッチの10番ポートの先に E0:5F:B9:0D:47:B5
の MAC アドレスを持つ端末が存在する.

\end{alltt}
\end{breakbox}
\end{center}
                


\clearpage
%% %% %% %%
%% %% %%
%% %% %%  ip route
%% %% %%
%% %% %%
\section{show arp}
\label{cmd:ios-arp}
Cisco IOS上で,ルータの ARP テーブルを確認する.

\noindent
{\bf ◆書式}
\begin{center}
\begin{screen}
\begin{alltt}
Router>show arp

\end{alltt}
\end{screen}
\end{center}


{\bf ◆機能説明}

どの IP アドレスが,どの MAC アドレスかを確認する.

ただし,ARP (Address Resolution Protocol) でこれまで確認できたものか,
もしくは静的に arp コマンドで管理者が設定したもののみ確認できる.

{\bf ◆使用例}
\begin{center}
\begin{breakbox}
\begin{alltt}
Router>show arp
Protocol  Address          Age (min)  Hardware Addr   Type   Interface
Internet  172.21.22.2           103   e0cb.4e7a.af48  ARPA   FastEthernet1
Internet  172.21.22.9             -   e05f.b90d.47b5  ARPA   FastEthernet1
Internet  172.21.22.10            3   04c5.a47e.a140  ARPA   FastEthernet1
Internet  192.168.0.1            14   0015.17ee.73a0  ARPA   FastEthernet0
Internet  192.168.0.189           -   e05f.b90d.47b4  ARPA   FastEthernet0
Internet  192.168.0.225          13   106f.3f04.2fd0  ARPA   FastEthernet0

Address が IP アドレスを,
Hardware Addr は MAC アドレスを示す.

\end{alltt}
\end{breakbox}
\end{center}
                


\clearpage
%% %% %% %%
%% %% %%
%% %% %%  ip route
%% %% %%
%% %% %%
\section{show vlan}
\label{cmd:ios-vlan}
Cisco IOS上で,ルータ・スイッチの VLAN 設定状況を確認する.

\noindent
{\bf ◆書式}
\begin{center}
\begin{screen}
\begin{alltt}
Switch>show arp

\end{alltt}
\end{screen}
\end{center}


{\bf ◆機能説明}

どのポートが,どの VLAN に所属しているかを確認する.

{\bf ◆使用例}
\begin{center}
\begin{breakbox}
\begin{alltt}
Switch>show vlan

VLAN Name                             Status    Ports
---- -------------------------------- --------- -------------------------------
1    default                          active    Fa0/1, Fa0/2, Fa0/3, Fa0/4
7    VLAN0007                         active
9    group09                          active    Fa0/18, Fa0/19
12   group12                          active    Fa0/20, Fa0/24
13   teststp                          active

ポート1-4 が VLAN 1 (デフォルト),18,19 が VLAN 9,
20, 24 が VLAN 12 である.
\end{alltt}
\end{breakbox}
\end{center}
                


\clearpage
%% %% %% %%
%% %% %%
%% %% %%  ip route
%% %% %%
%% %% %%
\section{show interface status}
\label{cmd:ios-show-int-status}
Cisco IOS上で,スイッチのポート(インターフェース)の状況を確認する.

\noindent
{\bf ◆書式}
\begin{center}
\begin{screen}
\begin{alltt}
Switch>show interface status
        (sh int status)

\end{alltt}
\end{screen}
\end{center}


{\bf ◆機能説明}

静的 VLAN, ポートの up/down (有効・無効,接続有り/無し)などが分かる.

{\bf ◆使用例}
\begin{center}
\begin{breakbox}
\begin{alltt}
Switch>show int status

Port      Name               Status       Vlan       Duplex  Speed Type
Fa0/1                        connected    1          a-full  a-100 10/100BaseTX
Fa0/2                        notconnect   1            auto   auto 10/100BaseTX
Fa0/3                        connected    1          a-full  a-100 10/100BaseTX
Fa0/4                        notconnect   1            auto   auto 10/100BaseTX
Fa0/5                        connected    1          a-full  a-100 10/100BaseTX
Fa0/6                        notconnect   1            auto   auto 10/100BaseTX
Fa0/7                        notconnect   1            auto   auto 10/100BaseTX
Fa0/8                        notconnect   1            auto   auto 10/100BaseTX
Fa0/9                        connected    1          a-full  a-100 10/100BaseTX
Fa0/10                       notconnect   1            auto   auto 10/100BaseTX
Fa0/11                       notconnect   1            auto   auto 10/100BaseTX
Fa0/12                       notconnect   1            auto   auto 10/100BaseTX
Fa0/13                       notconnect   1            auto   auto 10/100BaseTX
Fa0/14                       notconnect   1            auto   auto 10/100BaseTX
Fa0/15                       notconnect   1            auto   auto 10/100BaseTX
Fa0/16                       notconnect   1            auto   auto 10/100BaseTX
Fa0/17                       notconnect   1            auto   auto 10/100BaseTX
Fa0/18                       notconnect   1            auto   auto 10/100BaseTX
Fa0/19                       notconnect   1            auto   auto 10/100BaseTX
Fa0/20                       notconnect   1            auto   auto 10/100BaseTX
Fa0/21                       notconnect   1            auto   auto 10/100BaseTX
Fa0/22                       notconnect   1            auto   auto 10/100BaseTX
Fa0/23                       notconnect   1            auto   auto 10/100BaseTX
Fa0/24                       notconnect   1            auto   auto 10/100BaseTX

connected は端末が接続中であり,a-full, a100 は automatic
(自動認識)で全二重,100M で接続中であることを示す.
not connected は接続されていない,あるいは
電源が入っていないことを示す.
VLAN は全て 1 である.

\end{alltt}
\end{breakbox}
\end{center}
                
\clearpage

%% %% %% %%
%% %% %%
%% %% %%  ip route
%% %% %%
%% %% %%
\section{show ip interface brief}
\label{cmd:ios-show-ip-int-brief}
Cisco IOS上で,ルータ・スイッチのインターフェースのIPアドレスとリンク状況を確認する.

\noindent
{\bf ◆書式}
\begin{center}
\begin{screen}
\begin{alltt}
Router>show ip interface brief
        (sh ip int brief)

\end{alltt}
\end{screen}
\end{center}


{\bf ◆機能説明}

ルータ・スイッチのインターフェースのIPアドレス,up/down (有効・無効,接続有り/無し)などが分かる.




\clearpage
%% %% %% %%
%% %% %%
%% %% %%  ip route
%% %% %%
%% %% %%
\section{show ip nat translations, ip nat}
\label{cmd:ios-show-int-status-nat}
Cisco IOS上で,NAT (NAPT, IP masquerading) の設定,および,NATテーブル確認を行う.

\noindent
{\bf ◆書式}
\begin{center}
\begin{screen}
\begin{alltt}
静的 NAT の設定
Router(config)#ip nat inside source static tcp 172.21.22.100 80 222.229.69.3 8080

内側(inside) の(送信元)IPアドレス 172.21.22.100, TCP ポート 80 を,
外側(outside, インターネット側) に対して,222.229.69.3 TCP ポート 8080 で
公開する.

NATテーブルの確認
Router#show ip nat translations
Pro Inside global         Inside local          Outside local         Outside global
tcp 222.229.69.3:8080     172.21.22.100:80      ---                   ---

静的 NAT の設定削除
Router(config)#no ip nat inside source static tcp 172.21.22.100 80 222.229.69.3 8080

NATテーブルから削除
Router#clear ip nat translations *

\end{alltt}
\end{screen}
\end{center}


{\bf ◆機能説明}

どのポートが,どの VLAN に所属しているかを確認する.

{\bf ◆使用例}
\begin{center}
\begin{breakbox}
\begin{alltt}
静的 NAT の設定
Router(config)#ip nat inside source static tcp 172.21.22.100 80 222.229.69.3 8080

内側(inside) の(送信元)IPアドレス 172.21.22.100, TCP ポート 80 を,
外側(outside, インターネット側) に対して,222.229.69.3 TCP ポート 8080 で
公開する.

すなわち外から,222.229.69.3 TCP ポート 8080 へのアクセスがあれば,
内側のサーバ 172.21.22.100, TCP ポート 80 へ転送
(ポートフォワーディング)する.

Router#show ip nat translations
Pro Inside global         Inside local          Outside local         Outside global
tcp 222.229.69.3:8080     172.21.22.100:80      ---                   ---

NAT 設定の確認
\end{alltt}
\end{breakbox}
\end{center}
                


\clearpage
%% %% %% %%
%% %% %%
%% %% %%  ip nat
%% %% %%
%% %% %%
\section{show ip nat translations: NATの設定確認・削除}
\label{cmd:ios-show-int-status-nat}
Cisco IOS上で,NAT (NAPT, IP masquerading) の設定,および,NATテーブル確認を行う.

また,削除方法についても本項目に記述する.


\noindent
{\bf ◆書式}
\begin{center}
\begin{screen}
\begin{alltt}
【NATテーブルの確認】
Router#show ip nat translations
Pro Inside global         Inside local          Outside local         Outside global
tcp 222.229.69.3:8080     172.21.22.100:80      ---                   ---

【NATテーブルからNATエントリ削除】
Router#clear ip nat translation *

(次項NAT設定削除に先立って行う必要がある場合が多い.
 既にNAT通信中の場合でエントリがある場合は,NATの
 設定削除が行えないため)

【NATの設定削除】
静的 NAT の設定削除 (no を付けて消したい内容を記述)

Router(config)#no ip nat inside source static tcp 172.21.22.100 80 222.229.69.3 8080
\end{alltt}
\end{screen}
\end{center}


{\bf ◆使用例}
\begin{center}
\begin{breakbox}
\begin{alltt}
Router#show ip nat translations
Pro Inside global         Inside local          Outside local         Outside global
tcp 222.229.69.3:8080     172.21.22.100:80      ---                   ---

\end{alltt}
\end{breakbox}
\end{center}
                


\clearpage
