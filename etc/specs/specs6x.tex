\section{第9回 WWWサーバ 応用課題}

WWW サーバのアクセス制限および POP サーバでは,正当な権限のあるユーザ以
外のユーザは,勝手にウェブコンテンツやメールデータをダウンロードできな
いようにパスワードによる保護を行った.

しかし,HTTP プロトコル,POP プロトコルともに,データには一切の加工をせ
ずに,そのままネットワーク上へ送信されるので,通信経路の途中で悪意ある
第三者により盗聴がされた場合,パスワード,データなどが全て盗み見られて
しまう.このような行為を行うツールに,パケットキャプチャ,スニファと呼
ばれるソフトウェアがあるが,コマンドラインの tcpdump
(windump),WireShark などが有名なものである.

今回は,Windows に WireShark をインストールすることで,Windows から
FreeBSD への通信経路上にどのようなデータが流れているか確認を行う.

\subsection*{考慮すべき点}
今回の実験を行うに当たっては以下のようなことについて考慮する必要がある.
\begin{itemize}
  \item \textbf{パケットキャプチャ}\\
         どんな場合にどこでキャプチャするのが有効か.\\
         目的のパケットをどうやって探すか.
  \item \textbf{キャプチャしたパケット}\\
         ネットワーク上にはどのような通信が流れていたか.\\
         どのような人に登録を許可するか.\\
         どのような場合に登録者を削除するか.\\
       \item \textbf{防御案}\\
         パケットキャプチャからパスワードやデータを保護するには,どのよ
         うにしたら良いか.\\
         POP であれば,チャレンジ\&レスポンス認証をサポートし
         た APOP,http であれば MD5 を利用した Digest 認証などがあり,
         また SSL による暗号化なども一般的になってきたが,その他にどの
         ような安全性を高める仕組みが考えられるか.
\end{itemize}

