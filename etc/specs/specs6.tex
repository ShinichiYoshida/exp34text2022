\section{第6回 WWWサーバ}

ネットワーク上での情報発信の手段の1つとしてWWW(World Wide Web)サービス
がある.情報の発信は,WWWサーバにHTML(Hyper Text Markup Language)ファイ
ルなどを公開ディレクトリに保存することで可能となる.クライアント側
はWWWサーバとファイルを示す文字列(URL:Uniform Resource Locator)を用い,
ブラウザと呼ばれるソフトウェアで目的のファイルを閲覧することができる.
このWWWサーバとブラウザとの通信にはHTTP(Hyper Text Transfer Protocol)が
用いられている.WWWサービスはプラットフォームに依存せずにブラウザによっ
て情報を取得できることから,世界中で利用されており,今日のインターネッ
ト技術の発展の要因と言っても過言ではない.

サーバコンピュータを\textbf{\Underline{WWWサーバとして動作}}させるため
に,WWWサーバソフトウェアをインストールし,ファイルを置くディレクトリを
指定する必要がある.また,POPサーバなどと同様
に,\textbf{\Underline{DNSに別名を登録}}することによって理解しやす
いURLを用いることが可能となる.

\subsubsection*{Apache}
Apacheは1995年にNCSA(Natinal Center for Supercomputing Applications)
httpd 1.3にパッチを当てたものから開発が始まったWWWサーバ用のソフトウェ
アである.現在もApache Software
Foundation\footnote{http://www.apache.org/}で開発が続けられており,ほと
んどのプラットフォーム上で動作できるようになっている.

\subsubsection*{コンテンツの配置}
WWWサーバを構築し,テストページの表示を確認しただけでは情報発信したとは
いえない.実際にHTMLファイルなどを作成し\textbf{\Underline{WWWサービス
    を用いて公開}}する.また公開する情報によって
は,\textbf{\Underline{閲覧できる利用者を制限}}する必要も出てくる.グルー
プのホームページを作成し,WWWサービスを用いて公開する.さらに重要な情報
に対しては\textbf{\Underline{正当な利用者以外は閲覧できないように設
    定}}を施す.

\subsubsection*{グループのホームページの公開,設定記録の共有}
ホームページを作成し,WWWサービスを用いて公開する.ホームページの内容に
関しては特に規定はしないが,最低限の公序良俗に反しないようにする.ただ
し,全く情報の無いものはホームページとして認めない.さらに設定記録を共
有するためにWWWで公開する.設定記録はネットワーク内の重要な情報であるの
で,グループのメンバ以外は閲覧できないようにする.WWWサービスによる情報
に対して利用制限をかけるにはさまざまな方法が存在するが,今回
は Apache の機能を用いて行う.どのように制限をかけるかというポリシーを
決定し,それに即した設定をhttpd.confに記述する.

\subsubsection*{考慮すべき点}
今回の実験を行うにあたり,以下のようなことについて考慮する必要がある.

\begin{itemize}
{\bf \item{WWWサーバの動作}}\\
WWWサーバとブラウザソフトとの通信ではHTTP上で行われることは前述した.
HTTP上の通信でWWWサーバはどのような動作をするのか把握しておく必要がある.

{\bf \item{WWWサーバ用のソフトウェア}}\\
今回は Apache を使用したが,WWWサーバを構築するのに他のソフトウェアには
どのようなものが存在するのか,それぞれどのような特徴があるのか把握して
おく.その上で,Apache を使用する理由を考える必要がある.

{\bf \item{WWWページの閲覧制限}}\\
どのように不正な利用者を判別するか.
今回は Apache の機能を用いて実験を行ったが,他の利用制限方法にはどんなものがあるか.
\end{itemize}
