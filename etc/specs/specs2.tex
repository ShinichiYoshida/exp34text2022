\section{第2回 バックボーンの構築とネットワークの設定}

ネットワーク内で,メールの送受信やWWWページの閲覧といった機
能\footnote{このような機能をサービスと呼ぶ.}を提供するためには,ハブや
ルータ,サーバ/クライアントなどといったものが必要である.しかし,それら
の機器があるだけではネットワークとして機能しない.本実験では,それらの
機器を\textbf{\Underline{ネットワークとして機能するように接続}}し,機器
同士が\textbf{\Underline{TCP/IPで通信できるような環境}}を構築する.

まず,各機器を接続する Ethernet ケーブルとして,カテゴリ5 の UTP ケーブ
ルを作成し,\textbf{\Underline{各機器を接続する}}.その後で,ネットワー
ク内で機器同士が通信できるようにするため,IPアドレスやゲートウェイなど
を設定する.

\subsubsection*{ネットワークの構成機器}
本実験で構築するネットワークの構成機器とその役割との対応を表\ref{sp2:tab:network}に示す.

\begin{table}[htbp]
\begin{center}
\caption{ネットワーク機器とその役割}
\label{sp2:tab:network}
\begin{tabular}{|l|c|}
\hline
ネットワーク機器 & 役割 \\ \hline
Cisco Systems CISCO2611XM & ルータ \\ \hline
ASUS RS-100 & サーバ \\ \hline
DOS/V (Linux) & クライアント \\ \hline
DOS/V (WindowsXP) & クライアント \\ \hline
Acer Aspire One & クライアント \\ \hline
Allied Telesis centrecom8216XL2 & スイッチングハブ \\
\hline
\end{tabular}
\end{center}
\end{table}

\subsubsection*{考慮すべき点}
今回の実験を行うにあたり,以下のようなことについて考慮する必要がある.
\begin{itemize}
{\bf \item{ネットワーク機器の種類}}\\
OSと同様に,ネットワーク機器にもさまざまな種類があり,それぞれの用途や特徴が異なっている.
どのようなネットワーク機器がどのような動きをするのか把握し,目的に応じて正しいネットワーク構築をする必要がある.

{\bf \item{ネットワークを構成するための情報}}

どのようなネットワークを構築するかで,それぞれのネットワーク機器に設定する情報は全く違ってくる.
構築したいネットワークを意識し,入力すべき情報を考える必要がある.
\end{itemize}
