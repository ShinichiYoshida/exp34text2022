\section{第5回 メールサーバ 応用課題}

postfix のデフォルトのインスールでは,グループ内のハブに接続されたコン
ピュータから,他グループへ電子メールを発信する際に,SMTP サーバとして機
能することができるが,他グループから他グループへメールを発信することは
許可されていない.

この理由として,基本的に,他のネットワークから他のネットワークへのリレー
(中継)は,第三者中継に使われる可能性が高く,大量のスパムメール,犯罪
や悪質ないたずらといった,好ましくないメール送信の踏台に使われる恐れが
高いためである.

しかし,営業所が複数存在する会社などにおいてメール送信のための SMTP サー
バを全ての営業所に置かずに,本社のメールサーバを経由させて送信する場合
など,一定の信頼のおけるネットワークからのメール中継を許可したい場合が
ある.

そこで今回は,\Underline{他のグループの IP アドレスのうち,信頼のおけ
  る IP アドレスからのメールの送信を許可する設定を行う.}

\subsection*{メールの中継}

メールの中継(リレー)とは,他のメールサーバ(もしくは MUA クライアント)