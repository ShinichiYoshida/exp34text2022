\section{第9回 DHCPサーバ応用課題}

\subsection{NTPサービスのインストール}

一般的なコンピュータのオペレーテイングシステムでは,CPUなどのマザーボー
ド上のデジタル回路を同期させるためのクロックを用いて,時刻の情報を保持
している.これを,システム時計,ソフトウェア時計,カーネル時計などと呼
ぶ.また,電源が切られコンピュータが動作していない間も,時刻情報を保持
するために,マザーボード上には,水晶(クォーツ)による時計も内蔵されて
いる.これをリアルタイムクロック(RTC),CMOS クロック,ハードウェア時
計などと呼ぶ.オペレーティングシステムは,起動時にRTCを参照して時刻を読
み込み,同期クロックを用いて時刻を進めながら,時刻情報を保持する.

どちらの時計も水晶振動子によって時刻を進めているが,製造時の不純物の混
入などによる水晶の固体差,コンピュータ内部からの電磁波の影響,熱の影響
などにより,誤差が生じる.この誤差が累積するとコンピュータの時計の時刻
は大幅に狂うことになるため,定期的にコンピュータの時計を修正する必要が
ある.特に,ネットワークに接続された環境において,ネットワーク上のコン
ピュータの時刻が大きく異なると,ファイル共有時のファイルシステムに記録
された時刻や,ログの時刻が異なることになり,ネットワークの運営・運用に
支障をきたす.

ネットワーク上のコンピュータの時刻を自動的に修正するための仕組みとし
て, NTP(Network Time Protocol)がある.NTP では,ネットワーク上に時刻
情報を配信する NTP サーバを設置し,その他のコンピュータを NTP クライア
ントとして設定することで,NTP クライアントの時刻を NTP サーバに自動的に
合わせることができる.

NTP での接続には,下記の3つもモードが存在する.
 \begin{enumerate}
 \item {\bf Symmetric Active/Passive Mode}\\
   時刻を同期する2つのコンピュータの立場が対等であるモード.あるコン
   ピュータが NTP サーバに Symmetric Active モードアクセスすると,サー
   バは Symmetric Passive モードで応答する.Symmetric Active サーバ
   がSymmetric Passive サーバよりも,時刻の精度が悪いと判断されれ
   ば,Symmetric Passive Mode のサーバの時刻に同期する.通常は,複数の
   時刻サーバのアドレスを同期しておく場合に用いられる.

 \item {\bf Client/Server Mode}\\
   時刻の同期が対等でなく,NTP でサーバにアクセスしたコンピュータの時刻
   が,常に NTP サーバの時刻に同期する.通常のネットワークにおいては,
   クライアントはこのモードで NTP サーバにアクセスする.

 \item {\bf Broadcast/Muticast Mode}\\
   NTP サーバとクライアントは,直接,一対一の通信行わない.サーバは,ク
   ライアントが存在するか否かに関わらず,ブロードキャストアドレスあるい
   はマルチキャストアドレス宛に時刻情報を流し,クライアントは特定のサー
   バにアクセスすることなく,ネットワーク上をブロードキャストあるいはマ
   ルチキャストで流れてくる時刻情報があれば,その時刻に同期する.
 \end{enumerate}

\subsection{NTPサービスのインストール}

本実験では,freebsdX に NTP サーバをインストールし,自らが上位の NTP サーバ
のクライアントとして時刻を上位 NTP サーバに同期させ,また LAN 内のコン
ピュータの NTP サーバとして,時刻を同期させる.NTP サーバには,the
Network Time Protocol project (ntp.org) が開発している ntpd を用いる.
FreeBSD にも標準で ntpd が付属しているが,これはバージョンが古いため削
除し,最新の ntpd をインストールする.

まず,現在インストールされている ntpd を削除する.ntp に該当す
るFreeBSD のパッケージ名は,SUNWntpr と SUNWntpu である.これらを
pkgrm コマンドを用いて削除する(通常は pkgrm の引数にパッケージ名を記述
するだけで良い).

次に,ftp サーバから \texttt{ntp-4.2.4p6.tar.gz} をダウンロードし
\texttt{/usr/archive} に展開してコンパイルする.コンパイル
は,INSTALL ファイルに書かれている手順で行う.

標準では ntpd の実行ファイルは,\texttt{/usr/local/bin} にインストール
される.

次に,ntpd の設定ファイル ntp.conf の記述を行う.ntp.conf の設定につい
ては,html ディレクトリにある ntpd ファイルに詳細が記述されている.

ここでは,上位の NTP サーバ 172.21.10.1 にクライアントモードで接続する.
/etc/ntp.conf ファイルに server 172.21.10.1 の1行を書いておくだけで良い.

ntpdate 172.21.10.1 を3〜4回実行し,freebsdX の時計を合わせる.そして,
ntpd を実行する.

動作チェックは,下記のように ntpq コマンドで行う.remote サーバと同期が
取れている場合は,行頭に ``*'' が表示される.同期が取れていない場合
は,``='' が表示される.ntpd は,大きな時刻差を修正するようには作られて
おらず,小さい時刻間隔で微調整する.このため,ntpd が安定的に動作するま
で,少し時間がかかる.
\begin{center}
\begin{breakbox}
\begin{alltt}
# \underline{ntpq -p}
     remote           refid      st t when poll reach   delay   offset  jitter
==============================================================================
*172.21.10.1     LOCAL(0)         6 u   38  256  377    0.725    0.176   0.252
\end{alltt}
\end{breakbox}
\end{center}

Linux と Windows のクライアントコンピュータの時刻設定を,NTP を用いて行
うように設定する.Linux は,FreeBSD と同様に ntpd を用いて同期す
る./etc/ntp.conf で NTP サーバをインストールした FreeBSD サーバを指定
する.Windows では,タスクバー右の時計をダブルクリックし,「インターネッ
ト時刻」のタブを開き,「自動的にインターネット時刻サーバと同期する」の
チェックボックスにチェックを行い,「サーバ」として FreeBSD サーバを指定
する.「今すぐ更新」ボタンを押し,時刻を修正する.

以上で,NTP による時刻同期の環境が構築された.この他,NTP サーバの設定
には,上位サーバを用いず内蔵時計を配信するためのローカル時計参照モード
(server 127.127.1.0 を指定する)や,時刻のみでなく,内蔵時計の進み方の
調整(driftfile の設定.システムクロックの進み方を調整することで,より
正確な時刻を刻めるようになるので,NTP による同期を頻度を少なくできる)
など,アクセス制限,認証,暗号化などのセキュリティ設定など,多くの設定
を行うことができる.

