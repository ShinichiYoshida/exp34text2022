\section{第8回 バックアップ}
システムを正常に運用していくためには, 障害への対処が必要となる. 

例えば停電は予想することが非常に難しいが, 発生すれば確実にシステムが停止してしまう. 
\textbf{\Underline{停電に対処する方法}}として, UPS(無停電電源装置)がある. 

また, ディスクやテープなどに保存された電子データは, 喪失や破壊の可能性
がある. データの喪失や破壊に備えて, 重要なデータをコピーし二重に保存し
ておくことをデータバックアップと呼ぶ. データのコピーを定期的に保存する
ことにより, 保存を行った時点までのデータを保護でき, 喪失や破壊があった
時にはバックアップ時点の状態に復元できる.

データの喪失や破壊の原因は, ハードウェア障害, ユーザの誤操作, プログラ
ムの不具合など様々である. ハードディスクドライブなどの記憶装置自体の故
障では, ドライブ内に保存されたデータ全てが喪失してしまうことがある. ハー
ドディスクドライブの平均故障時間\footnote{MTBFとも呼ばれる. 計算式は 
  製品の稼働時間/故障件数 で値が大きいほど製品の信頼性は高い. }は長い
製品では100〜120万時間だが, これは平均時間であり, 当然これより短い時間
で故障することもある.

データのバックアップを行う時は, バックアップデータの保存先を別の記録媒
体にする. これは, 元のデータとバックアップデータを同じ記憶媒体に保存す
ると, 記憶媒体自体に障害が発生した場合, 元データとバックアップデータの
両方が喪失, 破壊されてしまうためである.

本実験ではFreeBSDを無停電電源装置を接続する.
また、\textbf{\Underline{ハードディスクドライブ内に記憶されたファイルの
    バックアップ}}を行う. また,バックアップは,FreeBSD サーバのデータを 
    Linux コンピュータ上のハードディスクへ NFS 経由で tar を用いて行
    う.

\subsection*{考慮すべき点}
今回の実験を行うに当たっては以下のようなことについて考慮する必要がある.
\begin{itemize}
  \item \textbf{バックアップ}\\
         どのファイルをバックアップするか. \\
         バックアップの頻度, 行うタイミング(時期や時間)をどのように設定するか. \\
         バックアップの方式にはどのようなものがあるか. 

       \item \textbf{バックアップソフトウェア}\\
         tar コマンド以外に,どのようなバックアップの手段,ソフトウェ
         アがあるか.それぞれの特徴や,それらの間の利点,欠点を考慮せよ.
         
\end{itemize}
