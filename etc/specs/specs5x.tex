\section{第5回 メールサーバ 応用課題}

メールサーバには,MTA である postfix と MDA である qpopper がインストー
ルされている.ユーザがメールを読み書きするためには,MUA (Mail User
Agent) が必要であり,これまではクライアントコンピュータ上にインストール
(または標準添付)されたものを用いた.

しかし,メールサーバの管理を行う上でサーバのトラブルの際の対処など,メー
ルサーバ上にも最低限の電子メールの読み書きの環境があることが望ましい.

UNIX に必ず添付されている MUA として mail コマンド
があるが,これは編集能力が無いため外部のテキストエディタであらかじめ文
面を作成しておくことが必要なこと,\textbf{\Underline{MIME規格}} に対応
できず最近の MUA から送信されたメールの確認が難しいことなど,使用には難
がある.

そこで,多機能テキストエディタの emacs と,emacs lisp で記述さ
れ emacs 上で動作する MUA である mew をインストールする.

まず,mew は FreeBSD のパッケージが用意されているため,これを利用する.

\subsubsection*{動作確認}
.emacs の設定を行い,smtp サーバや pop サーバの設定を適切に行ったら,メー
ルの送受信のテストを行う.

\subsubsection*{考慮すべき点}
今回の課題を行うにあたり,以下のようなことについて考慮する必要がある.
\begin{itemize}
  {\bf \item{メールサーバ上でのメールソフトの動作}}\\
  サーバコンピュータで直接メールの読み書きを行うことと,クライアントコ
  ンピュータの MUA でメールの読み書きを行うこととでは,どのような点に違
  いがあるか.
\end{itemize}
