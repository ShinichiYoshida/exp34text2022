\section{第14回 LANの分割 --サブネット--}

IPv4において,IPアドレスはクラスA,B,C 3つに分けられており\footnote{実
  験用にDとEもある.},それぞれネットワークアドレスは先頭か
ら8,16,24ビットとなっている.

オフィスや大学でのLANの運用を考えたときに,部署や学科,研究室などをそれ
ぞれ別のネットワークとして管理すると,\textbf{\Underline{ブロードキャス
    トなどのトラフィック}}が抑えられる.ネットワークごとに別のアドレス
を割り当てる方法は,IPアドレスは限りがあるため現実的ではない.そこ
で,A$\sim$Cクラスで分けたままでは,\textbf{\Underline{ホストアドレスが
    無駄}}になる可能性があることに着目し,ネットマスクを延長すること
で,1つのネットワークアドレスを複数のネットワークアドレスとして分割する.
分割されたそれぞれのネットワークをサブネットと呼び,この場合のネットマ
スクを特にサブネットマスクと呼ぶ.サブネットに分割すると,サブネットの
数だけ\textbf{\Underline{ブロードキャストアドレスとして本来ホストアドレ
    スであったIPアドレスが減ってしまう}}ので,どのようなポリシーで分割
するのかが重要である.

今回の実験では,各グループのネットワークをそれぞれ4つのサブネットに分割
する.そのうち1つのサブネットのみを用いて各グループのネットワークを構築
し直し,残りの3つのサブネットは他の組織に分配するものとする.自グループ
の各機器の\textbf{\Underline{IPアドレスとサブネットマスクを正しく設
    定}}し直す.

\subsubsection*{考慮すべき点}
今回の実験を行うにあたり,以下のようなことについて考慮する必要がある.
\begin{itemize}
  {\bf \item{ネットマスクの延長}}\\
  ネットマスクを延長することで,分割したネットワーク上ではいくつのホス
  トが存在できるのか,そのネットワーク自体はいくつできるのか把握してお
  くことが必要である.必要なホスト数とネットワーク数を把握した上で,適
  切なネットマスクを設定することが重要である.

  {\bf \item{分割のメリット,デメリット}}\\
  サブネットにLANを分割すると管理面やリソースの面で利点があるが,利用で
  きるIPアドレス自体は減少することは前述した.そこで,どのような場合に
  分割するべきなのか,逆に分割してはならないのはどんな場合なのか考える
  必要がある.
\end{itemize}

\subsubsection*{応用課題}
サブネットを設定する前の状態,すなわちすべての機器が172.21.1X.0/24のネッ
トワークに属するようにする.
