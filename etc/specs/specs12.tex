\section{第12回 TCP/IPネットワークのルーティング}

TCP/IP ネットワークではデータをパケットに分割し,パケット単位で通信を行
うパケット通信方式を採用する.各パケットは,ヘッダとペイロード(データ
部分)があり,途中の中継機器ではヘッダ部分のみが参照される.郵便に例え
れば,宛名,差出人がヘッダで,本文がペイロードである.途中の郵便局では,
配送のために本文を参照することはない.途中の中継機器は,ヘッダの中の宛
先IPアドレスのみを参照して,次に転送すべき中継機器を決定し転送を行う.
途中の中継機器は,全ての宛先について大まかな経路と距離を保持している.
全てのパケットは,この情報を元に配信される.このことから,TCP/IP ネット
ワークの IP レベルでの中継機器をルータと呼ぶ.

ルータは,宛先ネットワークと対応する次の転送先ルータの情報を保持する.
この保持した情報の組を,ルーティングテーブル(経路表)
最も簡単な宛先ネットワークの管理は,管理者の手動による静的なルーティン
グで,ルータの管理の最も重要なスキルである.

今回の実験では,これまで自動的に経路表を作成してきた RIP プロトコルを停
止して,Cisco ルータ IOS 上にてルーティングテーブルを追加する.

\subsection*{考慮すべき点}
今回の実験を行うに当たっては以下のようなことについて考慮する必要がある.
\begin{itemize}
  \item \textbf{ルーティング}\\
    動的ルーティングと静的ルーティングの違いを考えよ.\\
\end{itemize}

