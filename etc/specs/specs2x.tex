\section{第2回応用課題 クロスケーブルの作成}

第2回の基本実験では,カテゴリ5ケーブルを作成した.作成したケーブルはス
トレートケーブルと呼ばれ,集線装置(ハブ,スイッチ)と各機器との接続に
用いられる.ストレートケーブルに対して,集線装置を介さず
に\textbf{\Underline{各機器同士を直接接続}}したり集線装置同士を接続する
にはクロスケーブルを用いる.クロスケーブルを作成する際には,ストレート
ケーブルと同様に片方の端子では表\ref{tab:T568B}のように配線するが
(TIA/EIA-568-B),他方の端子では表\ref{tab:T568A}のように
(TIA/EIA-568-A)する.表に示される通り,片方の端の1・2番のペアは,もう
一端の3・6番のペアに接続される.

作成したクロスケーブルを用いて,xpX とノートPC をハブを介さずに直接接続
してみる.問題なく通信ができることを確認したら,配線を元に戻す.

\begin{table}[htbp]
\begin{center}
\caption{TIA/EIA-568-B 規格で定められる結線}
\label{tab:T568B}
\begin{tabular}{|c|c|c|c|c|c|c|c|}
\hline
1 & 2 & 3 & 4 & 5 & 6 & 7 & 8 \\ \hline
白/橙 & 橙 & 白/緑 & 青 & 白/青 & 緑 & 白/茶 & 茶 \\
\hline
\end{tabular}
\end{center}
\end{table}

\begin{table}[htbp]
\begin{center}
\caption{TIA/EIA-568-A 規格で定められる結線}
\label{tab:T568A}
\begin{tabular}{|c|c|c|c|c|c|c|c|}
\hline
1 & 2 & 3 & 4 & 5 & 6 & 7 & 8 \\ \hline
白/緑 & 緑 & 白/橙 & 青 & 白/青 & 橙 & 白/茶 & 茶 \\
\hline
\end{tabular}
\end{center}
\end{table}

\subsubsection*{考慮すべき点}
クロスケーブルを作成するに当たって,そもそもなぜ作成するのか考える必要がある.
使用する環境を想定し,実際に行ってみることが大切である.
