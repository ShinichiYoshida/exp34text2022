\section{第13回 ファイアウォールの構築}
他のネットワークと接続されたネットワークやサーバは,外部からのアクセス
も受けることになる.そのアクセスの中には外部向けに提供しているサービス
へのアクセス以外にも,\textbf{\Underline{サーバへの攻撃や侵入などの悪意
    のあるアクセスやウィルスによる攻撃}}なども含まれている可能性がある.
\textbf{\Underline{そこでサーバに対しては,外部との必要最小限の通信以外
    はルータで遮断する.}}

またクライアントコンピュータは,単体では防御が脆弱であることも多く,外
部からのアクセスを受ける必要性も無いため,通常は外部との通信を遮断して
おく方が安全な運用である.そこでサーバ以外の機器に対しては,ルータで外
部との通信を一切遮断する.

クライアントコンピュータを外部から遮断すると,メール・DNS はサーバ経由
で通信することができるが,クライアントコンピュータによる外部への Web ア
クセスは不可能である.このため,サーバに HTTP 中継サーバであるプロキシ
サーバを導入する.

\subsection*{アクセスコントロールリスト}
ルータのパケットフィルタリングによるアクセスの遮断は,アクセスコントロー
ルリストを用いて行う.
アクセスリストはパケットの送信元,送信先,プロトコルの種類などに基づい
て,どのパケットを通過させ,どのパケットを拒否するかを設定する.

\subsection*{HTTP プロキシサーバ}
クライアントPCが直接外部ネットワークにアクセスしないように,HTTPのアク
セスをプロキシサーバで中継させる.プロキシサーバはクライアントPCがアク
セスできる範囲で外部につながっているネットワークに設置し,クライアント
はHTTPでのアクセスの際にプロキシサーバを使うよう設定する.

\subsection*{使用するソフトウェア}
本実験では,ルータでのパケットフィルタリングにIOSのアクセスコントロール
リスト,プロキシサーバの構築に squid を用いる.

\subsubsection*{IOS}
Cisco製のルータに標準搭載されている制御ソフトウェアで,ルータの動作制御
は全てこのIOSが行っている.アクセスコントロールリストの設定もこのIOSに
対して行う.
\subsubsection*{squid}
HTTPのプロキシ機能を実装したフリーのソフトウェアである.FTP, SSLなどの
中継を行うこともできる.

\subsection*{考慮すべき点}
今回の実験を行うに当たっては以下のようなことについて考慮する必要がある.
\begin{itemize}
  \item \textbf{アクセスコントロールリスト}\\
         通過させてはいけないパケットはどんなものがあるか.\\
         提供しているサービス以外にも通過させなければならないパケットがないか.
  \item \textbf{プロキシ}\\
         アクセスを制限するホストはどうやって決定すればよいか.\\
         アプリケーションゲートウェイとしての機能の他にプロキシの機能はあるか.
\end{itemize}
