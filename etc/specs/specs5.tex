\section{第5回 メールサーバ}

電子メールを利用するには,SMTPサーバ,POPサーバ,MUAの3つが最低限必要で
ある.SMTPサーバはMTA(Mail Transfer Agent)とも呼ばれ,SMTP プロトコルを
用いてメールを受信し,宛先メールアドレスから自分が受け取るメールである
か否かを判断し,もし自分が受け取るメールであればメールスプールへ記憶,
そうでなければ,次に転送すべきサーバはどのサーバかを判断して,そのサー
バへメールを送信する.

ユーザはMUA(Mail User Agent)と呼ばれるソフトウェアを用いて,POPサーバに
メールの受信を命令する.POPサーバは\textbf{\Underline{ユーザの命令}}に
応じて,その\textbf{\Underline{ユーザが本物であるか認証}}を行い,本人の
確認ができたならばSMTPサーバに蓄えられているそのユーザ宛のメールをMUAに
送信する.

電子メールサービスを提供するには,\textbf{\Underline{メールサーバの構
    築}}とメールサーバの管理者による利用者への\textbf{\Underline{メール
    アドレスの発行}}が必要である.
SMTPサーバでは目的のSMTPサーバまでメールが転送できるように設定を行
い,POPサーバでは自ネットワーク内の全てのユーザそれぞれに宛てられ
た\textbf{\Underline{メールを受信できるように設定}}を行う.さらに,メー
ルを利用するクライアントにはメールサーバのホスト名やメールアドレスなど
の\textbf{\Underline{MUAの設定に必要な情報}}を通知しなければならない.

\subsubsection*{postfix}

postfix は,1999年に米 IBM トーマス J ワトソン研究所の Wietse Venema に
よって開発され,現在も精力的に開発が続けられているフリーの MTA であ
る.IBM Public License 1.0 で配布されている.最初の MTA として
は,1983 年に BSD UNIX とともに配布された sendmail がよく知られている
が,1999年代後半,sendmail のセキュリティホールの頻発や,時代にそぐわな
い設計の複雑さ,設定の困難さなどから,新たに相次いで開発された MTA の一
つである.

同様の目的で開発された qmail に比較して,sendmail との互換性がより高く,
そのまま置き換えて運用することができる.

\subsubsection*{qpopper}

qpopperはQualcomm社のPOP3サーバソフトウェアで,そもそもはBerkeley
popperを拡張したものである.また,RFC(Request For Comments)1939とRFC
2449を完全実装しており,全世界で広く用いられている.

\subsubsection*{Majordomo}

Majordomo は,1992年から世界中で広く使われてきたメーリングリスト管理ソ
フトである.Majordomo が開発された当時は,WWW サービスの仕組みはまだ一
般的ではなく,多数の人の間で共有される情報発信手段は,メーリングリスト
が一般的であった.メールベースによる自動的な登録・抹消その他のサービス
が行えることが特徴的で,現在も広く使われるソフトウェアの一つである.

\subsubsection*{考慮すべき点}
今回の実験を行うにあたり,以下のようなことについて考慮する必要がある.
\begin{itemize}
{\bf \item{電子メールサービス}}\\
電子メールはどのようにして届け先を判別するのか.\\
メールサーバは電子メールを送受信する上でどのような働きをするのか.\\
メールサーバを構築するのに他のソフトウェアにはどのようなものが存在し,どのような特徴があるのか.\\
メーリングリストはどのようにして多数のユーザへ配信されるのか.
  \item \textbf{メーリングリスト}\\
        どのような使い方が有効か.\\
         どんな機能があれば便利か.
\end{itemize}
