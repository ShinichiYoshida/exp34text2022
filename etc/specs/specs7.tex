\section{第7回 ファイルサーバ}
ファイルやディレクトリを公開し,\textbf{\Underline{ネットワーク経由で他
    の端末から利用できる状態}}にすることをファイル共有という.これによ
り,他のマシン上にあるファイルを,自分のマシン上のファイルと同じように
操作することが可能になる.このときファイルを公開しているマシンをファイル
サーバ,公開されたファイルを利用するマシンをクライアントという.  UNIX間
ではNFS(Network File System),WindowsではSMB(Server Message Block)によ
り,LANなどのネットワーク上でのファイル共有を実現している.

しかし,異なるOS間でのファイル共有サービスは提供されていない場合があ
る.これは,\textbf{\Underline{OSがファイル共有に使用するプロトコル
    群}}が異なっているためである.本実験では,FreeBSD-Linux間でのファイ
ル共有にはNFSを用いる.また,UNIXをサーバとして,ファイル共有プロトコル
が異なるWindows へのファイル共有サービスを提供する.これには,それぞれ
のOSで使用されるファイル共有プロトコルを実装するためのソフトウェアを用
いてファイル共有サービスを実現する.

\vspace{-1zh}
\subsection*{使用するソフトウェア}
\subsubsection*{NFS(Network File System)}
UNIX間でファイル共有を実現するサービスで,今回使用するFreeBSD,Linuxを
始め,ほとんどの UNIX 系 OS では OS に標準で添付されている.
\subsubsection*{Samba}
UNIXシステムをWindowsファイルサーバとして動作させるためのソフトウェアで
ある.日本Sambaユーザ会がオリジナルSambaに対して日本語取扱の改良・変
更 を加えたSamba日本語版とドキュメントを配布している.

\vspace{-1zh}
\subsection*{考慮すべき点}
今回の実験を行うに当たっては以下のようなことについて考慮する必要がある.
\begin{itemize}
  \item \textbf{Sambaのセキュリティモード}\\
Sambaのセキュリティモードにはいくつか種類がある.それぞれどのような特徴があるか.
\item \textbf{ファイル共有}\\
ファイル共有のメリットはどのようなものがあるか.\\
デメリットにはどのようなものがあるか.\\
UNIX / LinuxからWindowsのファイル共有を利用するにはどのような方法があるか.
\end{itemize}
