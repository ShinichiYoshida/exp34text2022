\section{第7回 ファイルサーバ応用課題}
前回の実験では, 各グループのLAN内でのファイル共有を行った.
応用課題では外部ネットワーク(実験環境では他グループのLAN)とのファイル共有を行う.
他のグループ(最低1グループ)と協力して,\textbf{\Underline{外部ネットワークと共有する設定}}を行い, 
FreeBSD(もしくはLinux)マシンを使い, 他のグループがNFSでエクスポートしたディレクトリがマウントできることを確認する.

また, 悪意ある第三者からの共有ファイルの改竄, 閲覧などを避ける必要がある.
\textbf{\Underline{ファイル共有を許可する範囲を限定}}することでこれらの危険性を減らすことができる.
実験では, エクスポートしたディレクトリに特定のネットワーク(もしくは端末)からのマウントのみ許可する設定を行い, マウントの許可と拒否が正しく行えることを確認する.

NFSサーバにおいて, ファイル共有を許可するネットワークの設定は, ファイル /etc/exports を編集することで行う.
FreeBSD では shareコマンドのオプションで設定する.

\subsection*{考慮すべき点}
今回の実験を行うに当たっては以下のようなことについて考慮する必要がある.
\begin{itemize}

\item \textbf{ファイル共有}\\
外部ネットワークとファイル共有するのはどのような場合が考えられるか.\\
それによりどのようなメリット・デメリットが生じるか.\\
外部ネットワークと共有する場合のサーバの管理方法\footnote{使用するディスク容量, ファイルに対する読み書きの権限の設定...etc}はどのように行うか.\\

\item \textbf{NFSによる外部ネットワークとのファイル共有}\\
クライアントからNFSサーバが共有しているディレクトリ名を知るにはどうすればよいか.

\end{itemize}
