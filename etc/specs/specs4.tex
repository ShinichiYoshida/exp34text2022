\section{第4回 DNSサーバ}
TCP/IPを用いたネットワークでは個々のネットワーク機器の識別にIPアドレス
と呼ばれる数字を用いる.この数字は32bit\footnote{IPv4の場
  合.IPv6では128bitに拡張される.}の2進数で定義されるが,通
常1byte(8bit)づつ10進数に直しピリオドで区切り表記する.しかし,数字によっ
てネットワーク機器の識別を行うことは人間にとっては難しく,覚えることも
困難である.そこで,IPアドレスに対しホスト名と呼ばれる名前を対応づけす
るDNS(ドメインネームシステム)が考案された.このサービスを利用すること
により,ユーザはTCP/IPネットワーク上で個々のネットワーク機器のIPアドレ
スを意識することなく,覚えやすいホスト名のみで通信を行うことができる.

DNSを提供するにはDNSサーバを導入する必要がある.DNSとはそもそ
も\textbf{\Underline{階層化されたシステム}}であり,自ネットワーク内
の\textbf{\Underline{ネットワーク機器の名前のみを管理}}し,外部のネット
ワークに対してはホスト名を管理しているDNSサーバで名前解決を行う必要があ
る.通常,これらの作業はクライアント側で意識することなく,DNSサーバ
が\textbf{\Underline{他の適切なネームサーバ}}に問い合わせることで他のの
ネットワークの名前解決まで行う.

\subsubsection*{BINDとは}
BIND(Berkeley Internet Name Domain)はカリフォルニア大学バークレイ校で
開発されたDNSサーバで現在もっともシェアのあるDNSサーバである。 現在で
はISC(Internet Software Consortium)で改良が進められおり,各種プラット
フォーム上で動作ができる.通常,BINDとはDNSサーバやツールなどのパッケー
ジのことである.

\subsubsection*{考慮すべき点}
今回の実験を行うにあたり,以下のようなことについて考慮する必要がある.
\begin{itemize}
{\bf \item{DNS}}\\
名前の解決方法にはどのようなものがあるのか.\\
DNSとはどのような仕組みで名前の解決を行うのか.\\
自分の管理外のホストに対しどのように名前の解決を行うのか.\\
DNSサーバを構築するには他にどのようなソフトが存在し,どのような特徴があるか.
\end{itemize}
