\section{第3回 パッケージのインストールとパッチ}
OSのインストールを行っただけの状態では,コンピュータを使用するため
の\textbf{\Underline{基本的な機能}}しか利用することができない.ネットワー
ク内にサービスを提供するためには,サービス毎に必要とするソフトウェアを
サーバに導入し,環境に応じた設定を行なわなければならない.ソフトウェア
の提供形態としては,既にコンパイルされた状態であるものと,ソースコード
を取得し自らの環境に合わせコンパイルを行わなければならないものがある.
どちらの形で提供されたソフトウェアでも,導入を行う際に
は\textbf{\Underline{サービスの用途に応じた設定}}を行う必要がある.また,
過去に導入したOSやソフトウェアにはバグやセキュリティーホール等と呼ばれ
る重大な欠陥がある場合がある.そのため,\textbf{\Underline{必要に応じパッ
    チを当てる}}等のアップデート作業を行わなければならない.

\subsubsection*{インストールするソフトウェアとその役割}
本実験で導入するソフトウェアとその役割との対応を表\ref{sp3:tab:software1}に示す.

\begin{table}[htbp]
\begin{center}
\caption{ソフトウェアとその役割}
\label{sp3:tab:software1}
\begin{tabular}{|c|c|}
\hline
ソフトウェア & 役割 \\ \hline
Bash (Bourne-Again Shell) & Bourne Shell 互換の GNU の高機能シェル \\ \hline
Windows XP ServicePack3 & Windows XPの各種アップデート \\
\hline
\end{tabular}
\end{center}
\end{table}

\subsubsection*{考慮すべき点}
今回の実験を行うにあたり,以下のようなことについて考慮する必要がある.
\begin{itemize}
{\bf \item{導入するソフトウェア}}\\
何をする事を目的として,どのような形態で提供されたものか.\\
ソフトウェアのライセンスはどうなっているのか.

{\bf \item{アップデート}}\\
どのようなバグやセキュリティホールを対象として行わなければならないのか.\\
アップデートを行わないとどのような問題が発生するか.
\end{itemize}
