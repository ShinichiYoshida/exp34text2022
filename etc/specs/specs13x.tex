\section{第13回 ファイアウォールの構築応用課題}

ポートスキャンとは,サーバがどのポートへのアクセスを受け付けているかを
調査するための手段の一つである.様々なポート番号に対してアクセスを試み,
応答があればそのポート番号へのアクセスを受け付けているということになる.
本来は管理者がネットワークやサーバのサービスの提供状況が意図した通りに
なっているか確認するために行うが,悪意ある者がネットワークを通じて侵入
を試みる際の初期調査手段としても使われることが多
い.\textbf{\Underline{アクセスを受け付けているポート番号が分かれば利用
    されているアプリケーションもほぼ特定できる}}ため,そのアプリケーショ
ンの脆弱性を利用した侵入の足がかりとして利用される.

こういったポートスキャンが行われていないかを\textbf{\Underline{通信ログ
    を監視}}して調査する.ポートスキャンをかけられている場合は同じアド
レスから多数のポートに対してアクセス要求が送信されていないか
を\textbf{\Underline{通信ログから判断}}する.不正なアクセスの兆候がある
場合,侵入や攻撃を防ぐため,一般的にはパケットの送信元を特定し,送信元
ネットワークの管理者に連絡をした上で,\textbf{\Underline{アクセス拒否な
    ど必要な措置}}を講じなければならない.ただし,必要な通信まで拒否し
てしまうような混乱を防ぐため,アクセス拒否などの\textbf{\Underline{対策
    は慎重に正しく講じなければならない}}.また,アクセス拒否は基本的に
ルータで行う.なお,ポートスキャンはサーバ以外のPCやルータも受ける可能
性があるので,ルータなど全てのパケットが通過する場所で監視するのが重要
である.ポートスキャン以外にも,大量の IP アドレスに対して特定のサービ
スが行われていないか調査するアドレススキャンもあり,こちらも同様に監視
を行う必要がある.

\subsection*{考慮すべき点}
今回の実験を行うに当たっては以下のようなことについて考慮する必要がある.
\begin{itemize}
  \item ポートスキャンにはどのような種類があるのか.
  \item ポートスキャンやアドレススキャンを受けた場合の対処はどのようにすべきか.
  \item どうすればポートスキャンをかけられても大丈夫な環境にできるのか.      
\end{itemize}
