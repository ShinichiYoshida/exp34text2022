\section{第11回 ネットワークアクセスの多様化}


\subsection*{無線LAN}
ノートPCなど移動できるようなネットワーク機器は,移動する度にケーブルを差し替えなければならない上,移動した先にLANケーブルがあるとは限らない.そのため,ケーブルがない場所や敷設が困難な場所にPCを移動してもLANに接続できるようにするため無線LANを導入する.

無線LANを使えるようにするには,無線アクセスポイントと呼ばれる有線LANとの接続装置を設置し,PCには無線LANカードを装着する.そしてアクセスポイントへの接続設定を行う.その際,電波の盗聴による情報漏えいを防ぐためには,必ずデータの暗号化を行う必要がある.
また,第三者による無線LANへのアクセスを防ぐため,無線LANのNICに割り当てられたMACアドレスをアクセスポイントに登録することで,そのNIC以外からの接続が行えないようにする.

\subsection*{使用する機器}
本実験では,リモートアクセスサービスで「NTT-ME MN128SOHO SL11」と「NTT DoCoMo Mobile Card P-in」を利用する.また,無線LANでは「BUFFALO AirStation WLA-L11G」と「BUFFALO AirStation WLI-PCM-L11GP」を利用する.
\subsubsection*{BUFFALO AirStation WLA-L11G}
IEEE802.11bの無線LAN規格に対応した無線LANのアクセスポイントである.
\subsubsection*{BUFFALO AirStation WLI-PCM-L11GP}
IEEE802.11bの無線LAN規格に対応した無線LANカードで,ノートPCなどに装着して利用する.

\subsection*{考慮すべき点}
今回の実験を行うに当たっては以下のようなことについて考慮する必要がある.
\begin{itemize}
  \item \textbf{無線LAN}\\
         電波の到達範囲\\
         利用者や利用するPCの管理
\end{itemize}
