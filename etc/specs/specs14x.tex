\section{第15回 LANの分割 --VLAN--}

サブネットなどのネットワーク分割ではネットワークはルータで区切られ,ハ
ブ,スイッチ,ケーブルなどは,それぞれ異なるネットワーク毎に必要であっ
た.また,ネットワークAに属するハブには,ネットワークBの機器を接続する
ことはできなかった.

VLAN を用いることで,同一のハブやケーブルの上に複数の異なる LAN を仮想
的に構築することができる.16ポートのハブのうち3ポートだけを,異な
るLANに割り当てたり,1本のケーブルに3つの LAN を載せることが可能となり,
多くのネットワークを構築しながら機器設置やケーブル敷設のコストを下げる
ことが可能となる.また,設定を変更するだけで,スイッチやポートの所属
LAN を変更できるので\textbf{\Underline{柔軟な LAN 構築}}が可能とな
る.

VLANは本来スイッチ(ハブ)の機能の一つであり,意図的にイーサネットフレー
ムの到達範囲を制限することで,LAN を分割することができる.つま
り\textbf{\Underline{ブロードキャストドメインも分割される}}.
VLAN にはポートVLAN とタグ VLAN があり,\textbf{\Underline{それぞれ別の
    目的で用いられる}}.
最近のコンピュータのイーサネットカードや,ルータなどでも タグVLAN機能が
実装されたものがある.

今回の実験は隣のグループと協力して行い,自グループのルータには隣のグルー
プのLANを,また隣のグループのルータには自グループの LAN を同居させる.
また,自グループのスイッチと隣のグループのスイッチをタグVLANにより接続
し,自グループの LAN を隣のグループまで延長し,また同じケーブルを使って
隣のグループの LAN も自グループまで延長する.

それぞれのグループにおいて,ポート1-12 は自グループ,ポート13-15 は隣の
グループ,ポート16は隣のグループのスイッチとの接続に用いる.
自グループのポート同士は常に通信可,自グループと隣のグループの間の通信
は一切行えないことを確認する.

\subsubsection*{考慮すべき点}
今回の実験を行うにあたり,以下のようなことについて考慮する必要がある.
\begin{itemize}
  {\bf \item{ポートVLANとタグVLANの違い,使い分け}}\\
  今回は,ポートVLAN,タグVLAN をそれぞれ用いたが,それぞれの違い,使用
  方法や利用場面の違いなどを考えて構築することが重要である.
\end{itemize}

\subsubsection*{応用課題}
ルータ,スイッチの設定を完全に初期状態(工場出荷状態)に戻す.

