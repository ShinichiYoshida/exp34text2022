\section{第8回 バックアップ応用課題}
システム管理者が変わる場合, 新たな管理者がその\textbf{\Underline{システ
    ムの状態を把握}}できなければ, 交代後のシステムの運用が円滑に行えな
くなる. このため, 新管理者への引継ぎの作業が必要になる.

管理者が交代する際, 前任の管理者からシステムの状態を直接聞くことができ
ないこともある. この場合はシステム構成・作業・設定などの記録がきちんと
残されていれば, 新管理者が今後システムを運用・管理してゆく上で助けとな
るはずである.

応用課題では, 最低限以下の点を満たすこと. 
\begin{itemize}
\item \textbf{元管理者(旧グループ)は新管理者(新グループ)にネットワークの構成・運用中のサービスなどの設定記録を渡す. }\\
  (記録は, 新管理者がそのネットワーク内の構成について, \underline{\textbf{読めば理解できる}}\footnote{新管理者は, これまで実験で使用してきたソフトウェアに関する知識はあるものとする}内容にすること. )\\
\item 最低限以下のことは伝えるものとする.
  \begin{itemize}
  \item 起動しているサービス.
  \item インストールされているソフトウェア.
  \item 各サービス,ソフトウェアの設定ファイル.
  \item 特殊なコンパイルオプションや設定.
  \item HDD の容量,使用状況,パーティション構成.
  \item ユーザアカウント.
  \item 管理者のログイン方法.
  \item ハードウェア,ソフトウェアのおかしい点,気になる挙動,気を付けた方が良い点.
  \item これまで起こった事故,不具合等.
  \item その他,安全に安定して運用するために伝えておくべき点.
  \end{itemize}

  \item \textbf{旧グループの環境に設定変更を行う. }\\
	(例として、新管理者のアカウント登録など. 必ず設定記録を残すこと. )\\
\end{itemize}

\subsection*{考慮すべき点}
今回の実験を行うに当たっては以下のようなことについて考慮する必要がある.
\begin{itemize}
  \item \textbf{引継ぎ作業}\\
         引継ぎの方法をどのようにするか. \\
         書類を作成する場合どのような情報を記載するのが適当か. \\
         管理者変更があった場合, 新管理者はどのような設定変更をするべきか.(必要があれば) \\
\end{itemize}
