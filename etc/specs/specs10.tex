\section{第10回 プリンタサーバ }
プリンタを用いて印刷を行う場合,クライアントからプリンタに対して印刷要
求を出し出力を得る.プリンタが個々のクライアントごとに用意されていれば
印刷は行える.しかし,会社や学校等での用途を想定した場合,設置面積・導
入コスト・管理コスト等を考えると同種の機器を同じ場所に複数台設置するの
は\textbf{\Underline{非効率的}}である.そのため,一つのプリンタを複数の
クライアントで共有して利用したい.

共有の方法として,一つのクライアントが自身に接続されたプリンタをローカ
ルプリンタとして管理しその他のクライアントはプリンタを管理しているクラ
イアントに対してネットワークを経由し印刷要求を出すローカルプリンタの共
有型と,プリンタサーバを用いクライアントの印刷要求はサーバが一旦全て受
け,印刷ジョブの調整を行いプリンタへの印刷要求をサーバが行うネットワー
クプリンタとがある.ローカルプリンタの共有型ではプリンタを管理するクラ
イアントに印刷が依存してしまうため,印刷を請け負うクライアントが電源が
切れていたり何らかの処理を行っているといった場合には全てのクライアント
が印刷を行えなくなってしまう・処理が遅くなる等の欠点がある.また,プリ
ンタと管理クライアントはパラレルケーブルやUSBケーブルで接続するため,設
置場所の制約を受ける場合があるが\textbf{\Underline{新たな導入コスト}}は
かからない場合が多いという利点がある.これに対しネットワークプリンタは
ローカルプリンタ共有型の印刷方法と違い\textbf{\Underline{明確なサーバ・
    クライアントモデルとして構成}}ができるためクライアントの状態に他の
クライアントからの印刷が依存しない利点がある.プリンタサーバは専用のハー
ドウェアで構成するものとPC等を利用しサーバを構築する方法とがある.どち
らにしても専用のハードウェアを用意しなくてはならず導入コストがかかって
しまう.

これらのプリンタ利用方法を\textbf{\Underline{ネットワークの規模・用途等に応じて適切に導入}}を行わなければならない.

\subsubsection*{使用する機器}
本実験ではEPSON社製プリンタESPER LASER LP-1800 を用いる.このプリンタはPSプリンタでは無い.
また,プリンタサーバとして他社製オプションのE-8550TNE を用いる.

\subsection*{考慮すべき点}
\begin{itemize}
\item 接続方法によって印刷データの流れはどのように変わっているか.
\item ローカルプリンタの共有はどのような方法で実現できるか.
\item 印刷データの解像度とプリンタの解像度とはどのようなものか.
\end{itemize}
