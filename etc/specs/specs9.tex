\section{第9回 DHCPサーバ}
TCP/IPネットワークでは, IPアドレスやネットマスク, デフォルトルート,
DNSサーバのIPアドレスなどを機器ごとに設定する必要がある.ネットワークが
小規模であれば, 管理者がIPアドレスなどの設定を個々に行って運用する場合
があり, また, ネットワークの構成に変更があってもIPアドレスなどの設定変
更にかかる労力は少ない.

しかし, ネットワークが大規模になると, 個々の機器の設定のために大きな負
担が生じることになる.ルータやサーバなどは, 静的IPアドレスに固定しないと
運用に支障をきたす場合があるが, サーバが提供するサービスを利用するクラ
イアントは静的IPアドレスを割り振らなくてもよいことが多い.

クライアントの台数はサーバの台数より多くなるのが一般的である.そこで,
IPアドレスの割り振りを自動で行うDHCP(Dynamic Host Configuration
Protocol)サーバを利用する.  DHCPはネットワーク上の機器に対して,
\textbf{\Underline{動的にIPアドレスなどの設定}}を行うためのプロトコルで
ある.このとき, 機器に割り振るIPアドレスなどの情報を通知する側をDHCPサー
バ, 通知されたIPアドレスなどの情報を受け取る側をDHCPクライアントと呼ぶ.
DHCPサーバでは, あらかじめ\textbf{\Underline{クライアントに割り振るIPア
    ドレスの範囲などを設定}}することができる.  DHCPクライアントはサーバ
から割り振られたIPアドレスなどの情報を基に, 自身のネットワーク設定を行
う.

このようにDHCPサーバはネットワークの運用においてIPアドレスの割り振りや
ネットワークの設定などを動的に行うことを目的として導入される.

\subsection*{使用するソフトウェア}
\subsubsection*{ISC版DHCPサーバ}
DHCPサーバソフトウェアのひとつで, Internet Software Consortium(ISC)が開発, バージョンアップを行っている.
DHCPサーバソフトウェアには, この他にWidely Integrated Distributed Environment(WIDE)が開発しているWIDE版DHCPサーバなど, 
多数のソフトウェアがある.
\subsection*{考慮すべき点}
今回の実験を行うに当たっては以下のようなことについて考慮する必要がある.
\begin{itemize}
  \item \textbf{DHCPサーバソフトウェアの種類}\\
	今回使用したISC版以外にDHCPサーバソフトウェアにはどのようなものがあるか.\\
	また, それぞれどのような特徴があるか.\\
	
  \item \textbf{DHCPサーバ}\\
	DHCPサーバとクライアント間では, どのような通信が行われるのか.\\
	どのような場合にDHCPサーバを導入するのが適当か.\\
	ネットワークの構成に応じて, どのような設定を行うのが適当か.\\
	導入後の運用ではどのような点に注意すべきか.\\
	
\end{itemize}
