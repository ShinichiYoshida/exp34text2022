\section{第3回 パッケージのインストールとパッチ 応用課題}
今回の応用課題で要求される内容は以下の3点である.
\begin{itemize}
\item グループ毎に\textbf{\Underline{必要と思われるソフトウェア}}を1つ
      FreeBSD上で\textbf{\Underline{適切に動作する環境}}を整えること.構
      築作業は,ソースコードからではなく,FreeBSD のパッケージシステムを
      用いたバイナリソフトウェアをインストールする.
\item FreeBSD 上で現在動作している全てのサービスの確認をし,
  それぞれがどのようなサービスであるか把握すること.
\item 現在,各OSにどのようなソフトウェアが導入されているか確認すること.
\item 現在,各OSにどのようなソフトウェアが起動しているか確認すること.
\end{itemize}

\subsubsection*{補足事項}

FreeBSD の package システムのコマンドは下記に示す.
\begin{description}
 \item[pkg\_add] パッケージの追加.
 \item[pkg\_info] インストールされているパッケージの情報の表示.
 \item[pkg\_delete] インストールされているパッケージの削除.
\end{description}

pkg\_add コマンドは,パッケージファイルをダウンロードして,pkg\_add ファ
イル名とすれば良いが,パッケージの依存関係などの問題を自動的に解決するた
めに下記のように用いると良い.

\begin{cli}
# pkg_add -r パッケージ名
\end{cli}

このようにすると,自動的にパッケージと,そのパッケージをインストールする
ために事前に必要なパッケージも自動的にダウンロード・インストールされる.

パッケージ名は,FreeBSD のパッケージが置いてある ftp サーバのディレクト
リ \texttt{/pub/FreeBSD/ports/i386/packages-8.0-release/Latest} のファイ
ル名(拡張子 .tgz 以外の部分)を見れば良い.例えば,emacs はパッケージ名 
emacs(別バージョンの emacs-21.3 は emacs21,emacs23 は emacs23 ,日本語
版 emacs である emcws は ja-emcws)であることが分かる.

本実験では,ftp サーバを 172.21.10.1 としているので,環境変数の設定をし
ておく(環境変数名 PACKAGEROOT, 変数値 ftp://172.21.10.1).

\subsubsection*{考慮すべき点}
今回の課題を行うにあたり,以下のようなことについて考慮する必要がある.
\begin{itemize}
{\bf \item{導入するソフトウェア}}\\
どのような機能を追加するためにソフトウェアを新たに導入するか.\\

{\bf \item{サービスの確認・修正}}\\
なぜ現在動作しているサービスを確認しておく必要があるか.\\
不要なサービスを動作させているとどのような問題が発生する可能性があるか.
\end{itemize}
