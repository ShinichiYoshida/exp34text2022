\section{第1回 OSのインストール}

パーソナルコンピュータ(PC)およびワークステーション(WS)は、一切ソフ
トウェアが入っていない状態では,ユーザに対して計算や通信といった機能を
提供することができない.本実験ではまず,これらPCやWSの機能を,ユーザ
が\textbf{\Underline{目的に応じて利用できるような環境}}を整える.

PCやWSはオペレーティングシステム(OS)と呼ばれる基本ソフトウェアを導入す
ることによって,ユーザからの指示を受け付け,その機能をユーザに提供する
ことができるようになる.OSはキーボード入力や画面出力といった入出力機能
やディスクやメモリの管理など,共通して利用される基本的な機能を提供し,
コンピュータシステム全体を管理するソフトウェアである.

\vspace{-1zh}
\subsubsection*{使用するOS}
本実験で使用するOSとインストール対象となるマシンの対応は表\ref{sp1:tab:osandcomp}に示す通りである.
% 表の挿入
\begin{table}[h]
 \caption{使用OSと対象コンピュータ}% {}内に表題を書く
 \label{sp1:tab:osandcomp}
 \begin{center}
  \begin{tabular}{|l|l|}
    \hline
     OS  &  対象コンピュータ  \\
    \hline
     FreeBSD 8.0-RELEASE  & ASUS RS-100   \\
    \hline
     Cent OS 5.2 (Linux)& DOS/V コンピュータ   \\
    \hline
     Windows XP Professional  & DOS/V コンピュータ   \\
    \hline
     Windows XP Home & Acer Aspire One   \\
    \hline
  \end{tabular}
 \end{center}
\end{table}

\vspace{-4zh}
\subsubsection*{考慮すべき点}
OSをインストールするに当たっては以下のようなことについて考慮する必要がある.
\begin{itemize}
  \item \textbf{OSの種類}\\
         OSには多くの種類があり,それぞれ得意とする用途や特徴,インストール対象となるコンピュータが異なっている.
         使用するコンピュータやその使用目的に応じて適切なOSを選択する必要がある.
  \item \textbf{コンピュータの種類}\\
         PCやWSのコンピュータにも多くの種類があり,構成部品や性能は千差万別である.コンピュータによっては特定のOSしか
         インストールできないものがあったり,OSに必要な性能を満たさない場合もある.
\end{itemize}
