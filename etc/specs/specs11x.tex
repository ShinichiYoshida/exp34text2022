\section{第11回 ネットワークアクセスの多様化 応用課題}
無線LAN,有線LANはそれぞれ\textbf{\Underline{規格上の通信速度}}が異なっている.
また,規格上で定められた通信速度は最大値であり,実際にはそれ以下の速度になる.
応用課題では,無線LAN,有線LANを使って,それぞれサーバからファイルのダウンロードを行い,その転送時間の違いから,\textbf{\Underline{各通信方式の通信速度を比較}}する.
ファイルサーバまたは,WWWサーバなどに適当なサイズ(数M〜数十MB程度)を持ったファイルを公開し,クライアントPCでそれぞれの通信手段を用いて同じファイルをダウンロードする.
ダウンロードにかかった時間とファイルサイズから実際の1秒間の通信速度(bit per second : bps)を求める.

\subsection*{結果比較}
それぞれの通信速度について以下のような表にまとめて比較してみること.
% 表の挿入
\begin{table}[h]
 \caption{各通信方式の通信速度}% {}内に表題を書く
 \begin{center}
  \begin{tabular}{|c|c|c|c|c|}
    \hline
     通信方式  &  規格上の通信速度  & ファイルサイズ & 転送時間 &  実際の通信速度 \\
    \hline
     有線LAN & & & & \\
     (FastEthernet)  &    &    &    &    \\
    \hline
     無線LAN & & & & \\
     (IEEE802.11b)  &    &    &    &    \\
    \hline
  \end{tabular}
 \end{center}
\end{table}

\subsection*{考慮すべき点}
今回の実験を行うに当たっては以下のようなことについて考慮する必要がある.
\begin{itemize}
  \item 規格上の通信速度はどう決定されているのか
  \item なぜ規格通りの通信速度が出ないのか
  \item どの程度の通信速度があればユーザにとって十分なのか
\end{itemize}
