\section{第12回 TCP/IPネットワークのルーティング 応用課題}

前回は,管理者による静的ルーティングの設定を行った.今回は,動的ルーティ
ングを行う.動的ルーティングには様々なプロトコルがあるが下記の3つがよく
用いられ,それぞれ使われる場面が異なる.

\begin{itemize}
\item RIP (Routing Information Protocol)
\item OSPF (Open Shortest Path First)
\item BGP (Border gateway protocol)
\end{itemize}

今回は,\textbf{\Underline{RIPが用いられるため}} ルータに,RIP の機能を
動作させる.

\subsection*{静的経路の削除}

まず,前回手動で設定した経路の削除を行う.Cisco 社の IOS システムでは,
アドミニストレイティブ・ディスタンス(Administrative Distance)という距
離が表\ref{sp12:tab:ad-distance}のように決まっていて,この距離が短い方が,経
路として優先される.
\begin{table}[h]
 \caption{アドミニストレイティブ・ディスタンス}% {}内に表題を書く
 \label{sp12:tab:ad-distance}
 \begin{center}
  \begin{tabular}{|l|l|}
    \hline
    ルーティングプロトコル & デフォルトディスタンス値\\
    \hline \hline
    Direct route(connected route) & 0\\
    \hline
    Static route & 1\\
    \hline
    External Border Gateway Protocol (BGP) & 20\\
    \hline
    OSPF & 110\\
    \hline
    Routing Information Protocol (RIP) & 120\\
    \hline
  \end{tabular}
 \end{center}
\end{table}

静的経路は,どの動的経路よりも優先されるため,この経路が存在している状
態では,RIPv2 が正しく動作しない.そこでまず,前回 route コマンドで入力
した静的経路の削除を行う.削除は,ip route コマンドを削除すればよい.

\subsection*{RIP version 2 の導入}

RIP には,version 1 と version 2 がある.version 1 は,ネットワークアド
レスの情報のみしか近隣ルータに広報することができず,サブネットマスク情
報を伝えることができない.このため,異なったサブネットマスクを利用する
ネットワークが1つでもある場面では,誤った経路を学習することがあり,用い
ることができない.ネットワークアドレスとそのサブネットマスクの両方を近
隣ルータに伝えられるよう改良したプロトコルが RIP version 2 である.

今回はこの RIPv2 の導入を行う.RIP の導入は,第\ref{ch:nwconfig}章と同様
に行えば良いが,router rip コマンドの後,network コマンドの他に,
``version 2'' というコマンドを入力する.

以上で RIP version 2 の設定は終了である.show ip route で正しいルーティ
ングテーブルが構成されているか確認すること.また,ping で各グループへの
通信が確保されていることを確認すること.
