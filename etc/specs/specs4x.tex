\section{第4回 DNSサーバ 応用課題}
DNSによる名前解決を用いるネットワークでは,\textbf{\Underline{何らかの
    原因}}によりDNSサーバが停止するとそれ以後ホスト名による通信が不可能
となる.常時稼働を求められるネットワークでは通信が行えなくなる問題を回
避するため,バックアップを行うDNSサーバを用意する必要がある.しかし,常
に最新のアドレスの対応表を保持するためには変更があるたびに手作業でバッ
クアップ用DNSサーバの対応表も書き換えねばならず繁雑であり,本来のDNSサー
バとバックアップ用DNSサーバのアドレス対応表が一致しなくなるなどの問題も
考えられる.そこでマスターとして設定されたDNSサーバか
ら\textbf{\Underline{自動的にアドレスの対応表を取得する}}セカンダ
リDNSサーバをバックアップ用DNSサーバとして用い,クライアント側にセカン
ダリDNSサーバも登録することによりネットワークのDNSの安定性を高くする.

hostsによる名前解決を用いるネットワークは,各端末のhostsファイルの更新が繁雑であるという
欠点はあるが,小規模なネットワークで名前解決を行うにはDNSサーバを構築する必要がなく手軽である
という利点がある.

\subsubsection*{セカンダリDNSサーバ構築}
セカンダリDNSサーバとは先にも述べたよう,マスターとなるDNSサーバから自動的に
ゾーンファイルを取得するDNSサーバである.バックアップ目的に使われることが多いが,
大規模ネットワークではDNSの負荷分散を目的とし複数台のセカンダリDNSサーバを構築し
端末のアドレス管理はマスターのDNSサーバが行うがクライアントからはセカンダリDNSサーバに
アクセスさせる等の利用法もある.
\begin{itemize}
{\bf \item{Cent OS での構築}}\\
Cent OS ではGUIによるDNSサーバ構築ができるようになっている.
FreeBSD で行ったような作業をせずにサーバの構築が可能だが,
{\bf もちろん Cent OS でも手作業によるインストールとエディタによる入力も可能である.}
どちらを選ぶかはシステム管理者の判断によるが,GUIによる設定を行った場合も最低限
どこにあるどのファイルをどのように変更し,どの実行ファイルが動いているのか等は
知っておくべきである.
{\bf \item{対象とするファイル}}\\
BINDの設定ファイルは基本的にnamed.confである.DNSサーバをセカンダリとして
構築する際もこのファイルを書き換えることになる.まず,typeの指定を行いマスターとなる
サーバを指定する.このとき,転送するゾーン名とセカンダリのゾーン名を一致させていなければ
転送が行えないので気を付ける必要がある.
\end{itemize}

\subsubsection*{考慮すべき点}
今回の実験を行うにあたり,以下のようなことについて考慮する必要がある.
\begin{itemize}
{\bf \item{hosts}}\\
どのような場合にhostsを用いるのが妥当であるか.
{\bf \item{セカンダリDNS}}\\
プライマリサーバとの同期はどのようなタイミングで行われているか.\\
プライマリサーバが停止した場合にセカンダリサーバはどのような挙動をするか.
\end{itemize}
