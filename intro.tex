%
%   はじめに
%        written by Shinichi Yoshida
%        April 13, 2018

昨今,人工知能(機械学習とビッグデータ処理)関連の技術の進展が目覚しく,
人間の行う職業がなくなりつつあるように見受けられる.例えば,自動運転技
術が進展することで,バスやトラック,タクシーの運転手などの職業が衰退す
ることも考えられる.エンジニア職も例外ではない.ネットワークエンジニア
は将来なくなる職業にとりあげられることもある.しかしながら,人工知能技
術は,明確なルールがあるものや,単調な作業などはすぐに置き換えることが
できるが,複雑なものになればなるほど,機械学習アルゴリズムでは学習がで
きない.一方,ネットワーク技術が様々な技術の集合体であり,場所によって
異なる形で構築がされるような複雑なシステムである.全ての機器が設計通り
に動いているのならばまだしも,停電や経年劣化,障害等で一部の機器が動作
停止や誤動作するようなこともある.こうした複雑かつ困難な状況に,即座に
柔軟に対処できるような AI はまだ見当たらない.このような観点から考える
と,ますます複雑になるネットワークシステムを支える技術者は,今後しばら
くは(少なくとも我々が職業を持って働く間は)なくなることはないであろう.

さらに言えば,Meta, Twitter, LINE などの SNS,Google, Dropbox,
Amazon, Microsoft 等の様々なサービスやクラウド技術など,インターネッ
トサービスが発展し,多くの人々が利用している.これらは,インターネット
のインフラが稼働していないと全く利用できない.すなわち,これらに障害が
起きると生活に支障が出るくらい,重要な生活のインフラストラクチャ(基盤)
になっていると言え,これらを支えることは現代では電気・エネルギーや水道,
運輸インフラを支えると同じくらい重要なことである.

また,セキュリティも重要な事項であり,他のインフラと異なり,インターネッ
トインフラはサイバー攻撃による影響がとても大きく,一つの企業や国の生活に
影響を与えるほどである.実際に,通常の軍隊と同様に,サイバー軍を創設し他
国への攻撃,他国からの攻撃に備える国も増えている.さらには,先ほどの人工
知能も人々のインターネットの利用が進み,さらには IoT (Internet of
Things) などで多くの機器がインターネット接続され,多くのデータが蓄積され
たことで,はじめて人工知能が構築可能になったとも言える.

以上のことから,インターネット技術を実際に体験し習得することは,今後の情
報技術者にとって重要なことであるのは間違いない.この実験では,こうしたイ
ンターネット技術を支える技術について,実際に構築をしながら,幅広い角度か
らネットワーク技術を学ぶ.本実験で扱う技術は,必ずしも最新のサービスでは
なく,今後も長く使われ続ける基礎技術に重点を置いている.UNIX系OS(Ubuntu
Linux,CentOS), Windows, Macintosh といった広く普及しているオペレーティ
ングシステムを用いて,それぞれのコンピュータのネットワーク設定ならびに,
サーバでのWWWやデータベースシステム,暗号化など,身近に使われているネッ
トワークサービスの構築を行う.さらに,ルーティング,ネットワーク設計,ス
イッチ設定,VLAN,セキュリティといったインターネットを支える基盤技術の構
築を,ネットワーク業界で広く使われる Cisco IOS を実際に用いて学ぶ.IOS 
は CCNA 試験など,ネットワーク業界で認められている資格試験にも使われてお
り,多くのベンダが IOS 互換のコマンド体系を採用していることからネットワー
クを学ぶのに最も適したシステムといっても良い.

本実験の内容は初学の者にとっては十分に複雑で難しいものであるかも知れな
い.しかし,本実験の内容はまだまだ基礎であり,この先にはさらに複雑で設
計の難易度の高いシステムが数多く存在している.ぜひ,この実験を通して基
礎技術,基盤技術を身に付け,将来,更なる発展技術の習得,および新技術,
新サービスの開発へと進んでいき,インターネットの発展に寄与して欲しいと
考える.

全ての専攻の学生が,このテキストと実験での経験をもとに,ネットワーク技術
の仕組みを理解し,様々な業界で活躍することを心より期待し,巻頭の言とする.

\begin{flushright}
 2022年4月13日

 情報学群実験3i・4C 教育担当スタッフ一同
\end{flushright}
