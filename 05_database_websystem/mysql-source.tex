\section{MySQL インストールの補足}

\subsection{インストール}

ここでは,\texttt{mysql-5-5-22} のインストールについて記す.

ソースコードからのインストール手順についての詳細は,
\texttt{mysql-5.5.22.tar.gz} に付属の \texttt{INSTALL-Source} にあるが,
冒頭の「4. Install the distribution」に書いてあるように,特に,Section
2.9 (より具体的な手順は 2.9.2)(less でサーチする時は,スラッシュ ``/'' キー
の後,``\textasciicircum 2.9''と打ち込み Enter キーをタイプするとその箇
所がすぐに表示される\footnote{「\textasciicircum」は行頭の意味の正規表現})
に従って進める.

まず,make を行うための Makefile を作成するため,cmake をインストールす
る\footnote{通常のソフトウェアでは,configure スクリプト(シェル /bin/sh
で動作するスクリプト)で Makefile を作成することが多いが,cmake は OS 依
存のより少ない Makefile 作成(環境にあったコンパイル手順作成)ソフトウェ
アである.}.

\begin{cli}
1. cmake-2.8.7.tar.Z をダウンロード・展開
2. 展開後のディレクトリで,configure,コンパイル,インストール
\end{cli}
(cmake の拡張子は .Z であるが,これは UNIX 初期からの圧縮形式である
compress 形式である\footnote{最も初期からあるものが,compress ``.Z'', 次
が gzip ``.gz'' で圧縮率が向上し, その後特許問題の回避や圧縮率のさらなる
向上を達成した bzip2 ``.bz2'' が開発され,現在最も新しいものは,XZ
``.xz'' 形式であり,最近の UNIX の tar コマンドであればどの形式でも自動
的に認識して伸長できる.古いものほど圧縮率は低いが,互換性が高く,計算
機の性能が低くても動作する.make や tar, gzip プログラムなど,それ自身が
ソフトウェアを構築するソフトの場合は,開発環境がまだ整備されていない環境
向けに,古い compress 系式で配布されることも多い(gzip がないから gzip 
をインストールしたいのに,.gz 系式で配布されたらどうしようもない).}

次に,\texttt{mysql-5.5.22} をダウンロード・展開し,そのディレクトリ下で
\texttt{INSTALL-SOURCE} の Section 2.9.2 の ``Preconfiguration Setup''
からの操作を行う(groupadd, useradd のオプションが Linux 用のため,
FreeBSD ではやや異なるので下記を参照).

\begin{cli}
# pw groupadd mysql
# pw useradd -g mysql -n mysql
\end{cli}

ここから先は,2.9.2 ``Preconfiguration Setup'' の通りで良い.

\textbf{注意!} make install の際,/usr/local/mysql 以下にインストールさ
れていることを確認する.

起動時に自動的に mysqld を実行させるためのスクリプト
\texttt{support-files/mysql.server} は,FreeBSD の場合,
\texttt{/etc/init.d/mysql.server}ではなく,
/texttt{/etc/rc.d/mysql.server} におく(テキスト p.122).

また,(14) にあるPATH の設定は,下記のようにする.
\begin{cli}
# setenv PATH /usr/local/mysql/bin:$PATH
 (↑exit するまで有効)
# vi ~/.cshrc
最下行の上記の setenv PATH の一文を追加しておくことで,
次回ログイン時にも自動で PATH が設定される.
\end{cli}
%$
なお,テキスト p.122 (13) のパスワードの設定は,「2.10.2. Securing the
Initial MySQL Accounts」の「Assigning root Account Passwords」にあるよう
に mysqld が起動している環境で下記のコマンドを発行すれば良い.

\begin{cli}
# mysqladmin -u root password "パスワード"
\end{cli}
