\section{目的}

中規模以上のネットワークを構築する際は,ネットワークをある程度の大きさご
とに分割する方が良い.この理由は下記の通りである.

IP アドレスは,ネットワーク部とホスト部に分かれており,同じネットワーク
部を有する IP アドレスの端末は,同じ LAN\footnote{「LAN」の定義について
は世界中でコンセンサスのとれている厳密な定義が存在するわけではなく,文脈
によって意味が多少異なるが,ここでの「LAN」は,一つのデータリンク層ネッ
トワーク,例えばイーサネットで接続されたネットワークという意味で用いる.
すなわち,ブロードキャスト(より正確には,L2のブロードキャストフレーム,
あるいはL3のリミテッド(ローカル)ブロードキャストパケット)が届く範囲を
意味する.ちなみに,ブロードキャストが届く範囲をブロードキャストド
メインと呼ぶ.ルータ(ルーティング)を用いずに直接通信できる範囲のネッ
トワークとも言える.} を構成する.同一LAN のコンピュータ同士で通信を行う
場合は,ARP (Address Resolution Protocol) を用いて,宛先 IP アドレスから,
宛先端末の MAC アドレスを取得し,イーサネットフレームを送信してパケット
を送る\footnote{データリンク層にイーサネットを用いない場合は,必ずしも 
ARP を行わない}.

異なる組織に属する端末は,お互い異なるネットワーク部を持つ IP アドレスを
持つので,1台以上のルータを介して通信を行う.しかし,大きな組織において
は,下記のような理由で,組織内で更に LAN を分割する方が良い.
\begin{enumerate}
 \item 管理上の理由によりネットワークを分割したい場合.例えば,セキュリ
       ティなどの管理ポリシーが異なる場合や,部署間で情報のやりとりに制
       限を設けたいため,通信トラフィックを分けたい場合.
 \item 物理的な距離の制限で,データリンク層に用いるプロトコルでは,距離
       が届かない場合(現在は,光ファイバやイーサネットのスイッチによ
       り,データリンク層の距離制限は事実上無い).
 \item SMB などのブロードキャストを用いるプロトコルが,大きな LAN 全体に
       送信されることが望ましくない場合.
\end{enumerate}

特に近年は,上記の (2) はイーサネットスイッチと長距離光イーサネットの普
及で必要がなくなり,1Gbps や 10Gbps の普及で帯域上の理由もあまりない.む
しろ,(1) のように異なる組織の通信がなるべく干渉しないようにしたい,組織
外からのあるいはセキュリティを強化したいなどの理由でサブネット分割を行う.
例えば,大きなネットワークで (3) を許すと,Windows などでは数百以上の PC 
やプリンタ,ファイルサーバなどが一度に見えるようになってしまい,管理上好
ましくない.

このような場合に,IPアドレスのホスト部のうち上位ビットの一部分を,ネット
ワーク部とみなし,複数の組織間でネットワーク部の異なる IP アドレスを割り
当てることで,複数のIPネットワークを作り,それらのネットワークをルータで
接続することで,複数のネットワークからなる IP ネットワークを構築する.も
とのネットワークアドレスから,複数のネットワークアドレスを作ることをサブ
ネット分割という.

インターネットは,ルータを介して異なるネットワーク部を持つ LAN を接続し
ていったものであり,正しくルーティングの設定を行うことで,世界中の全ての
LAN と接続され,世界中の全ての端末と相互に通信が可能になる.


\begin{table}[tb]
  \centering
  \caption{ホスト部のうちネットワーク部に使われるビット数とサブネットの大きさ(7ビット以降もホスト部のビットが2ビットになるまで続く)}
  \begin{tabular}{c|c|c}
    \hline
   ビット数 & サブネット数 & ホスト台数(元のネットワークに対する比率)\\
    \hline
   1 & 2 & 1/2 \\
    \hline
   2 & 4 & 1/4 \\
    \hline
   3 & 8 & 1/8 \\
    \hline
   4 & 16 & 1/16 \\
    \hline
   5 & 32 & 1/32 \\
    \hline
   6 & 64 & 1/64 \\
    \hline
  \end{tabular}
\end{table}

\section{TCP/IPネットワークのルーティング}
既に説明してきたように,IP ネットワークでは IP アドレスを全世界のコン
ピュータにユニークに割り振り,IP アドレスを送信元コンピュータから宛先コ
ンピュータまで,(宛先の)IP アドレスのみを用いて転送されていくことが基本であ
る.IPネットワークでは,データは IP パケットに分割され,このパケットが
多くの中継機器を介して,宛先 IP アドレスまで転送されていく.IP パケット
は,ヘッダとデータ(ペイロード)に分けられる.ヘッダは,郵便における宛
名・差出人・消印のようなもので,送信元から宛先まで転送される過程で用い
られる情報である.ヘッダの後に,実際に転送したいデータが含まれる.パケッ
トが宛先まで無事に到着すると,ヘッダは破棄され,データの部分がオペレー
ティングシステムに渡される.

途中のパケット中継機器では,隣接する中継機器から送信されたパケットを受
信し,宛先 IP アドレスを見て,次に転送すべき中継機器(次ホップ:Next
Hop)を隣接する中継機器から選択し,その機器へ送信する.この経路選択およ
びパケットの転送動作をルーティングと呼び,ルーティングを行う中継機器を
ルータと呼ぶ.

ルータが,IPアドレスから対応する次ホップを選択する際,IPのアドレス長
は,IPv4 で 32ビット,IPv6 では 128ビットあるため,全ての IP アドレスと
対応する次ホップを記憶することは現実的ではない.このため,IP アドレスは,
ネットワークアドレス部とホストアドレス部に分けて構成され,一つのまとまっ
た組織は,一つの共通するネットワークアドレスを持ったIPアドレス群を取得
し,その組織内の個々の端末に対して,ホストアドレスを変えながら IP アド
レスを割り振る.このようにすることで,パケットを中継するルータで
は,IP アドレス全てを見る必要はなく,ネットワークアドレス部のみを見て次
ホップを選択できる.ルータが記憶しているネットワークアドレス部とそれに
対応する次ホップの表を,ルーティングテーブルと呼ぶ.

\subsection{クラスフルルーティング}
IPv4 では,IP アドレスのネットワーク部とホスト部は,32ビットからなって
いる.これを1オクテット(8ビット)ずつに分け,第1オクテットから第4オク
テットまでをドットで区切り,それぞれのオクテットを10進数で表す (dotted
decimal notation).IP アドレスの第1オクテットの値により,IPアドレス
は5つのクラスに分けられる.クラスにより,IPアドレスのネットワーク部とホ
スト部の長さが決まる.
\begin{table}[tb]
  \centering
  \caption{IPアドレスのクラス(現在はクラスレスに移行しており,クラスの概
 念は歴史的なもので,現在はクラスの囚われずにネットワークが割り当てられる)}
  \label{tab:16:ipclass}	
  \begin{tabular}{l|c|c|c|l}
    \hline
    クラス & 第1オクテットの値 & ネットワーク部 & ホスト部 & \\
    \hline
    クラスA & 0-127 & 第1オクテット & 第2〜4オクテット & \\
    \hline
    クラスB & 128-191 & 第1・2オクテット & 第3・4オクテット & \\ 
    \hline
    クラスC & 192-223 & 第1〜3オクテット & 第4オクテット & \\
    \hline
    クラスD & 224-239 & N/A & 第1〜4オクテット & マルチキャストアドレス\\
    \hline
    クラスE & 240-255 & N/A & 第1〜4オクテット & 未割り当て\\
    \hline
  \end{tabular}

  \caption{(旧来のクラスフルネットワークにおける)各クラスのネットワーク数・収容可能ホスト数}
   \label{tab:16:numnethost}
  \begin{tabular}{l|l|l}
    \hline
    クラス & 最大ネットワーク数 & 1ネットワーク内の最大ホスト数 \\
    \hline
    クラスA & 128 & 16,777,214\\
    \hline
    クラスB & 16,384 & 65,534\\
    \hline
    クラスC & 2,097,152 & 254\\
    \hline
  \end{tabular}
\end{table}

IPアドレスのうちホスト部のビットがすべて0であるような IPアドレスをネッ
トワークアドレスと呼ぶ.ネットワークアドレスは,ネットワークそのものを
示すアドレスであり,通常ホストやルータのインターフェイスに付与すること
はない.

クラスに基づいて,ネットワーク部を決定しルーティングテーブルを構築する
方法をクラスフルルーティングと呼ぶ.

ルータのルーティング動作は以下の手順で行われる.

\begin{enumerate}
\item ルータのいずれかのインターフェイスからパケットが受信される
\item 受信したパケットの宛先IPアドレスを読み出す
\item 宛先IPアドレスとルーティングテーブルのあるエントリのネットマスク
  とを AND 演算し,その結果がそのエントリのネットワークアドレスと一致す
  るか調べる
\item 前項の操作をすべてのエントリについて行う
\item 一致したエントリのうち,最も長いネットマスクのエントリの経路を採用
      する.このエントリに書かれている,次ホップ
      ルータの IP アドレスを経路として採用する(Longest Match と呼ぶ)
\item 前項の操作で,等しいネットマスク長の経路が複数マッチした場合は,
  最も小さいメトリック(距離,またはコスト)の経路を採用する
\item 採用した経路の次ホップルータへパケットを送信する
\item 一致するエントリが1つもない場合は,すなわちパケットの宛先IPアドレ
      スへの経路をルータが知らない場合,ルータはパケットは破棄し,
      パケットの送信元アドレスに宛てて,ICMP(type 3:Destiation
      Unreachable) パケットを送信する.
\end{enumerate}

例えば,ルーティングテーブル上で,
\begin{itemize}
\item 宛先ネットワークアドレス:192.168.1.0
\item 宛先ネットマスク 255.255.255.0
\item 次ホップルータ 10.1.2.3
\end{itemize}
というエントリがあった時,192.168.1.13 を宛先 IP アドレスとするパケット
が受信されたら,ネットマスク 255.255.255.0 とその宛先 IP アドレスとを
AND 演算すると 192.168.1.0 となり,宛先ネットワークアドレスに一致するの
で,他に一致する経路が無ければ,この経路が採用される.

ネットワークアドレス 0.0.0.0,ネットマスク 0.0.0.0 のエントリには,全て
の IP アドレスが一致するが,ネットマスク長が 0 であるため最も優先順位が
低いエントリである.このエントリがある場合,他のエントリに該当することが
なかった全てのパケットに対して,この経路が採用される.これをデフォルトルー
ト,デフォルトゲートウェイなどと呼ぶ.

逆に,ネットワークアドレスとして例えば 192.168.2.33,ネットマスク
255.255.255.255 のエントリの場合,マスク長は最も長い 32 ビットであるから,
宛先IPアドレスとして 192.168.2.33 を持つパケットはこのエントリが最優先で
採用される.これは単一のホスト専用のルートのため,ホストルートと呼ぶ.

このようにして,インターネット上のすべての IP アドレスについて,適切に
次ホップルータを選択できるようにルーティングテーブルを構築する作業が,
ルータの基本的な運用作業である.

\subsection{クラスレスルーティング}
クラスフルルーティングに対して,ネットワークアドレスとサブネットマスクを
用いて,クラスの大きさに依存せずに任意の大きさのサブネットマスクを持った
ネットワークの経路情報を,ルーティングテーブルに設定する方法をクラスレス
ルーティングと呼ぶ.

関連する技術としては,1つの(クラスフルな)ネットワークのネットマスクを
延長し,ネットワークを複数に分割するサブネット化,そのネットマスクをネッ
トワーク毎に変化させる VLSM (Variable Length Subnet Mask),複数のネット
ワークを束ねる CIDR (Classless Inter-Domain Routing) \footnote{より小さ
なネットワークに分割するサブネットに対して,複数の(隣接する)小さな(ク
ラスC)ネットワークをまとめて大きなネットワークにすることからスーパーネッ
トと考えることもできる.ルーティングテーブルに現れるネットワークは大小様々
なものとなり,クラスの概念は最早ないことから,クラスレスルーティングと呼
ぶ}などがある.

\subsection{ルーティングテーブルの構築の種類}
ルーティングテーブルの構築には,大きく下記の2種類がある.
\begin{description}
\item[静的ルーティング] 管理者がルーティングエントリを入力するもので,ルー
	   ティングテーブルは管理者が設定した状態のまま運用される.再設
	   定しない限り,経路が変化することはない.「静的(static)」と
	   は,変化しないという意味である.ルーティング情報は通常,手動
	   (Manual)で入力する必要がある.
\item[動的ルーティング] ルーティングプロトコルを用いて,ルータが近隣の
  ルータとルーティング情報を交換し,随時ルーティングテーブルの更新を行う.
	   ルーティングテーブルはネットワークの変化に応じて,随時更新さ
	   れ変化していく.すなわちある経路が遮断されたら,ルーティング
	   テーブルの経路が,別のルータを経由するように変更されるなどの
	   処理が行われる.管理者は,近隣のルータの管理者と,用いるルー
	   ティングプロトコルや設定を調整し,動的ルーティングプロトコル
	   を用いる設定を行う.近隣のルータ同士で同一のルーティングプロ
	   トコルを用いることで,他のルータからルーティング情報を取得し,
	   ルーティングテーブルは自動(Automatic)で更新する.
\end{description}

%\subsection{静的ルーティング}
静的ルーティングは,ネットワーク管理者がルーティングテーブルを構築する方
法である.また,ルーティングテーブルが自動的に変更されことがないため,経
路が不安定になることがない.ネットワーク接続の構成が変化せず,管理者が全
ての経路情報を管理できる場合は,あらかじめ必要な経路を全て静的に設定する
ことで,安定した通信を行うことができる.欠点として,ネットワーク接続の状
態が変化した場合,管理者が新たな経路情報を設定するまで,ルーティングテー
ブルは変化しないため,通信に障害が出るなどの不具合が出る場合がある.

ルーティングの登録に必要となる情報は,宛先ネットワークと次ホップルータの 
IP アドレスである.宛先ネットワークは,ネットワークアドレスとネットマス
クで与える.次ホップルータの IP アドレスは,自ルータのインターフェイスの
いずれか1つと同じネットワークに所属している必要がある.

動的ルーティングでは,管理者がどのルーティングプロトコルを用いるか
(OSPF, RIPなど\footnote{RIPには,RIPv1, RIPv2 があり,他にも旧来の
IS-IS, シスコ社製品でのみ用いる IGRP, EIGRP,インターネット全般の経路制
御用の BGP (BGP4) などがある})を決め,参加するネットワークをどのように
するかやパラメータなどを設定しておく.

\section{実験内容 (1)}
本章において必要となる作業は下記である.

\begin{itemize}
\item 現状のネットワークの確認
  \begin{itemize}
      \item traceroute コマンドにて,他のグループのサーバやwww.yahoo.co.jpなどへの経路を確認する (WindowsのPowerShellまたはコマンドプロンプトcmdなら tracert)
      \item Windows のエクスプローラー (explorer) で「ネットワーク」をクリックし他のコンピュータの状況を確認
  \end{itemize}
\item ルータの設置によるL3ネットワーク分割.各グループとバックボーンとを接続するルータで,静的にルーティングテーブルを構築し,
自グループと他グループおよびバックボーンと通信が行えるようにする.
  \begin{itemize}
    \item ルータへのコンソール接続
      \begin{itemize}
        \item コンソール(CONSOLE)ケーブル(水色)を,片方の Dsub 9ピンの形状のコネクタをサーバのシリアル端子(RS-232C端子,Dsub 9ピン)に,RJ45コネクタ(LANと同一の形状)をルータのCONSOLE端子に接続する.
        \item サーバから cu コマンドで接続する.
        \item シリアル端子の無いノートパソコン等から接続する場合は,RS-232C-USB変換コネクタを使いコンソール接続をする(本実験では用いない).
      \end{itemize}
    \item ルータの初期設定
    \item サブネット(ネットワークアドレス,ネットマスク)の決定,IPアドレスの決定
    \item ルータへのIPアドレス付与
    \item ルータでのルーティングテーブルの作成
  \end{itemize}
\item ネットワークアドレス変更に伴う各PCのネットワーク設定の変更
\item DNSのゾーン設定変更(IPアドレス変更)
\item Apache の IPアドレス制限の変更
\item Postfix の \texttt{mynetworks}の変更
\end{itemize}

現在,全端末が同一ネットワーク (192.168.0.X/24) となっている実験室全体のネットワークに対し
てサブネット分割を行って,各グループごとにネットワークを分割する.クラス
B のアドレスである 172.21.0.0/16 を256個のサブネットに分割し,表
\ref{tab:16:subnet}のように,各ネットワークに割り当てる.また,グループ内ネットワークでのIPアドレスは図\ref{fig:16:ipadress}のように割り当てる.

\begin{figure}[tb]
\begin{center}
 \includegraphics[width=15cm]{\chaprouter router_before.pdf}
 \caption{既に構築されているネットワーク.1台のインターネット接続ルータの配下に,L2スイッチのみでネットワークが構築されている.表中のYは,グループ番号が10の位となるアドレス.}
\end{center} 
\end{figure}

\begin{figure}[tb]
\begin{center}
 \includegraphics[width=15cm]{\chaprouter router_after.pdf}
 \caption{サブネット化後のネットワーク.ルータを各グループごとに配置し,FE0 (インターフェース fasterEthernet0) はバックボーン(外側=インターネット側),FE1 はグループスイッチに接続する.各グループの IP (GX IP) は,指示されたものを使う.}
\end{center} 
\end{figure}

\begin{table}[tb]
    \centering
    \caption{サブネット割り当ては,\textbf{172.21.0.0/16 を1024分割}したサブネットを下のように各班で用いる.}
    \label{tab:16:subnet}
    \begin{tabular}{c|c}
        \hline
         Group & Subnet \\
         \hline
         Group 1 & 分割したサブネットで12番目に小さいもの\\
         Group 2 & 分割したサブネットで13番目に小さいもの\\
         Group 3 & 分割したサブネットで14番目に小さいもの\\
         Group 4 & 分割したサブネットで15番目に小さいもの\\
         Group 5 & 分割したサブネットで16番目に小さいもの\\
         Group 6 & 分割したサブネットで17番目に小さいもの\\
         Group 7 & 分割したサブネットで18番目に小さいもの\\
         Group 8 & 分割したサブネットで19番目に小さいもの\\
         Group 9 & 分割したサブネットで20番目に小さいもの\\
         Group 10 & 分割したサブネットで21番目に小さいもの\\
         Group 11 & 分割したサブネットで22番目に小さいもの\\
         Group 12 & 分割したサブネットで23番目に小さいもの\\
         \hline
    \end{tabular}
\end{table}

\begin{figure}[tb]
\begin{center}
 \includegraphics[width=11cm,bb=50 50 720 580]{\chaprouter router2.pdf}
 \caption{サブネット内のIP割り当て}
 \label{fig:16:ipadress}
\end{center} 
\end{figure}

これらのネットワークを構築し,各ルータで静的ルーティングテーブルの設定を
行い,正常に通信が行えるようにする.

\section{必要となる知識}


%%%%%%%%%%%%%%%%%%%%%%%%%%%%%%%%%%%%%%%%%%%%%%%%%%%% GEEEEEEEEEEEEEEE
%\subsubsection{コンソールケーブルの準備}
\subsection{コンソール}
ルータ・スイッチの設定は,コンソールを用いて行う.
コンソールとは,キーボードやディスプレイなどの,人間との入力・出力を行うデバイスであるが,ルータやスイッチなどの製品は,わずかなボタンやインジゲータしか備えておらず,初期設定をすべてこれらのボタンで行うのは難しい.また,ネットワークの設定が整い,他の端末と通信が行える環境ができるまでは,通信を行うこともできない.このような場面で初期設定を行うのに用いられるものが,コンソールである.

ここでは,シリアルコンソールと呼ばれる,RS-232C シリアル通信(無手順)を使用したコンソール(文字のみのキーボード入力・画面出力)を用いる.

パーソナルコンピュータのシリアル端子に接続して,PCの端末(ターミナル)画面をコンソール(画面表示先)として用いる.

シリアル端子は,RS-232C とも呼ばれるが,通信速度は,9600bps程度\footnote{8ビットコンピュータ等の遅いものでは,300bps, 2400bps,早いものでも 14400bps,28800bps,56Kbps 程度}と,現在の高速コンピュータに比較して遅いため,コンピュータに装備されていないことも多く,その場合は,USBシリアル,あるいは USB-RS232C などの変換ケーブルを使って接続する.Windows からは「COM」ポートとして\textbf{デバイスマネージャ}から確認でき,「COM1」,「COM2」など番号が付与されている(USBの場合,接続のたびに番号は可変のため,デバイスマネージャで確認する必要がある).

本実験では,Ubuntu Linux Server にシリアル端子があるため,これを用いる.

\subsection{Cisco ルータ,スイッチへの接続}
Ubuntu Server Linux のシリアル端子(Dsub 9ピン端子=RS-232C端子)に,コンソールケーブルを接続する(水色のケーブル).RJ-45 端子(LAN端子と同じ形状)の方は,ルータ (Cisco) のCONSOLEポートに接続する.

次に端末で,root 権限になり,下記のコマンドで接続する.
\begin{cli}
cu コマンドをインストール
(シリアル通信コマンド)

# apt install cu

シリアルポート(のデバイスファいる)に読み書き権限を付与
ポート名はttyS0であり,ファイル名は /dev/ttyS0

# chmod 777 /dev/ttyS0

以上でルータ設定のためのコンソール接続準備が終了.

■コンソール接続

# cu -l ttyS0

ルータの電源をON
(既に ON の場合は,Enter を1-2回か押してみて接続を確認)

■コンソール接続終了

端末で Enter を1-2回押してから
「~.」
と入力する.

(プロンプトを確認し,linux に戻っていることを確認)

\end{cli}


\subsection{Cisco IOS ルータの設定方法}
ルータの設定はCisco IOS上で行うことができる.
IOS (Internetworking Operating System) は,Cisco 社製のルータ・スイッチ製品
に用いられている OS であり,現在,多くのネットワーク機器において,IOS に
似た設定体系が,設定方法として採用されている.

ルータの初期状態では,パスーワード等は設定されていないため,
Enterでログインできる.

\begin{cli}
Would you like to enter initial configure dialog?
   → このメッセージが表示された場合は,
      初期設定メニュー(ダイアログ)に入るか否かを確認しており,
      これからマニュアル(手動)で設定するため,
      [no] を選択する (「no」と入力しEnter).

     : 起動メッセージが表示される
     :
Press RETURN to get started!

  → ここで,[Enter] を入力

Router#
(↑プロンプト:入力コマンド待ち状態)
(上記メッセージが表示されず,直接プロンプトが出る場合もある)
\end{cli}

ここから,ルータのユーザ名,パスワードを設定し,cisco ユーザの削除,ホスト名の設定等を行うことができる.
詳しくは巻末付録のIOSコマンド集やHELPを参照すること.
以下は,ユーザとホスト名の管理およびIPアドレスの設定例である.

また,それに先だって,Cisco IOS ではデフォルトとなっている DNS の逆引き
設定,および,ログ表示とコマンド入力が混乱しないように,以下の2つの設定
も有効にしておく.
\clearpage

\begin{cli}

Router>
(この状態はユーザモードで設定はできない)

管理者になる (イネーブル)

Router> enable
(TABで補完可.「en」と省略可)

Router#

Router# configure terminal                                 ←(設定モードに入る)
  (IOS では,[Tab] 補完が可能.
   例えば,「conf」まで入力し[TAB],「t」だけ入力し[TAB]など.
    「c」で[TAB]を入力すると,cで始まるコマンド群の説明が表示される.
   候補項目が複数で無ければ,[TAB]自体も省略可能で,
   「conf term」「conf t」などと入力することも可能
   途中まで入力し,「?」を入力することで,
   その場面で入力できる項目の HELP を表示できる)

Router(config)#no ip domain lookup                 ←(DNSの逆引き設定の無効化)
  (通常,コマンドを打ち間違えるとホスト名と認識され,
  DNS検索が始まるので,これを無効にする)
  
コンソールでは,入力中にメッセージが表示されると,打っていた文字が流れてしまうのが,そのまま打ち込み続ければ正しく処理される.

しかし,分かりにくいので,メッセージ表示後,再度打ち込んでいた文字を再表示させる設定を行う.
  
Router(config)#line con 0   
Router(config-line)#logging synchronous  ←(ログ表示とコマンド入力画面の分離)
Router(config-line)#exit       ←(console 画面の設定の終了)

次にユーザ・パスワードを追加

Router(config)#username exp password 0 root00

次に管理者パスワードを追加

Router(config)#enable password 0 root00

ホスト名設定

Router(config)#hostname routerX  (Xはグループ名)  

router12(config)#exit  (設定確認のため設定モードから抜ける)

router12#show running-config                                 ←(現在の設定を確認)
Building configuration...

Current configuration : XXX bytes
        :
        :
(入力した内容が設定に反映されていることを確認)

■パスワードの暗号化

このままでは,show run したときにパスワードが生で見えてしまうため,設定ファイルではハッシュ化して見えなくする.

router12#conf t
router12(config)#service password-encryption

show run して,再度設定内容を確認する
(パスワードがハッシュ化されていること)

■IP設定

FastEthernet0 (FE0), FastEthernet1 (FE1) にIPアドレス設定
まず,FE0, FE1 にケーブルを接続する.
FE0にはバックボーン(インターネットへ)のケーブル.
FE1はグループのスイッチへ接続.

router12#conf t

router12(config)#interface fastethernet 0                ← (I/Fの設定)
router12(config-if)#ip address IPアドレス  ネットマスク
router12(config-if)#no shutdown      ← インターフェースの有効化
router12(config-if)#exit

router12(config)#interface fastethernet 1                ← (I/Fの設定)
router12(config-if)#ip address IPアドレス  ネットマスク
router12(config-if)#no shutdown      ← インターフェースの有効化
router12(config-if)#exit

■遠隔ログインの設定

CONSOLEだけでなく,TCP/IP経由でリモートログインできるようにする.telnet と ssh が使えるがここでは,telnetを使う.

router12(config)#line vty 0 4
router12(config-line)#password 0 root00
router12(config-line)#login
router12(config-line)#exit


\end{cli}

\subsection{PCのIPアドレスを変更}

Server, Linux, Windows の IPアドレスを,新しいサブネットのものに変更する.

\begin{itemize}
    \item Server で cu 接続している場合は,一旦抜ける.
    \item Linux に telnet コマンドがなければ,apt install telnet にてインストール
    \item Windows のPowershell, コマンドプロンプトに telnet コマンドがなければ,「設定」→「アプリ」→「プログラムと機能(一番下)」→「Windowsの機能の有効化または無効化」→「Telnet Client」のチェックを入れる.
    \item Windows の端末の telnet ではなく,ここでは Putty を使って接続することを勧める(上記のtelnetコマンドはネットワーク管理で必要になることが多いので,入れておく).Putty では,接続時に,「Connection type:」を「SSH」でなく「Telnet」を選択する.
    \item IPアドレス・ネットマスクは新しいもの
    \item デフォルトゲートウェイはルータのグループ側IPアドレス
    \item DNSはそのまま
\end{itemize}


IP設定,遠隔ログイン設定が正しくされると,Linux, Mac, Windows の telnet コマンド (WindowsはPutty)で,ログインできるようになる.ただし,PCのIP設定の変更も終えている必要がある.

\begin{cli}
telnet接続の例
# telnet ルータのIPアドレス
\end{cli}

これが成功して以降は,cu コマンドでなく,telnet 接続して設定することが可能である.また,最大5人(0-4)まで同時にルータに接続して設定することができる.

\subsection{ルーティングテーブルの登録}
Cisco IOS上でルーティングテーブルを登録する例を示す.
まず,現在のルーティングテーブルを確認する.

\begin{cli}
router12#show ip route                                    ←(ルーティングテーブルを確認)
Codes: C - connected, S - static, R - RIP, M - mobile, B - BGP
       D - EIGRP, EX - EIGRP external, O - OSPF, IA - OSPF inter area
       N1 - OSPF NSSA external type 1, N2 - OSPF NSSA external type 2
       E1 - OSPF external type 1, E2 - OSPF external type 2
       i - IS-IS, L1 - IS-IS level-1, L2 - IS-IS level-2, ia - IS-IS inter area
       * - candidate default, U - per-user static route, o - ODR
       P - periodic downloaded static route

Gateway of last resort is not set

     172.21.0.0/24 is subnetted, 2 subnets
C       192.168.0.0 is directly connected, FastEthernet0
C       172.21.X.0 is directly connected, FastEthernet1
\end{cli}

経路が ``C'' と書かれた,直接接続されたネットワークのみになっていることを確認する.これは,そのルータが直接接続されている(ルーティング設定をする必要のない)ネットワークである.

次に,静的ルーティングを手動で追加する.追加は,設定モードに入り,ip route コマンドで追加する.

■ルーティング設定

(他の)グループへの経路を,そのグループのネットワークアドレス,ネットマスク,次ホップルータの情報をルーティングテーブルに登録して設定する.

\begin{cli}
router12# configure terminal                              ←(設定モードに入る)
Enter configuration commands, one per line.  End with CNTL/Z.
router12(config)#ip route ネットワークアドレス サブネットマスク 次ホップルータ
\end{cli}

これを,自分以外の全てのグループについて設定する.

\section{ルーティングの設定}
Cisco ルータを用いて,サブネット分割されたネットワークを構築し,ルーティングの設定を行う.

コンソール用のPCとルータとをコンソールケーブルで接続し,ルータのユーザ名およびパスワード,ホスト名などの設定を行う.
また,図\ref{fig:16:ipadress}や図\ref{tab:16:subnet}を参照し,
\underline{静的ルーティングが行うために必要な経路を考え},ルーティングテーブルを登録する.なお,グループは1から12まである.


\subsection{コンピュータのネットワーク設定の変更}

次に各コンピュータの IP アドレス,ネットマスク,デフォルトゲートウェイ,
DNS サーバ等,ネットワーク設定の情報を変更する.

Ubuntu server では,/etc/netplan 以下の YAML ファイルを編集し,変更後は,再起動を行う.

なお,実際にはミッションクリティカルなサーバ運用においては,netplan 編集後にダウンさせずに ifconfig で手動変更も行って,止めずに運用を続ける場合も多いが,ここ
では,編集ミスなどによる不具合の可能性も考え,念のため再起動を行い正常に起動するか確認する.

\section{動作確認}
経路の設定およびIPアドレスの再設定が終わったら,ping コマンドおよび,
traceroute (Windows は tracert) コマンドで実際に宛先ネットワークまで到達
できるか確認する.ネットワーク内のコンピュータから,以下の各ネットワーク
内のコンピュータへの到達性を確認する.

\begin{itemize}
  \item 他のグループのルータ,サーバ,Windows, Linux等
  \item 実験室のメインサーバ 192.168.0.1
  \item インターネット www.yahoo.co.jp
  \item Mac, Chrome からのグループ内PCに接続可能であること
\end{itemize}

上記すべての経路への ping, traceroute が成功すれば,終了である.もし,ping が失敗す
る場合は,原因を考察し設定を修正する必要がある.

\paragraph{Destination Unreachable エラーの場合}

Destination Host Unreachable(宛先ホスト到達不可能) あるいは
Destination Network Unreachable(宛先ネットワーク到達不可能)エラーが出
る場合,そのエラーがどの IP アドレスから来ているかを確認する.そ
の IP アドレスのルータにおいて,宛先の IP アドレスへの経路がルーティン
グテーブルに無いことが原因であるので,適切な経路を設定する.

もし,そのルータが自グループの管轄外である場合,適切な管理者に必要な経
路の設定を要請する.

\paragraph{エラーメッセージが出ない場合}

ping のパケットがどこかで消失している場合は,エラーが出ない.この場合は,
明確な原因が分からない.より詳しい原因を知るためには,traceroute コマン
ドの使用を検討する.traceroute コマンドは,Linux,Windows,Ciscoルータ
にはデフォルトでインストールされている.ただし,Windows では tracert と
いうコマンド名である.traceroute コマンドの詳細は付録を参照のこと.こ
の traceroute コマンドを使用することで,ネットワークのどの部分に不具合
があるかを推測することができる.

\section{考慮すべき点}
ルーティングは何のために必要であり,ルータはルーティングを行う際にパケットのどのような情報を見るのか,ルーティング先を決定するのにどのような情報を必要とするかを考える.

サブネット化を行い,L3 ネットワーク分割を行うことで,どのようなメリット
があるのか,またどのようなデメリットが生じるかを考え,適切なサブネット化
は,どのように行えば良いかを考慮する.

\clearpage